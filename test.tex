\documentclass{article}
\usepackage[utf8]{inputenc}
\usepackage[T2A]{fontenc}
\usepackage[russian]{babel}

% Математические пакеты
\usepackage{amssymb, amsmath}

% Пакеты для оформления
\usepackage[
    a5paper,
    left=15mm,
    right=15mm,
    top=10mm,       % Изменено с 0mm для нормального отступа
    bottom=15mm,
    footskip=5mm
]{geometry}
\usepackage{enumitem}
\usepackage{hyperref} % Всегда подключать последним!

\begin{document}

% Перенесено в преамбулу (рекомендуется)
\title{Множества и способы их задания}
\author{}
\date{}
\makeatletter
\renewcommand{\maketitle}{
  \begin{center}
    {\Large\mdseries % \mdseries - среднее (нежирное) начертание
     \@title \par}
    \vspace{0em}
  \end{center}
}
\makeatother
\maketitle

\section*{Что такое множество?}

\textbf{Множество} — это совокупность объектов, которые рассматриваются как единое целое. Эти объекты называются \textit{элементами множества}.

Примеры множеств:
\[
A = \{1, 2, 3\}, \quad B = \{\text{красный}, \text{зелёный}, \text{синий}\}
\]

Обозначение: если $x$ принадлежит множеству $A$, пишем $x \in A$. Если не принадлежит — $x \notin A$.

\section*{Способы задания множеств}

Существует два основных способа задания множеств:

\begin{enumerate}[label=\arabic*)]
  \item \textbf{Перечислением элементов} — когда мы явно указываем все элементы множества:
  \[
  A = \{2, 4, 6, 8\}
  \]
  Такой способ подходит, когда множество конечное и небольшое.

  \item \textbf{Указанием свойства (предиката)} — когда множество задаётся условием:
  \[
  B = \{x \in \mathbb{N} \mid x \text{ — чётное и } x \leq 10\}
  \]
  Здесь $\mathbb{N}$ — множество натуральных чисел. Значит, $B$ — это все чётные натуральные числа, не превосходящие 10.
\end{enumerate}

\section*{Подмножества и другие понятия}

Если все элементы множества $A$ входят в множество $B$, то $A$ называется \textbf{подмножеством} $B$:
\[
A \subseteq B
\]

Пример:
\[
\{1, 2\} \subseteq \{1, 2, 3\}
\]

\textbf{Пустое множество} — это множество, не содержащее ни одного элемента:
\[
\varnothing
\]

\section*{Мощность множества}

\textbf{Мощность множества} (или \textit{кардинальное число}) — это количество элементов в нём. Обозначается $|A|$.

Пример:
\[
A = \{a, b, c\} \Rightarrow |A| = 3
\]

\section*{Замечания}

\begin{itemize}[leftmargin=*]
  \item В математике \textbf{порядок элементов и повторы не имеют значения}:
  \[
  \{1, 2, 3\} = \{3, 1, 2, 2\}
  \]
  \item Главное — какие элементы входят в множество, а не как они записаны.
\end{itemize}

\section*{Источники}

\begin{itemize}
  \item Г.С. Михалев, \textit{Дискретная математика. Базовый курс для вузов}.
  \item Р. Джонсонбауг, \textit{Дискретная математика}, Pearson Education.
  \item \href{https://ru.wikipedia.org/wiki/Множество}{Википедия: Множество}
\end{itemize}

\end{document}