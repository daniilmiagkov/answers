\documentclass{article}
\usepackage[utf8]{inputenc}
\usepackage[T2A]{fontenc}
\usepackage[russian]{babel}

% Математические пакеты
\usepackage{amssymb, amsmath}

% Пакеты для оформления
\usepackage[
    a5paper,
    left=15mm,
    right=15mm,
    top=10mm,
    bottom=15mm,
    footskip=5mm
]{geometry}
\usepackage{enumitem}
\usepackage{hyperref}
\usepackage{tikz}

\begin{document}

\title{Отношение эквивалентности\\и классификация множеств}
\author{}
\date{}
\makeatletter
\renewcommand{\maketitle}{
  \begin{center}
    {\Large\mdseries\@title\par}
    \vspace{0.5em}
  \end{center}
}
\makeatother
\maketitle

\section*{4. Отношение эквивалентности и классификация множеств}

\subsection*{4.1. Определение отношения эквивалентности}

Пусть $A$ — множество. Бинарное отношение $R\subseteq A\times A$ называется \textbf{отношением эквивалентности}, если оно обладает тремя свойствами:
\begin{enumerate}[label=\arabic*)]
  \item \textbf{Рефлексивность:} 
    \[
      \forall a\in A\; (a,a)\in R.
    \]
  \item \textbf{Симметричность:} 
    \[
      \forall a,b\in A\; (a,b)\in R \;\Rightarrow\; (b,a)\in R.
    \]
  \item \textbf{Транзитивность:} 
    \[
      \forall a,b,c\in A\; \bigl((a,b)\in R \wedge (b,c)\in R\bigr) \;\Rightarrow\; (a,c)\in R.
    \]
\end{enumerate}

\subsection*{4.2. Классы эквивалентности}

Для каждого элемента $a\in A$ определим его \emph{класс эквивалентности}:
\[
  [a] = \{\,x\in A \mid (a,x)\in R\}.
\]
\begin{itemize}[leftmargin=*]
  \item Если $(a,b)\in R$, то $[a]=[b]$.
  \item Классы эквивалентности попарно не пересекаются:
    \[
      [a]\neq [b]\;\Longrightarrow\;[a]\cap [b]=\varnothing.
    \]
\end{itemize}

\subsection*{4.3. Фактор‑множество}

Множество всех классов эквивалентности обозначается
\[
  A / R = \{ \,[a] \mid a\in A\,\},
\]
и называется \textbf{фактор‑множеством} или \textbf{множество классов эквивалентности}. Имеет натуральное отображение:
\[
  \pi\colon A \;\longrightarrow\; A/R,\qquad \pi(a) = [a].
\]

\subsection*{4.4. Примеры}

\begin{enumerate}[label=\arabic*)]
  \item \textbf{Конгруэнция по модулю $n$} на $\mathbb{Z}$:
  \[
    a \equiv b \pmod{n}
    \quad\Longleftrightarrow\quad
    n \mid (a-b).
  \]
  Здесь класс $[a] = \{\,a + kn \mid k\in\mathbb{Z}\}$, а фактор‑множество $\mathbb{Z}/n\mathbb{Z}$ содержит $n$ классов: 
  \(\{[0],[1],\dots,[n-1]\}.\)

  \item \textbf{Равенство длины слов} над алфавитом $\Sigma$:
  \[
    u \sim v \;\Longleftrightarrow\; |u| = |v|.
  \]
  Класс $[u]$ — все слова фиксированной длины $|u|$.

  \item \textbf{Классификация точек плоскости} по цвету: две точки эквивалентны, если окрашены в один цвет. Фактор‑множество — набор всех используемых цветов.
\end{enumerate}

\subsection*{4.5. Графическое представление}

\begin{center}
\begin{tikzpicture}[scale=0.9]
  % универсум U
  \draw[thick] (-0.5,-0.5) rectangle (4.5,2.5) node[above left]{$U$};
  % классы эквивалентности
  \draw (1,1) circle (1cm) node at (0.3,1.8){$[a]$};
  \draw (3,1) circle (1cm) node at (2.3,1.8){$[b]$};
  % точки
  \node at (0.7,1) {$a$};
  \node at (1.2,0.6) {$x$};
  \node at (3.0,1) {$b$};
  \node at (2.6,0.6) {$y$};
\end{tikzpicture}

\end{center}

\subsection*{4.6. Классификация множеств}

Отношения эквивалентности дают естественный способ \textbf{классификации} элементов множества $A$:
\begin{itemize}[leftmargin=*]
  \item Каждый класс эквивалентности можно воспринимать как \emph{класс признаков} или \emph{категорию}.
  \item Фактор‑множество $A/R$ — это множество \emph{категорий}, упорядоченных произвольным образом.
  \item Если требуется работать не с элементами $A$, а лишь с их классами (например, вычислять «цвета», «остаток при делении» и т.\,д.), удобно перейти к $A/R$.
\end{itemize}

\subsection*{Источники}

\begin{itemize}
  \item Г.\,С. Михалев, \emph{Дискретная математика. Базовый курс для вузов}.
  \item Р. Джонсонбауг, \emph{Дискретная математика}, Pearson Education.
  \item В.\,Э. Пахомов, \emph{Введение в дискретную математику}.
  \item \href{https://ru.wikipedia.org/wiki/Класс_эквивалентности}{Википедия: Класс эквивалентности}
  \item \href{https://ru.wikipedia.org/wiki/Фактор-множество}{Википедия: Фактор‑множество}
\end{itemize}

\end{document}
