\documentclass[a5paper]{article}

\usepackage{fontspec}
\usepackage[russian]{babel}
\usepackage{amssymb, amsmath}
\usepackage[
    a5paper,
    left=15mm,
    right=15mm,
    top=10mm,
    bottom=15mm,
    footskip=5mm
]{geometry}
\usepackage{enumitem}
\usepackage{tikz}
\usepackage{amsthm}

% \usepackage{tocloft}  % Необязательно, но можно кастомизировать оглавление

% % Отключаем нумерацию всех секций
% \setcounter{secnumdepth}{0}

% % (Опционально) убираем место под номер в оглавлении
% \renewcommand{\cftsecpresnum}{}
% \renewcommand{\cftsecaftersnum}{}
% \renewcommand{\cftsecnumwidth}{0pt}

\usepackage[
    unicode=true,
    colorlinks=true,
    linkcolor=blue,
    citecolor=blue,
    urlcolor=blue
]{hyperref}

\setmainfont{Times New Roman}

\begin{document}

\tableofcontents

% Input all section files
\section{Дискретная математика}
\subsection{Что такое множество?}

\textbf{Множество} — это совокупность объектов, которые рассматриваются как единое целое. Эти объекты называются \textit{элементами множества}.

Примеры множеств:
\[
A = \{1, 2, 3\}, \quad B = \{\text{красный}, \text{зелёный}, \text{синий}\}
\]

Обозначение: если $x$ принадлежит множеству $A$, пишем $x \in A$. Если не принадлежит — $x \notin A$.

\subsubsection{Способы задания множеств}

Существует два основных способа задания множеств:

\begin{enumerate}[label=\arabic*)]
  \item \textbf{Перечислением элементов} — когда мы явно указываем все элементы множества:
  \[
  A = \{2, 4, 6, 8\}
  \]
  Такой способ подходит, когда множество конечное и небольшое.

  \item \textbf{Указанием свойства (предиката)} — когда множество задаётся условием:
  \[
  B = \{x \in \mathbb{N} \mid x \text{ — чётное и } x \leq 10\}
  \]
  Здесь $\mathbb{N}$ — множество натуральных чисел. Значит, $B$ — это все чётные натуральные числа, не превосходящие 10.
\end{enumerate}

\subsubsection{Подмножества и другие понятия}

Если все элементы множества $A$ входят в множество $B$, то $A$ называется \textbf{подмножеством} $B$:
\[
A \subseteq B
\]

Пример:
\[
\{1, 2\} \subseteq \{1, 2, 3\}
\]

\textbf{Пустое множество} — это множество, не содержащее ни одного элемента:
\[
\varnothing
\]

\subsubsection{Мощность множества}

\textbf{Мощность множества} (или \textit{кардинальное число}) — это количество элементов в нём. Обозначается $|A|$.

Пример:
\[
A = \{a, b, c\} \Rightarrow |A| = 3
\]

\subsubsection{Замечания}

\begin{itemize}[leftmargin=*]
  \item В математике \textbf{порядок элементов и повторы не имеют значения}:
  \[
  \{1, 2, 3\} = \{3, 1, 2, 2\}
  \]
  \item Главное — какие элементы входят в множество, а не как они записаны.
\end{itemize}

\subsubsection{Источники}

\begin{itemize}
  \item Г.С. Михалев, \textit{Дискретная математика. Базовый курс для вузов}.
  \item Р. Джонсонбауг, \textit{Дискретная математика}, Pearson Education.
  \item \href{https://ru.wikipedia.org/wiki/Множество}{Википедия: Множество}
\end{itemize}

\subsection{Диаграммы Венна}

\subsubsection{Определение и назначение}

\textbf{Диаграммы Венна} (иногда называемые диаграммами Эйлера–Венна) служат для наглядного изображения отношений между множествами: объединений, пересечений, разностей и дополнений.

\subsubsection{2.2. Основные операции}

\begin{enumerate}[label=\arabic*)]
  \item \textbf{Объединение}: \(A \cup B\) — все элементы, принадлежащие хотя бы одному из множеств.
  \item \textbf{Пересечение}: \(A \cap B\) — элементы, общие для обоих множеств.
  \item \textbf{Разность}: \(A \setminus B\) — элементы из \(A\), не входящие в \(B\).
  \item \textbf{Дополнение}: \(\overline{A}\) — все элементы универсального множества \(U\), не входящие в \(A\).
\end{enumerate}

\subsubsection{2.3. Примеры диаграмм}

\begin{center}
\begin{tikzpicture}[scale=0.8]
  % Объединение A ∪ B
  \fill[blue!20] (0,0) circle (1.2);
  \fill[blue!20] (2,0) circle (1.2);
  \draw (0,0) circle (1.2) node[left]{$A$};
  \draw (2,0) circle (1.2) node[right]{$B$};
  \node at (1,0) {$(A\cup B)$};
  \node at (1,-1.8) {Объединение};
\end{tikzpicture}

\vspace{1em}

\begin{tikzpicture}[shift={(0,-4)},scale=0.8]
  % Пересечение A ∩ B
  \fill[red!20] (0,0) circle (1.2);
  \fill[white] (0:0.6) circle (0); % placeholder
  \begin{scope}
    \clip (0,0) circle (1.2);
    \fill[red!20] (2,0) circle (1.2);
  \end{scope}
  \draw (0,0) circle (1.2) node[left]{$A$};
  \draw (2,0) circle (1.2) node[right]{$B$};
  \node at (1,0) {$(A\cap B)$};
  \node at (1,-1.8) {Пересечение};
\end{tikzpicture}

\vspace{1em}

\begin{tikzpicture}[shift={(0,-8)},scale=0.8]
  % Разность A \ B
  \fill[green!20] (0,0) circle (1.2);
  \begin{scope}
    \clip (2,0) circle (1.2);
    \fill[white] (0,0) circle (1.2);
  \end{scope}
  \draw (0,0) circle (1.2) node[left]{$A$};
  \draw (2,0) circle (1.2) node[right]{$B$};
  \node at (1,0) {$(A\setminus B)$};
  \node at (1,-1.8) {Разность};
\end{tikzpicture}

\vspace{1em}

\begin{tikzpicture}[shift={(0,-12)},scale=0.8]
  % Дополнение ¬A
  \fill[gray!20] (-2,0) rectangle (4,2);
  \begin{scope}
    \clip (-2,0) rectangle (4,2);
    \fill[white] (1,1) circle (1.2);
  \end{scope}
  \draw (1,1) circle (1.2) node{$A$};
  \node at (1,-0.2) {$\overline{A}$};
  \node at (1,-1.8) {Дополнение};
\end{tikzpicture}
\end{center}

\subsubsection{2.4. Свойства}

\begin{enumerate}[label=\arabic*)]
  \item Ассоциативность:
    \[
      (A\cup B)\cup C = A\cup(B\cup C), 
      \quad (A\cap B)\cap C = A\cap(B\cap C).
    \]
  \item Коммутативность:
    \[
      A\cup B = B\cup A, 
      \quad A\cap B = B\cap A.
    \]
  \item Дистрибутивность:
    \[
      A\cap (B\cup C) = (A\cap B)\cup (A\cap C),
      \quad A\cup (B\cap C) = (A\cup B)\cap (A\cup C).
    \]
  \item Законы де Моргана:
    \[
      \overline{A\cup B} = \overline{A}\cap\overline{B},
      \quad \overline{A\cap B} = \overline{A}\cup\overline{B}.
    \]
\end{enumerate}
\subsection{3. Отношения и их свойства}

\subsubsection{3.1. Что такое отношение}

\textbf{Бинарное отношение} $R$ между двумя множествами $A$ и $B$ — это множество упорядоченных пар:
\[
R \subseteq A \times B,
\]
где $A \times B$ — декартово произведение:
\[
A \times B = \{ (a, b) \mid a \in A,\, b \in B \}.
\]

Если $(a, b) \in R$, то говорят, что \textit{$a$ связано с $b$} отношением $R$, и пишут $a\,R\,b$.

\subsubsection{3.2. Примеры}

\begin{itemize}[leftmargin=*]
  \item Отношение \textbf{«меньше»} на $\mathbb{N}$: $R = \{(a, b) \mid a < b\}$.
  \item Отношение \textbf{«быть делителем»} на $\mathbb{N}$: $R = \{(a, b) \mid a \mid b\}$.
  \item Отношение \textbf{«равенство по модулю»} на $\mathbb{Z}$: $a \equiv b \pmod{n}$.
\end{itemize}

\subsubsection{3.3. Область и область значений}

\begin{itemize}[leftmargin=*]
  \item \textbf{Область определения (domain)}:
  \[
  \operatorname{dom}(R) = \{ a \in A \mid \exists b \in B\colon (a,b) \in R \}.
  \]
  \item \textbf{Область значений (range)}:
  \[
  \operatorname{ran}(R) = \{ b \in B \mid \exists a \in A\colon (a,b) \in R \}.
  \]
\end{itemize}

\subsubsection{3.4. Свойства бинарных отношений (на $A \times A$)}

Пусть $R \subseteq A \times A$. Тогда отношение может обладать следующими свойствами:

\begin{itemize}[leftmargin=*]
  \item \textbf{Рефлексивность:}
  \[
  \forall a \in A\colon (a,a) \in R.
  \]
  Пример: $=$, $\le$.

  \item \textbf{Антирефлексивность (иррефлексивность):}
  \[
  \forall a \in A\colon (a,a) \notin R.
  \]
  Пример: $<$.

  \item \textbf{Симметричность:}
  \[
  \forall a,b \in A\colon (a,b) \in R \Rightarrow (b,a) \in R.
  \]
  Пример: «$a$ и $b$ живут в одном доме».

  \item \textbf{Антисимметричность:}
  \[
  \forall a,b \in A\colon (a,b)\in R \wedge (b,a)\in R \Rightarrow a = b.
  \]
  Пример: $\le$.

  \item \textbf{Транзитивность:}
  \[
  \forall a,b,c \in A\colon (a,b)\in R \wedge (b,c)\in R \Rightarrow (a,c)\in R.
  \]
  Пример: $\le$, $<$.
\end{itemize}

\subsubsection{3.5. Особые классы отношений}

\begin{itemize}[leftmargin=*]
  \item \textbf{Отношение эквивалентности} — рефлексивное, симметричное и транзитивное.  
  Пример: $a \equiv b \pmod{n}$.

  Такое отношение разбивает множество $A$ на \textit{классы эквивалентности}.

  \item \textbf{Отношение частичного порядка} — рефлексивное, антисимметричное и транзитивное.  
  Пример: $\le$ на $\mathbb{N}$.

  Если дополнительно выполняется, что любые два элемента сравнимы, то это \textbf{полный порядок}.
\end{itemize}

\subsubsection{3.6. Графическое представление}

Бинарное отношение на множестве $A$ можно представить в виде \textbf{ориентированного графа}:

\begin{itemize}[leftmargin=*]
  \item Вершины соответствуют элементам $A$.
  \item Направленное ребро $a \to b$ рисуется, если $(a,b) \in R$.
\end{itemize}

Пример: на множестве $A = \{1, 2, 3\}$ отношение $R = \{(1,2), (2,3), (1,3)\}$ — транзитивное.

\subsubsection{3.7. Табличное представление}

Отношение $R$ на множестве $A = \{a_1, a_2, \dots, a_n\}$ можно представить в виде \textbf{таблицы}, где в ячейке на пересечении строки $i$ и столбца $j$ стоит $1$, если $(a_i, a_j) \in R$, и $0$ — иначе. Это называется \textbf{матрицей смежности}.

\subsubsection{Источники}

\begin{itemize}
  \item Г.С. Михалев, \textit{Дискретная математика}.
  \item Р. Джонсонбауг, \textit{Дискретная математика}, Pearson Education.
  \item \href{https://ru.wikipedia.org/wiki/Бинарное_отношение}{Википедия: Бинарное отношение}
\end{itemize}
\subsection{4. Отношение эквивалентности и классификация множеств}

\subsubsection{4.1. Что такое отношение эквивалентности?}

Отношение $R$ на множестве $A$ связывает между собой некоторые пары элементов.  
Мы называем его \emph{отношением эквивалентности}, если оно позволяет считать связанные элементы «равными» по какому‑то признаку.  

Формально $R\subseteq A\times A$ удовлетворяет трём ключевым свойствам:

\begin{enumerate}[label=\arabic*)]
  \item \textbf{Рефлексивность.}  
    Каждый элемент эквивалентен сам себе:
    \[
      \forall a\in A\quad (a,a)\in R.
    \]
    \emph{Пояснение:} это значит, что сравнивая элемент с самим собой, мы всегда получаем «да» — элемент всегда «равен» самому себе.

  \item \textbf{Симметричность.}  
    Если $a$ эквивалентен $b$, то и $b$ эквивалентен $a$:
    \[
      \forall a,b\in A\;\bigl((a,b)\in R \;\Rightarrow\; (b,a)\in R\bigr).
    \]
    \emph{Пояснение:} эквивалентность — взаимное отношение. Нельзя иметь «одностороннюю» равенство.

  \item \textbf{Транзитивность.}  
    Если $a$ эквивалентен $b$, а $b$ эквивалентен $c$, то $a$ эквивалентен $c$:
    \[
      \forall a,b,c\in A\;\bigl((a,b)\in R \wedge (b,c)\in R\bigr) \;\Rightarrow\; (a,c)\in R.
    \]
    \emph{Пояснение:} признак эквивалентности «передаётся» по цепочке.
\end{enumerate}

Без одного из этих свойств отношение нельзя назвать «эквивалентностью», потому что нарушится идея «равности» как симметричной и непротиворечивой связи.

\subsubsection{4.2. Классы эквивалентности: интуитивный смысл}

\paragraph{Идея.} Все элементы, которые попарно эквивалентны друг другу, можно «собрать в одну корзинку» — \emph{класс эквивалентности}.  

Для каждого $a\in A$ определим
\[
  [a] \;=\; \{\,x\in A \mid (a,x)\in R\}.
\]
\begin{itemize}[leftmargin=*]
  \item Если $b\in[a]$, то по симметричности и транзитивности получаем $[b]=[a]$.
  \item Если два класса не совпадают, то они не имеют общих элементов:
    \[
      [a]\neq[b]\;\Longrightarrow\;[a]\cap[b]=\varnothing.
    \]
\end{itemize}

Таким образом, классы эквивалентности \emph{разбивают} всё множество $A$ на непересекающиеся «группы равных элементов».

\subsubsection{4.3. Фактор‑множество и фактор‑отображение}

Обозначим множество всех таких классов:
\[
  A/R \;=\; \{\, [a] \mid a\in A\}.
\]
Это называется \emph{фактор‑множеством}.  
С ним связано естественное отображение
\[
  \pi: A \;\longrightarrow\; A/R,\qquad
  \pi(a) = [a].
\]
\begin{itemize}[leftmargin=*]
  \item $\pi$ «сводит» каждый элемент в его класс.
  \item $\pi$ является сюръекцией (покрывает все классы).
  \item Если $aRb$, то $\pi(a)=\pi(b)$, и наоборот.
\end{itemize}

\subsubsection{4.4. Развёрнутые примеры}

\begin{enumerate}[label=\arabic*)]
  \item \textbf{Конгруэнция по модулю $n$} на $\mathbb{Z}$.  
    Определение: 
    \[
      a \equiv b \pmod{n}
      \quad\Longleftrightarrow\quad
      n \mid (a-b).
    \]
    Проверим свойства:
    \begin{itemize}[leftmargin=*]
      \item Рефлексивность: $n\mid(a-a)=0$ всегда.
      \item Симметричность: если $n\mid(a-b)$, то $n\mid(b-a)$.
      \item Транзитивность: $n\mid(a-b)$ и $n\mid(b-c)$ даёт $n\mid(a-c)$.
    \end{itemize}
    Класс $[a]=\{a+kn \mid k\in\mathbb{Z}\}$.  
    Всего $n$ различных классов: $[0], [1],\dots,[n-1]$.

  \item \textbf{Равенство длины слов} над алфавитом $\Sigma$.  
    Правило: $u\sim v \iff |u|=|v|$.  
    \begin{itemize}[leftmargin=*]
      \item Все слова длины 3 формируют один класс $[u]$.
      \item В фактор‑множестве $\Sigma^*/{\sim}$ каждый класс соответствует конкретной длине.
    \end{itemize}

  \item \textbf{Цвет точек на плоскости.}  
    Определим отношение: две точки эквивалентны, если они имеют одинаковый цвет.  
    Тогда каждый цвет — это один класс; фактор‑множество — набор всех цветов.
\end{enumerate}

\subsubsection{4.5. Геометрическая иллюстрация}

\begin{center}
\begin{tikzpicture}[scale=0.9]
  % универсум U
  \draw[thick] (-0.5,-0.5) rectangle (4.5,3) node[above left]{$U$};
  % класс [a]
  \fill[blue!20] (1,1.2) circle (1cm);
  \node at (1,1.2) {$[a]$};
  % класс [b]
  \fill[red!20] (3,1.2) circle (1cm);
  \node at (3,1.2) {$[b]$};
  % примеры точек
  \node at (0.6,1.5) {$a,x$};  
  \node at (2.6,1.5) {$b,y$};
\end{tikzpicture}
\end{center}

Здесь каждый круг — класс эквивалентности, внутри него лежат все «равные» элементы.

\subsubsection{4.6. Зачем это нужно?}

\begin{itemize}[leftmargin=*]
  \item Упрощает работу: вместо множества элементов оперируем множеством классов.
  \item В алгебре: фактор‑группы, фактор‑кольца.
  \item В теории языков: выделение всех слов одинаковой длины, одинакового суффикса и т.\,д.
  \item В анализе данных: кластеризация, когда каждый кластер — класс эквивалентности по выбранному критерию.
\end{itemize}

\subsubsection{Источники и литература}

\begin{itemize}
  \item Г.\,С. Михалев, \emph{Дискретная математика. Базовый курс для вузов}.
  \item Р. Джонсонбауг, \emph{Дискретная математика}, Pearson Education.
  \item В.\,Э. Пахомов, \emph{Введение в дискретную математику}.
  \item \href{https://ru.wikipedia.org/wiki/Класс_эквивалентности}{Википедия: Класс эквивалентности}
  \item \href{https://ru.wikipedia.org/wiki/Фактор-множество}{Википедия: Фактор‑множество}
\end{itemize}
\subsection{5. Планарные графы}

\subsubsection{5.1. Определение}

\textbf{Планарным} называется неориентированный граф $G$ (множество вершин $V$ и ребёр $E$), который можно нарисовать на плоскости так, чтобы никакие два ребра не пересекались, кроме общих концов. Такое представление называется \emph{планарным вложением} графа.

\subsubsection{5.2. Примеры}

\begin{itemize}[leftmargin=*]
  \item Граф $K_4$ (полный граф на четырёх вершинах) является планарным.
  \item Графы $K_5$ и $K_{3,3}$ не являются планарными (теорема Куратовского, см.~ниже).
\end{itemize}

\subsubsubsection*{5.2.1. Планарный пример: $K_4$}

\begin{center}
\begin{tikzpicture}[scale=1, every node/.style={circle,draw,inner sep=1.5pt}]
  \node (1) at (90:1.2) {$1$};
  \node (2) at (210:1.2) {$2$};
  \node (3) at (330:1.2) {$3$};
  \node (4) at (0,0) {$4$};
  \foreach \u/\v in {1/2,1/3,1/4,2/3,2/4,3/4}
    \draw (\u) -- (\v);
\end{tikzpicture}

\small Рис. 1. Планарное вложение полного графа $K_4$.
\end{center}

\subsubsubsection*{5.2.2. Непланарный пример: $K_5$}

\begin{center}
\begin{tikzpicture}[scale=1, every node/.style={circle,draw,inner sep=1.5pt}]
  \node (1) at (90:1.5) {$1$};
  \node (2) at (162:1.5) {$2$};
  \node (3) at (234:1.5) {$3$};
  \node (4) at (306:1.5) {$4$};
  \node (5) at (18:1.5) {$5$};
  \foreach \u/\v in {1/2,1/3,1/4,1/5,2/3,2/4,2/5,3/4,3/5,4/5}
    \draw (\u) -- (\v);
\end{tikzpicture}

\small Рис. 2. Попытка вложения полного графа $K_5$ с неизбежными пересечениями.
\end{center}

\subsubsection{5.3. Формула Эйлера}

Для связного планарного графа справедлива \emph{формула Эйлера}:
\[
  V - E + F = 2,
\]
где $V = |V(G)|$ — число вершин, $E = |E(G)|$ — число ребер, а $F$ — число граней (областей плоскости, включая внешнюю).

\paragraph{Пример.} В графе $K_4$ имеем $V=4$, $E=6$. Рассчитаем $F$:
\[
  4 - 6 + F = 2 \;\Longrightarrow\; F = 4.
\]
Действительно, при планарном вложении мы получаем три внутренних треугольника и одну внешнюю область.

\subsubsection{5.4. Критерии планарности}

\begin{itemize}[leftmargin=*]
  \item \textbf{Теорема Куратовского:} Граф планарен тогда и только тогда, когда он не содержит подграфа, гомоморфного $K_5$ или $K_{3,3}$.
  \item \textbf{Теорема Вагнера:} Упрощённый критерий: нет миноров $K_5$ и $K_{3,3}$.
\end{itemize}

\subsubsection{5.5. Свойства и ограничения}

\begin{enumerate}[label=\arabic*)]
  \item Для простого планарного графа с $V\ge3$ всегда выполняется
  \[
    E \le 3V - 6.
  \]
  Если, кроме того, нет треугольников (циклов длины 3), то
  \[
    E \le 2V - 4.
  \]
  \item Минимальный непланарный граф имеет $V=5$, $E=10$ (граф $K_5$) или $V=6$, $E=9$ (граф $K_{3,3}$).
\end{enumerate}

\subsubsection{5.6. Применения}

\begin{itemize}[leftmargin=*]
  \item \emph{Географические карты}: раскраска областей так, чтобы соседние области различались цветом (теорема о четырёх красках).
  \item \emph{Схемотехника}: прокладка дорожек на печатных платах без пересечений.
  \item \emph{Графический дизайн}: автоматическая укладка элементов схем и диаграмм.
\end{itemize}

\subsubsection{Источники}

\begin{itemize}
  \item Д.Б.\,West, \emph{Introduction to Graph Theory}, Prentice Hall.
  \item В.\,Д.\,Мазурин, \emph{Дискретная математика: графы и алгоритмы}.
  \item \href{https://ru.wikipedia.org/wiki/Планарный_граф}{Википедия: Планарный граф}
  \item \href{https://ru.wikipedia.org/wiki/Теорема_Куратовского}{Википедия: Теорема Куратовского}
\end{itemize}
\subsection{6. Матрицы смежности и инцидентности}

\subsubsection{6.1. Граф и его представления}

Пусть задан простой неориентированный граф $G = (V, E)$, где:
\begin{itemize}[leftmargin=*]
  \item $V = \{v_1, v_2, \dots, v_n\}$ — множество вершин ($|V| = n$),
  \item $E = \{e_1, e_2, \dots, e_m\}$ — множество рёбер ($|E| = m$).
\end{itemize}

Для хранения и анализа структуры графа удобно использовать его представление в виде матриц:
\begin{enumerate}[label=\arabic*)]
  \item \textbf{Матрица смежности} (adjacency matrix),
  \item \textbf{Матрица инцидентности} (incidence matrix).
\end{enumerate}

\subsubsection{6.2. Матрица смежности}

Матрица смежности $A$ — это квадратная матрица $n \times n$, где:
\[
  a_{ij} = \begin{cases}
    1, & \text{если вершины } v_i \text{ и } v_j \text{ соединены ребром}, \\
    0, & \text{иначе}.
  \end{cases}
\]

\textbf{Свойства:}
\begin{itemize}[leftmargin=*]
  \item Для неориентированного графа $A$ симметрична: $a_{ij} = a_{ji}$.
  \item Диагональные элементы $a_{ii}$ равны 1, если в графе есть петли (в простом графе всегда 0).
  \item Сумма элементов $i$‑й строки (или столбца) даёт степень вершины $v_i$.
\end{itemize}

\textbf{Пример:} граф с $V = \{v_1, v_2, v_3\}$ и рёбрами $E = \{(v_1,v_2), (v_2,v_3)\}$:

\[
A =
\begin{pmatrix}
  0 & 1 & 0 \\
  1 & 0 & 1 \\
  0 & 1 & 0 \\
\end{pmatrix}
\]

\subsubsection{6.3. Матрица инцидентности}

Матрица инцидентности $B$ — это матрица $n \times m$, где:
\[
  b_{ij} = \begin{cases}
    1, & \text{если вершина } v_i \text{ инцидентна ребру } e_j, \\
    0, & \text{иначе}.
  \end{cases}
\]

\textbf{Особенности:}
\begin{itemize}[leftmargin=*]
  \item Каждое ребро соединяет две вершины, значит в столбце $j$ ровно два значения 1 (если граф простой и без петель).
  \item В ориентированном графе обычно используют $-1$ и $+1$:  
    \[
    b_{ij} = \begin{cases}
      -1, & \text{если } v_i \text{ — начало дуги } e_j, \\
      +1, & \text{если } v_i \text{ — конец дуги } e_j, \\
      0, & \text{иначе}.
    \end{cases}
    \]
\end{itemize}

\textbf{Пример:} тот же граф, где $e_1 = (v_1,v_2)$, $e_2 = (v_2,v_3)$:

\[
B =
\begin{pmatrix}
  1 & 0 \\
  1 & 1 \\
  0 & 1 \\
\end{pmatrix}
\]

\subsubsection{6.4. Сравнение представлений}

\begin{itemize}[leftmargin=*]
  \item Матрица смежности подходит для быстрого ответа на вопрос: «Есть ли ребро между $v_i$ и $v_j$?»
  \item Матрица инцидентности удобна для анализа структуры рёбер, особенно в ориентированных графах.
  \item Для разреженных графов (мало рёбер) матрица смежности неэффективна по памяти.
\end{itemize}

\subsubsection{6.5. Визуальный пример}

\begin{center}
\begin{tikzpicture}[scale=1.2, every node/.style={circle, draw}]
  \node (v1) at (0,1.5) {$v_1$};
  \node (v2) at (2,1.5) {$v_2$};
  \node (v3) at (1,0)   {$v_3$};

  \draw (v1) -- (v2) node[midway, above] {$e_1$};
  \draw (v2) -- (v3) node[midway, right] {$e_2$};
\end{tikzpicture}

\vspace{0.5em}
\small Рис. 1. Граф с вершинами $v_1, v_2, v_3$ и рёбрами $e_1, e_2$
\end{center}

\subsubsection{6.6. Применения}

\begin{itemize}[leftmargin=*]
  \item Алгоритмы поиска в графе (например, обход в глубину, поиск кратчайших путей).
  \item Сетевые задачи (анализ маршрутов, потоков, связности).
  \item Работа с графами в программировании, машинном обучении и обработке изображений.
\end{itemize}

\subsubsection{Источники}

\begin{itemize}
  \item Гросс, Йелл: \emph{Теория графов и её приложения}.
  \item Д.Б. Уэст, \emph{Введение в теорию графов}.
  \item \href{https://ru.wikipedia.org/wiki/Матрица_смежности}{Википедия: Матрица смежности}
  \item \href{https://ru.wikipedia.org/wiki/Матрица_инцидентности}{Википедия: Матрица инцидентности}
\end{itemize}
\subsection{7. Пути и контуры в графе}

\subsubsection{7.1. Основные определения}

Пусть задан неориентированный простой граф $G=(V,E)$.

\begin{itemize}[leftmargin=*]
  \item \textbf{Путь} (walk) в графе $G$ — это последовательность вершин
  \[
    P = (v_0, e_1, v_1, e_2, \dots, e_k, v_k),
  \]
  где каждое ребро $e_i = \{v_{i-1},v_i\}\in E$. Говорят, что путь ведёт из $v_0$ в $v_k$.
  \item \textbf{Длина пути} — число ребер на пути, равное $k$.
  \item \textbf{Начальная вершина} — $v_0$, \textbf{конечная вершина} — $v_k$.
  \item \textbf{Открытый путь} — начальная и конечная вершины различны ($v_0 \neq v_k$).
  \item \textbf{Замкнутый путь} — начальная и конечная вершины совпадают ($v_0 = v_k$).
\end{itemize}

\subsubsection{7.2. Простые пути и контуры}

\begin{enumerate}[label=\arabic*)]
  \item \textbf{Простой путь} — путь, в котором все вершины различны:
  \[
    v_i \neq v_j \quad\text{для }0\le i<j\le k.
  \]
  Простота гарантирует отсутствие «заходов в тупик» и повторов.
  \item \textbf{Контур} (cycle) или \textbf{простой замкнутый путь} — замкнутый простой путь длины $k\ge3$, в котором кроме совпадения $v_0=v_k$ все промежуточные вершины различны.
\end{enumerate}

\subsubsection{7.3. Специальные виды путей}

\begin{itemize}[leftmargin=*]
  \item \textbf{Тrail} — путь, в котором рёбра не повторяются, но вершины могут.
  \item \textbf{Цепь} (trail) и \textbf{цепь без повторов} (simple trail) в ориентированных графах аналогично.
  \item \textbf{Эйлеров путь} — путь, проходящий по каждому ребру ровно один раз. Если он замкнут, то это \emph{цикл Эйлера}.
  \item \textbf{Гамильтонов путь} — простой путь, проходящий через каждую вершину ровно один раз. Если он замкнут (возвращается в начальную вершину), то это \emph{цикл Гамильтона}.
\end{itemize}

\subsubsection{7.4. Примеры и иллюстрации}

\paragraph{Пример 1.} Простой путь длины 4 на графе:

\begin{center}
\begin{tikzpicture}[scale=1, every node/.style={circle,draw,inner sep=1.2pt}]
  \node (A) at (0,0) {$A$};
  \node (B) at (1.5,0) {$B$};
  \node (C) at (3,0) {$C$};
  \node (D) at (4.5,0) {$D$};
  \node (E) at (6,0) {$E$};
  \foreach \u/\v in {A/B,B/C,C/D,D/E}
    \draw (\u) -- (\v);
  \node at (3,-0.7) {Путь $P=(A,B,C,D,E)$, длина $4$};
\end{tikzpicture}
\end{center}

\paragraph{Пример 2.} Контур (цикл) длины 4:

\begin{center}
\begin{tikzpicture}[scale=1, every node/.style={circle,draw,inner sep=1.2pt}]
  \node (1) at (0,0) {$1$};
  \node (2) at (2,0) {$2$};
  \node (3) at (2,2) {$3$};
  \node (4) at (0,2) {$4$};
  \foreach \u/\v in {1/2,2/3,3/4,4/1}
    \draw (\u) -- (\v);
  \node at (1,-0.5) {Контур $1\to2\to3\to4\to1$};
\end{tikzpicture}
\end{center}

\subsubsection{7.5. Свойства путей и контуров}

\begin{itemize}[leftmargin=*]
  \item \emph{Комбинирование путей:} если существует путь из $u$ в $v$ и из $v$ в $w$, то их конкатенация даёт путь из $u$ в $w$.
  \item \emph{Связность:} граф $G$ называется связным, если для любых $u,v\in V$ существует путь из $u$ в $v$.
  \item \emph{Минимальный путь:} путь минимальной длины называют \textbf{коротким путём} или \emph{найдём его с помощью алгоритма Дейкстры}.
  \item \emph{Кycle Space:} множество всех циклов (контуров) образует векторное пространство над $\mathbb{F}_2$ (для ориентированных графов).
\end{itemize}

\subsubsection{7.6. Матрица смежности и подсчёт путей}

Если $A = (a_{ij})$ — матрица смежности графа $G$, то элемент матрицы $A^k$ в позиции $(i,j)$ равен числу различных путей длины $k$ из вершины $v_i$ в вершину $v_j$.

\[
  (A^k)_{ij} = \#\{\text{walks of length }k \text{ from }v_i\text{ to }v_j\}.
\]

Это позволяет:
\begin{itemize}[leftmargin=*]
  \item Вычислить количество путей фиксированной длины.
  \item Определить достижимость: существует путь любой длины $k\le n-1$.
\end{itemize}

\subsubsection{7.7. Заключение}

Пути и контуры — фундаментальные понятия теории графов, лежащие в основе алгоритмов поиска (BFS, DFS), анализа связности, планарности и многих применений в сетевых и прикладных задачах.

\subsubsection{Источники}

\begin{itemize}
  \item Д.Б.\,West, \emph{Introduction to Graph Theory}, Prentice Hall.
  \item Р.\,Diestel, \emph{Graph Theory}.
  \item \href{https://ru.wikipedia.org/wiki/Путь_в_графе}{Википедия: Путь в графе}
  \item \href{https://ru.wikipedia.org/wiki/Цикл_в_графе}{Википедия: Цикл (граф)}
\end{itemize}


\subsection{8. Симметрия графа и его дополнения}

\subsubsection{8.1. Автоморфизмы графа и группа симметрий}

Пусть $G=(V,E)$ — простой граф. \emph{Автоморфизмом} графа называется биекция
\[
  \varphi\colon V\;\to\;V
\]
такая, что для любых двух вершин $u,v\in V$ выполняется
\[
  \{u,v\}\in E \;\iff\;\{\varphi(u),\varphi(v)\}\in E.
\]
Другими словами, $\varphi$ сохраняет структуру смежности.

\begin{itemize}[leftmargin=*]
  \item Множество всех автоморфизмов графа $G$ образует группу при композиции отображений, называемую \textbf{группой автоморфизмов} $\operatorname{Aut}(G)$.
  \item Тривиальный автоморфизм — тождественное отображение $\mathrm{id}:v\mapsto v$.
  \item Если $\varphi\in\operatorname{Aut}(G)$ не является тождественным, говорят о \emph{неявной} (или неполной) симметрии.
\end{itemize}

\emph{Пояснение:} автоморфизмы — это «симметрии» графа, аналоги зеркальных и поворотных симметрий фигур. Они показывают, какие вершины и ребра можно «переставить», не меняя общей формы графа.

\subsubsection{8.2. Примеры симметрий}

\paragraph{Пример 1.} Цикл $C_4$ (четырёхвершинный цикл).  
Вершины можно пронумеровать $1,2,3,4$ по кругу. Автоморфизмы:
\[
  \text{поворот на }90^\circ:\;1\to2\to3\to4\to1,
\]
\[
  \text{отражение: }1\leftrightarrow4,\;2\leftrightarrow3,
\]
и их композиции. Группа симметрий изоморфна диhedral group $\mathrm{D}_4$ порядка 8.

\paragraph{Пример 2.} Полный граф $K_n$.  
Любая перестановка вершин сохраняет все рёбра, поэтому
\[
  \operatorname{Aut}(K_n)\cong S_n,
\]
симметричная группа порядка $n!$.

\subsubsection{8.3. Граф‑дополнение}

\emph{Дополнением} графа $G=(V,E)$ называется граф
\[
  \overline{G} = (V,\,\overline{E}),
  \quad \overline{E} = \bigl\{\{u,v\}\mid u\neq v,\;\{u,v\}\notin E\bigr\}.
\]
То есть в $\overline{G}$ все отсутствующие в исходном $G$ связи становятся рёбрами, а все прежние исчезают.

\begin{itemize}[leftmargin=*]
  \item $(\overline{G})\!\!\overline{\phantom{G}} = G$.
  \item Если $G$ простой, то и $\overline{G}$ простой.
  \item $\deg_{\overline G}(v) = |V|-1 - \deg_G(v)$.
\end{itemize}

\subsubsubsection*{Группа автоморфизмов и дополнение}

\vspace{-0.3em}
\[
  \operatorname{Aut}(\overline{G}) = \operatorname{Aut}(G).
\]
\emph{Пояснение:} перестановка вершин сохраняет и отсутствующие в $G$ связи, значит сохраняет рёбра дополнения.

\subsubsection{8.4. Иллюстрация: граф и его дополнение}

\begin{center}
\begin{tikzpicture}[scale=0.8, every node/.style={circle,draw,inner sep=1.2pt}]
  % Исходный граф G
  \node (1) at (0,0) {$1$};
  \node (2) at (2,0) {$2$};
  \node (3) at (1,1.7) {$3$};
  \draw (1) -- (2) -- (3);
  \draw (1) -- (3);
  \node at (1,-1) {$G$};
  % Стрелка
  \node at (3.5,0.8) {$\longrightarrow$};
  % Дополнение overline{G}
  \begin{scope}[xshift=5cm]
    \node (1) at (0,0) {$1$};
    \node (2) at (2,0) {$2$};
    \node (3) at (1,1.7) {$3$};
    % В дополнении G все 3 вершины не связаны, но добавляем отсутствующие
    % В G имели все три, значит complement has none: empty graph
    % Для примера возьмем другой: пусть в G все ребра, complement empty.
    % Но чтобы увидеть отличия, изменим: пусть G только ребро 1-2:
  \end{scope}
\end{tikzpicture}
\end{center}

\emph{Пример.} Пусть $G$ — треугольник $K_3$ (все три ребра). Тогда $\overline{G}$ — три изолированные вершины (нет рёбер).

\subsubsection{8.5. Свойства и применения}

\begin{itemize}[leftmargin=*]
  \item \textbf{Симметрия упрощает алгоритмы:} при поиске путей, раскраске и проверке изоморфизма можно работать с представителем орбиты.
  \item \textbf{Дополнение и свойства связности:} $G$ связен $\nRightarrow \overline G$ связен, но часто изучают одновременно пару $(G,\overline G)$, например в теореме Рамсея.
  \item \textbf{Оптимизация:} задачи клики в $G$ переходят в задачи независимого множества в $\overline G$.
\end{itemize}

\subsubsection{Источники}

\begin{itemize}
  \item Д.Б.\,West, \emph{Introduction to Graph Theory}, Prentice Hall.
  \item Р.\,Diestel, \emph{Graph Theory}.
  \item \href{https://ru.wikipedia.org/wiki/Автоморфизм_графа}{Википедия: Автоморфизм графа}
  \item \href{https://ru.wikipedia.org/wiki/Дополнение_графа}{Википедия: Дополнение графа}
\end{itemize}

\subsection{9. Двоичные алгебры}

\subsubsection{9.1. Понятие двоичной (бинарной) операции}

\begin{definition}
Пусть $A$ — непустое множество. \emph{Двоичной операцией} на $A$ называется отображение
\[
  * : A \times A \;\longrightarrow\; A,
  \quad (x,y)\mapsto x*y.
\]
\end{definition}

\emph{Интуиция:} берём два элемента из $A$, «складываем» их по правилу $*$ и получаем снова элемент из $A$.

\subsubsection{9.2. Свойства двоичной операции}

Пусть $*$ — двоичная операция на $A$. Говорят, что $*$ обладает свойствами:

\begin{itemize}[leftmargin=*]
  \item \textbf{Замкнутость}: по определению $x*y\in A$ для любых $x,y\in A$.
  \item \textbf{Ассоциативность}:
    \[
      (x*y)*z = x*(y*z),\quad \forall x,y,z\in A.
    \]
    Позволяет не ставить скобок при многократном применении.
  \item \textbf{Коммутативность}:
    \[
      x*y = y*x,\quad \forall x,y\in A.
    \]
  \item \textbf{Нейтральный (единичный) элемент}: существует $e\in A$ такое, что
    \[
      e*x = x*e = x,\quad \forall x\in A.
    \]
    Его часто обозначают $0$ или $1$ в зависимости от контекста.
  \item \textbf{Обратимые элементы}: элемент $x\in A$ называется обратимым, если существует $y\in A$ такой, что
    \[
      x*y = y*x = e.
    \]
    Тогда $y$ называют \emph{обратным} к $x$ и обозначают $x^{-1}$.
\end{itemize}

\subsubsection{9.3. Классификация двоичных алгебр}

\begin{enumerate}[label=\arabic*)]
  \item \textbf{Магма}: $(A,*)$ — любое множество с двоичной операцией (требуется лишь замкнутость).
  \item \textbf{Полугруппа}: магма с ассоциативной операцией.
  \item \textbf{Моноид}: полугруппа, в которой есть единица $e$.
  \item \textbf{Группа}: моноид, в котором каждый элемент обратим.
  \item \textbf{Абелева (коммутативная) группа}: группа с коммутативным $*$.
\end{enumerate}

\subsubsection{9.4. Примеры}

\begin{enumerate}[label=\arabic*)]
  \item $(\mathbb{Z}, +)$ — абелева группа, где единица $0$, обратный к $x$ есть $-x$.
  \item $(\mathbb{N}, +)$ — моноид (нет обратных элементов, кроме $0$).
  \item $(\{0,1\}, \wedge)$ — коммутативная монода, где $0\wedge1=0$, единица $1$.
  \item $(\{0,1\}, \oplus)$ (сумма по модулю 2) — абелева группа:  
    \[
      0\oplus0=0,\quad 0\oplus1=1,\quad1\oplus1=0;
      \quad e=0,\;x^{-1}=x.
    \]
  \item $(M_n(\mathbb{R}), \cdot)$ — полугруппа матриц; моноид при наличии единичной матрицы.
\end{enumerate}

\subsubsection{9.5. Таблица Кэли}

Для конечных алгебр удобно задавать операцию таблицей.  
\emph{Пример:} группа $(\{0,1\},\oplus)$:

\[
\begin{array}{c|cc}
\oplus & 0 & 1 \\ \hline
0 & 0 & 1 \\
1 & 1 & 0
\end{array}
\]

\subsubsection{9.6. Связь с булевыми алгебрами}

Булева алгебра — это \emph{расширенная} коммутативная группа с дополнительными операциями «и», «или» и «не» на множестве $\{0,1\}$.  
В частности, структура $(\{0,1\},\wedge,\vee,\neg)$ удовлетворяет ряду аксиом идемпотентности и дистрибутивности.

\subsubsection{9.7. Зачем нужны двоичные алгебры?}

\begin{itemize}[leftmargin=*]
  \item Моделирование и анализ абстрактных операций (сложение, умножение, логические связки).
  \item Основа теории групп и её приложений: симметрии, криптография, теории кодирования.
  \item В информатике: операции над битами, булевы функции, конечные автоматы.
\end{itemize}

\subsubsection{Источники}

\begin{itemize}
  \item С.\,Ланг, \emph{Алгебра}.
  \item Д.\,С. Джонсонбауг, \emph{Дискретная математика}, Pearson.
  \item \href{https://ru.wikipedia.org/wiki/Бинарная_операция}{Википедия: Бинарная операция}.
\end{itemize}
% End of sections

\end{document}

