% ===== Документ класс =====
\documentclass[12pt,a4paper]{article}

% ===== Шрифты и unicode (для LuaLaTeX) =====
\usepackage{fontspec}                % управление системными/otf/ttf шрифтами
\setmainfont{Times New Roman}        % основной текстовый шрифт (если установлен)
% альтернатива, если нет Times New Roman: \setmainfont{TeX Gyre Termes}

% ===== Математика в Unicode =====
\usepackage{unicode-math}            % подключаем современную math-систему
\setmathfont{XITS Math}              % математический шрифт с поддержкой Unicode (XITS Math)
% альтернатива: \setmathfont{Latin Modern Math} или \setmathfont{TeX Gyre Termes Math}

% ===== Язык =====
\usepackage{polyglossia}             % для LuaLaTeX удобно polyglossia
\setdefaultlanguage{russian}
\setotherlanguages{english}

% ===== Базовые математические и вспомогательные пакеты =====
\usepackage{amsmath,amssymb,amsthm}  % стандартные AMS-пакеты
\usepackage{bm}                      % жирная математика (\bm)
\usepackage{mathtools}               % расширения amsmath

% ===== Графика и рисунки =====
\usepackage{graphicx}
\usepackage{caption}
\usepackage{subcaption}

% ===== TikZ и pgfplots для графиков =====
\usepackage{tikz}
\usepackage{pgfplots}
\pgfplotsset{compat=1.18}            % поставь версию, доступную в твоём TeX-дистрибутиве

% ===== Ссылки и гиперссылки =====
\usepackage{hyperref}
\hypersetup{
  unicode=true,
  colorlinks=true,
  linkcolor=blue,
  citecolor=blue,
  urlcolor=blue
}

% ===== Мелочи типографики =====
\usepackage{microtype}               % опционально, улучшает переносы и кернинг

% ===== (Опционально) fallback: если unicode-math недоступен или XITS не установлена =====
% \usepackage{newtxtext,newtxmath}   % Times-подобное оформление с math под pdfLaTeX/LuaLaTeX
% комментарий: newtxtext/newtxmath хорошая замена, если нет системных шрифтов

% ===== Конец преамбулы =====


\begin{document}

\tableofcontents

% Input all section files
\section{Векторы}

\textbf{Вектор} — это математический объект, описывающий как направление, так и величину. Векторы часто изображаются как направленные отрезки (стрелки), начинающиеся в начале координат.

В алгебраической записи вектор в $n$-мерном пространстве $\mathbb{R}^n$ — это упорядоченный набор $n$ чисел:
\[
\vec{v} = (v_1, v_2, \dots, v_n)
\]

\subsection*{Примеры векторов}
\begin{itemize}
  \item В $\mathbb{R}^2$: $\vec{a} = (3, -1)$
  \item В $\mathbb{R}^3$: $\vec{b} = (0, 2, 1)$
  \item В $\mathbb{R}^4$: $\vec{c} = (1, 0, 0, -1)$
\end{itemize}

\vspace{1em}
\textbf{Геометрическая интерпретация:} вектор — это перемещение из одной точки в другую. Например, вектор $(2, 1)$ соответствует сдвигу на 2 единицы вправо и 1 вверх.

\vspace{1em}
\begin{center}
\begin{tikzpicture}[scale=1.2,>=stealth]
\draw[->,gray!30] (-1,0) -- (4,0) node[right] {$x$};
\draw[->,gray!30] (0,-1) -- (0,3) node[above] {$y$};
\draw[->,thick,blue] (0,0) -- (2,1) node[above right] {$\vec{v}$};
\end{tikzpicture}
\end{center}

\subsection*{Операции с векторами}

\begin{enumerate}
  \item \textbf{Сложение:}
  \[
  (x_1, y_1) + (x_2, y_2) = (x_1 + x_2, y_1 + y_2)
  \]
  \item \textbf{Умножение на число (скаляр):}
  \[
  \lambda \cdot (x, y) = (\lambda x, \lambda y)
  \]
  \item \textbf{Нулевой вектор:}
  \[
  \vec{0} = (0, 0, \dots, 0)
  \]
\end{enumerate}

\textbf{Графически сложение векторов} выглядит как «перенос конца первого вектора к началу второго» — получаем диагональ параллелограмма:

\vspace{1em}
\begin{center}
\begin{tikzpicture}[scale=0.9,>=stealth]
\draw[->,gray!40] (-1,0) -- (5,0) node[right] {$x$};
\draw[->,gray!40] (0,-1) -- (0,4) node[above] {$y$};

\draw[->,blue,thick] (0,0) -- (2,1.5) node[above right] {$\vec{a}$};
\draw[->,red,thick] (0,0) -- (1,2) node[above left] {$\vec{b}$};
\draw[->,green!70!black,thick] (0,0) -- (3,3.5) node[right] {$\vec{a} + \vec{b}$};

\draw[dashed] (2,1.5) -- (3,3.5);
\draw[dashed] (1,2) -- (3,3.5);
\end{tikzpicture}
\end{center}

---

\section{Линейная зависимость системы векторов}

Пусть заданы векторы $\vec{v}_1, \vec{v}_2, \dots, \vec{v}_k$ в пространстве $V$. Мы говорим, что они \textbf{линейно зависимы}, если существует набор чисел $\lambda_1, \dots, \lambda_k$, не все равные нулю, такой что:
\[
\lambda_1 \vec{v}_1 + \lambda_2 \vec{v}_2 + \dots + \lambda_k \vec{v}_k = \vec{0}
\]

Если же такое равенство возможно только при $\lambda_1 = \dots = \lambda_k = 0$, то векторы \textbf{линейно независимы}.

\subsection*{Интуитивно:}
Если один вектор можно выразить через другие — система зависима.

\textbf{Пример 1.}
\[
\vec{v}_1 = (1, 2), \quad \vec{v}_2 = (2, 4)
\]
Очевидно, что $\vec{v}_2 = 2 \vec{v}_1$, значит, они линейно зависимы.

\vspace{1em}
\begin{center}
\begin{tikzpicture}[scale=1.2,>=stealth]
\draw[->,gray!40] (-0.5,0) -- (3,0);
\draw[->,gray!40] (0,-0.5) -- (0,3);

\draw[->,blue,thick] (0,0) -- (1,2) node[above left] {$\vec{v}_1$};
\draw[->,red,thick] (0,0) -- (2,4) node[above right] {$\vec{v}_2$};
\end{tikzpicture}
\end{center}

\textbf{Пример 2.}
Векторы $\vec{u}_1 = (1, 0)$ и $\vec{u}_2 = (0, 1)$ линейно независимы, так как невозможно выразить один через другой. Они формируют базис в $\mathbb{R}^2$.

---

\section{Базис линейного пространства}

Базис — это система векторов, которая:
\begin{enumerate}
  \item линейно независима;
  \item порождает всё пространство (любой вектор можно выразить через неё).
\end{enumerate}

Если базис состоит из $n$ векторов, говорят, что размерность пространства равна $n$.

\textbf{Пример.} В $\mathbb{R}^2$ стандартный базис:
\[
\vec{e}_1 = (1, 0), \quad \vec{e}_2 = (0, 1)
\]
Тогда любой вектор $\vec{v} = (x, y)$ записывается как:
\[
\vec{v} = x \vec{e}_1 + y \vec{e}_2
\]

\vspace{1em}
\begin{center}
\begin{tikzpicture}[scale=1.1,>=stealth]
\draw[->,gray!40] (-0.5,0) -- (3.5,0) node[right] {$x$};
\draw[->,gray!40] (0,-0.5) -- (0,3.5) node[above] {$y$};

\draw[->,thick] (0,0) -- (2,0) node[below] {$x\vec{e}_1$};
\draw[->,thick] (2,0) -- (2,1.5) node[right] {$y\vec{e}_2$};
\draw[->,blue,thick] (0,0) -- (2,1.5) node[above right] {$\vec{v}$};
\end{tikzpicture}
\end{center}

\subsection*{Важно:}
Базис может быть не единственным. Например, вектора $\vec{e}_1 = (1, 1)$ и $\vec{e}_2 = (1, -1)$ тоже образуют базис в $\mathbb{R}^2$.

---

\section{Скалярное произведение векторов}

Скалярное произведение двух векторов $\vec{a} = (a_1, \dots, a_n)$ и $\vec{b} = (b_1, \dots, b_n)$:
\[
\langle \vec{a}, \vec{b} \rangle = a_1b_1 + a_2b_2 + \dots + a_n b_n
\]

\textbf{Пример:}
\[
\vec{a} = (1, 2), \quad \vec{b} = (3, 4) \Rightarrow \langle \vec{a}, \vec{b} \rangle = 1\cdot3 + 2\cdot4 = 11
\]

\textbf{Свойства:}
\begin{itemize}
  \item Коммутативность: $\langle \vec{a}, \vec{b} \rangle = \langle \vec{b}, \vec{a} \rangle$
  \item Линейность по каждому аргументу
  \item $\langle \vec{a}, \vec{a} \rangle = \|\vec{a}\|^2$
\end{itemize}

\textbf{Геометрическая формула:}
\[
\langle \vec{a}, \vec{b} \rangle = \|\vec{a}\|\|\vec{b}\|\cos\theta
\]

\textbf{Если } $\langle \vec{a}, \vec{b} \rangle = 0$ $\Rightarrow$ векторы перпендикулярны (ортогональны).

\vspace{1em}
\begin{center}
\begin{tikzpicture}[scale=1.2,>=stealth]
\coordinate (O) at (0,0);
\coordinate (A) at (2,0);
\coordinate (B) at (1,2);
\draw[->,gray!40] (-0.5,0) -- (3,0);
\draw[->,gray!40] (0,-0.5) -- (0,3);
\draw[->,thick] (O) -- (A) node[below] {$\vec{a}$};
\draw[->,thick] (O) -- (B) node[above] {$\vec{b}$};
\draw (1,0) arc[start angle=0,end angle=63,radius=1cm];
\node at (1.1,0.2) {$\theta$};
\end{tikzpicture}
\end{center}

\subsection*{Приложение: длина и угол}
Длина вектора $\vec{a}$ (её называют \textbf{нормой}) выражается так:
\[
\|\vec{a}\| = \sqrt{\langle \vec{a}, \vec{a} \rangle}
\]

А угол между двумя векторами вычисляется по формуле:
\[
\cos\theta = \frac{\langle \vec{a}, \vec{b} \rangle}{\|\vec{a}\|\|\vec{b}\|}
\]

---

\subsection*{Выводы}
\begin{itemize}
  \item Векторы — базовые элементы линейной алгебры, описывающие направление и величину.
  \item Линейная зависимость позволяет понять, насколько векторы "разные" и важны.
  \item Базис даёт возможность представить любое состояние системы как комбинацию базовых движений.
  \item Скалярное произведение связывает векторы с геометрией: длиной и углом.
\end{itemize}

\section{Матрицы. Их свойства. Транспонированная матрица. Ранг матрицы}

\textbf{Матрица} — это прямоугольная таблица чисел, организованная в строки и столбцы. Она записывается в виде:
\[
A =
\begin{pmatrix}
a_{11} & a_{12} & \dots & a_{1n} \\
a_{21} & a_{22} & \dots & a_{2n} \\
\vdots & \vdots & \ddots & \vdots \\
a_{m1} & a_{m2} & \dots & a_{mn}
\end{pmatrix}
\]
где $a_{ij}$ — элемент матрицы на $i$-й строке и $j$-м столбце.

\subsection*{Обозначения и размерность}

Матрицу обозначают заглавной латинской буквой ($A$, $B$, $C$ и т.д.). Размерность матрицы — это количество строк и столбцов. Если в матрице $m$ строк и $n$ столбцов, её размер обозначают как $m \times n$.

\textbf{Примеры:}
\[
A =
\begin{pmatrix}
1 & 2 \\
3 & 4
\end{pmatrix}, \quad
B =
\begin{pmatrix}
1 & 0 & -1 \\
2 & 3 & 5
\end{pmatrix}
\]
Здесь $A$ — квадратная матрица $2\times2$, $B$ — прямоугольная матрица $2\times3$.

---

\subsection*{Основные типы матриц}

\begin{itemize}
  \item \textbf{Нулевая матрица:} все элементы равны нулю.
  \item \textbf{Единичная матрица $I_n$:} квадратная матрица с единицами на главной диагонали и нулями вне её.
  \[
  I_3 =
  \begin{pmatrix}
  1 & 0 & 0 \\
  0 & 1 & 0 \\
  0 & 0 & 1
  \end{pmatrix}
  \]
  \item \textbf{Диагональная матрица:} все элементы вне главной диагонали равны нулю.
  \item \textbf{Квадратная матрица:} одинаковое число строк и столбцов.
  \item \textbf{Столбец (вектор-столбец):} матрица размером $m \times 1$.
  \item \textbf{Строка (вектор-строка):} матрица размером $1 \times n$.
\end{itemize}

---

\subsection*{Операции с матрицами}

\begin{enumerate}
  \item \textbf{Сложение:} складываются поэлементно. Возможно только для матриц одинакового размера.
  \[
  A + B = \left(a_{ij} + b_{ij}\right)
  \]

  \item \textbf{Умножение на скаляр:}
  \[
  \lambda A = \left(\lambda \cdot a_{ij}\right)
  \]

  \item \textbf{Умножение матриц:} если $A$ — матрица размера $m \times n$, а $B$ — $n \times k$, то их произведение $C = AB$ будет размером $m \times k$:
  \[
  c_{ij} = \sum_{r=1}^{n} a_{ir} \cdot b_{rj}
  \]

  \item \textbf{Транспонирование (см. ниже)} — замена строк и столбцов.
\end{enumerate}

---

\subsection*{Свойства операций}

\begin{itemize}
  \item Коммутативность сложения: $A + B = B + A$
  \item Ассоциативность: $(A + B) + C = A + (B + C)$
  \item Дистрибутивность: $\lambda(A + B) = \lambda A + \lambda B$
  \item $(AB)^T = B^T A^T$ — важное свойство транспонирования
\end{itemize}

---

\subsection*{Транспонированная матрица}

Матрица $A^T$ (читается: «A транспонированная») получается из $A$ заменой строк на столбцы. Формально:
\[
(A^T)_{ij} = A_{ji}
\]

\textbf{Пример:}
\[
A =
\begin{pmatrix}
1 & 2 \\
3 & 4 \\
5 & 6
\end{pmatrix} \quad \Rightarrow \quad
A^T =
\begin{pmatrix}
1 & 3 & 5 \\
2 & 4 & 6
\end{pmatrix}
\]

\textbf{Свойства транспонирования:}
\begin{itemize}
  \item $(A^T)^T = A$
  \item $(A + B)^T = A^T + B^T$
  \item $(\lambda A)^T = \lambda A^T$
  \item $(AB)^T = B^T A^T$
\end{itemize}

---

\subsection*{Ранг матрицы}

\textbf{Ранг матрицы} — это максимальное число линейно независимых строк (или столбцов) в матрице.

Обозначается: $\operatorname{rank}(A)$.

\textbf{Интуитивно:} ранг показывает, сколько "уникальной" информации содержится в строках или столбцах.

\textbf{Пример:}
\[
A =
\begin{pmatrix}
1 & 2 \\
2 & 4
\end{pmatrix}
\Rightarrow \text{строки линейно зависимы} \Rightarrow \operatorname{rank}(A) = 1
\]

\textbf{Другой пример:}
\[
B =
\begin{pmatrix}
1 & 2 & 3 \\
0 & 1 & 4 \\
0 & 0 & 1
\end{pmatrix}
\Rightarrow \operatorname{rank}(B) = 3
\]

\subsection*{Как находить ранг?}

Обычно с помощью преобразования матрицы к \textbf{ступенчатому виду} методом Гаусса. Количество ненулевых строк после преобразования и будет рангом.

\subsubsection*{Пример пошагово:}

Дана матрица:
\[
A =
\begin{pmatrix}
1 & 2 & 1 \\
2 & 4 & 2 \\
3 & 6 & 3
\end{pmatrix}
\]

Видим: вторая и третья строки — кратные первой. После приведения:
\[
\begin{pmatrix}
1 & 2 & 1 \\
0 & 0 & 0 \\
0 & 0 & 0
\end{pmatrix}
\Rightarrow \operatorname{rank}(A) = 1
\]

---

\subsection*{Геометрическая интерпретация ранга}

Векторы-строки (или столбцы) матрицы можно представить как векторы в пространстве. Ранг говорит о том, какое пространство они натягивают:
\begin{itemize}
  \item Ранг 1: все лежат на одной прямой
  \item Ранг 2: в одной плоскости
  \item Ранг 3: в трёхмерном пространстве и т.д.
\end{itemize}

---

\subsection*{Важность ранга}

Ранг используется в:
\begin{itemize}
  \item Исследовании решений линейных систем: число решений зависит от ранга матрицы коэффициентов.
  \item Анализе линейной зависимости строк/столбцов.
  \item Проверке обратимости матрицы: квадратная матрица обратима $\iff$ её ранг равен размерности.
\end{itemize}

---

\subsection*{Выводы по теме}

\begin{itemize}
  \item Матрицы — основа линейной алгебры. Они обобщают векторы, храня данные и операции.
  \item Транспонирование меняет строки и столбцы местами.
  \item Ранг показывает, сколько независимых строк/столбцов содержит матрица.
  \item Если ранг меньше полной размерности — значит, матрица "выражает" только подпространство.
\end{itemize}

\section{Сложение, умножение матрицы на число, умножение матриц, транспонирование матриц. Обратная матрица}

\subsection*{Сложение матриц}

Две матрицы $A$ и $B$ одинакового размера ($m \times n$) можно сложить, если у них совпадают размеры. Сложение происходит поэлементно:
\[
(A + B)_{ij} = A_{ij} + B_{ij}
\]

\textbf{Пример:}
\[
A =
\begin{pmatrix}
1 & 2 \\
3 & 4
\end{pmatrix}, \quad
B =
\begin{pmatrix}
5 & 6 \\
7 & 8
\end{pmatrix}
\Rightarrow
A + B =
\begin{pmatrix}
6 & 8 \\
10 & 12
\end{pmatrix}
\]

\textbf{Свойства сложения:}
\begin{itemize}
  \item Коммутативность: $A + B = B + A$
  \item Ассоциативность: $(A + B) + C = A + (B + C)$
  \item Существование нулевой матрицы $O$ (нулевая поэлементно): $A + O = A$
\end{itemize}

---

\subsection*{Умножение матрицы на число}

Если $\lambda \in \mathbb{R}$ — число (скаляр), то умножение $\lambda \cdot A$ означает умножение каждого элемента матрицы на это число:
\[
(\lambda A)_{ij} = \lambda \cdot A_{ij}
\]

\textbf{Пример:}
\[
A =
\begin{pmatrix}
2 & -1 \\
0 & 3
\end{pmatrix}, \quad \lambda = 4
\Rightarrow
4A =
\begin{pmatrix}
8 & -4 \\
0 & 12
\end{pmatrix}
\]

\textbf{Свойства:}
\begin{itemize}
  \item $\lambda (\mu A) = (\lambda \mu) A$
  \item $(\lambda + \mu) A = \lambda A + \mu A$
  \item $\lambda (A + B) = \lambda A + \lambda B$
\end{itemize}

---

\subsection*{Умножение матриц}

Матрицы $A$ и $B$ можно перемножить, если \textbf{число столбцов в $A$} равно \textbf{числу строк в $B$}.

Если $A$ — размера $m \times n$, а $B$ — $n \times k$, то произведение $C = AB$ — это матрица $m \times k$, где:
\[
C_{ij} = \sum_{r=1}^{n} A_{ir} \cdot B_{rj}
\]

То есть: элемент $C_{ij}$ получается как скалярное произведение $i$-й строки $A$ и $j$-го столбца $B$.

\textbf{Пример:}
\[
A =
\begin{pmatrix}
1 & 2 \\
3 & 4
\end{pmatrix}, \quad
B =
\begin{pmatrix}
0 & 1 \\
2 & 3
\end{pmatrix}
\]

\[
AB =
\begin{pmatrix}
1\cdot0 + 2\cdot2 & 1\cdot1 + 2\cdot3 \\
3\cdot0 + 4\cdot2 & 3\cdot1 + 4\cdot3
\end{pmatrix}
=
\begin{pmatrix}
4 & 7 \\
8 & 15
\end{pmatrix}
\]

\textbf{Важно:} $AB \ne BA$ в общем случае! Умножение матриц \textbf{не коммутативно}.

\textbf{Свойства:}
\begin{itemize}
  \item Ассоциативность: $A(BC) = (AB)C$
  \item Дистрибутивность: $A(B + C) = AB + AC$
  \item $(AB)^T = B^T A^T$ — транспонирование произведения
\end{itemize}

---

\subsection*{Транспонирование матрицы}

Транспонирование — это операция, при которой строки становятся столбцами, а столбцы — строками.

Для любой матрицы $A$, её транспонированная матрица $A^T$ определяется как:
\[
(A^T)_{ij} = A_{ji}
\]

\textbf{Пример:}
\[
A =
\begin{pmatrix}
1 & 2 & 3 \\
4 & 5 & 6
\end{pmatrix}
\Rightarrow
A^T =
\begin{pmatrix}
1 & 4 \\
2 & 5 \\
3 & 6
\end{pmatrix}
\]

\textbf{Свойства транспонирования:}
\begin{itemize}
  \item $(A^T)^T = A$
  \item $(A + B)^T = A^T + B^T$
  \item $(\lambda A)^T = \lambda A^T$
  \item $(AB)^T = B^T A^T$
\end{itemize}

Эта операция часто используется при симметризации, а также в определениях симметрических и ортогональных матриц.

---

\subsection*{Обратная матрица}

\textbf{Обратная матрица} $A^{-1}$ к квадратной матрице $A$ определяется как:
\[
A \cdot A^{-1} = A^{-1} \cdot A = I
\]
где $I$ — единичная матрица той же размерности.

\textbf{Условия существования:}
\begin{itemize}
  \item Матрица должна быть \textbf{квадратной}.
  \item Её \textbf{определитель не должен равняться нулю} ($\det A \ne 0$).
  \item Ранг $A$ должен равняться её размерности: $\operatorname{rank}(A) = n$
\end{itemize}

\textbf{Пример:}
\[
A =
\begin{pmatrix}
1 & 2 \\
3 & 4
\end{pmatrix}
\Rightarrow
A^{-1} = \frac{1}{\det A}
\begin{pmatrix}
4 & -2 \\
-3 & 1
\end{pmatrix}
=
\frac{1}{(1\cdot4 - 2\cdot3)}
\begin{pmatrix}
4 & -2 \\
-3 & 1
\end{pmatrix}
=
\begin{pmatrix}
-2 & 1 \\
1.5 & -0.5
\end{pmatrix}
\]

\subsection*{Способы нахождения обратной матрицы}

\begin{enumerate}
  \item Для $2\times2$-матриц:
  \[
  A = \begin{pmatrix}a & b\\ c & d\end{pmatrix}
  \Rightarrow
  A^{-1} = \frac{1}{ad - bc} \begin{pmatrix}d & -b \\ -c & a\end{pmatrix}
  \]

  \item Для больших матриц:
  \begin{itemize}
    \item Через присоединённую матрицу (алгебраические дополнения + транспонирование + деление на определитель)
    \item Метод Гаусса: расширение $A$ до $[A | I]$ и приведение к $[I | A^{-1}]$
  \end{itemize}
\end{enumerate}

---

\subsection*{Свойства обратной матрицы}

\begin{itemize}
  \item $(A^{-1})^{-1} = A$
  \item $(AB)^{-1} = B^{-1} A^{-1}$
  \item $(A^T)^{-1} = (A^{-1})^T$
\end{itemize}

\textbf{Важно:} не все матрицы имеют обратную. Такие матрицы называются \textbf{вырожденными}.

---

\subsection*{Применения обратной матрицы}

\begin{itemize}
  \item Решение систем уравнений: $A\vec{x} = \vec{b} \Rightarrow \vec{x} = A^{-1}\vec{b}$
  \item Вывод формул в статистике и машинном обучении
  \item Нормализация линейных преобразований
  \item Преобразование координат
\end{itemize}

---

\subsection*{Выводы}

\begin{itemize}
  \item Операции над матрицами (сложение, умножение, транспонирование) формируют алгебраическую структуру.
  \item Умножение матриц — основа линейных отображений и систем уравнений.
  \item Транспонирование — полезная симметризующая операция.
  \item Обратная матрица существует только у невырожденных квадратных матриц и даёт способ обращения линейных операторов.
\end{itemize}

\section{Аппроксимация и интерполяция функций}

\textbf{Аппроксимация} и \textbf{интерполяция} — это два метода приближённого описания функций, основанные на наборе дискретных точек.

\textbf{Интерполяция} — это построение функции, которая точно проходит через заданные точки.  
\textbf{Аппроксимация} — это построение функции, которая приближённо описывает данные, но может не проходить через все точки.

---

\subsection*{Постановка задачи}

Пусть дана таблица значений:
\[
(x_0, y_0), \ (x_1, y_1), \ \dots, \ (x_n, y_n)
\]
Наша цель — построить функцию $f(x)$ такую, что:
\begin{itemize}
  \item \textbf{Для интерполяции:} $f(x_i) = y_i$ для всех $i$
  \item \textbf{Для аппроксимации:} $f(x_i) \approx y_i$
\end{itemize}

---

\subsection*{Интерполяция: идея и цель}

Интерполяция позволяет восстанавливать значение функции в промежуточных точках, не выходя за пределы интервала $[x_0, x_n]$.

\textbf{Пример:} если известно, что
\[
f(1) = 2, \quad f(2) = 4, \quad f(3) = 6
\]
можем интерполировать $f(x)$, скажем, через многочлен второй степени и вычислить $f(1.5)$.

---

\subsection*{Линейная интерполяция}

Между двумя точками $(x_0, y_0)$ и $(x_1, y_1)$ интерполяционная функция задаётся по формуле:
\[
f(x) = y_0 + \frac{y_1 - y_0}{x_1 - x_0}(x - x_0)
\]

Это уравнение прямой, проходящей через две точки. Очень просто, но недостаточно точно для сложных функций.

---

\subsection*{Полиномиальная интерполяция}

Если заданы $n+1$ точек, можно построить единственный многочлен степени не выше $n$, который проходит через все точки.

\textbf{Формула Лагранжа:}
\[
P_n(x) = \sum_{i=0}^{n} y_i \cdot L_i(x), \quad \text{где } L_i(x) = \prod_{\substack{j=0 \\ j \ne i}}^{n} \frac{x - x_j}{x_i - x_j}
\]

Каждое $L_i(x)$ — базисный многочлен Лагранжа, равный 1 в точке $x_i$ и 0 в остальных $x_j$.

\textbf{Проблема:} при увеличении числа узлов интерполяция может сильно колебаться (эффект Рунге), особенно на концах интервала.

---

\subsection*{Сплайны (кубическая интерполяция)}

\textbf{Сплайн-интерполяция} делит интервал на участки и на каждом строит многочлен степени 3 (кубический сплайн), с условием сглаженности в стыках.

\begin{itemize}
  \item Гладкость первого и второго порядка: $C^2$-непрерывность
  \item Сплайны хорошо подходят для графиков, траекторий и данных с шумом
\end{itemize}

\textbf{Визуально:}
\begin{center}
\begin{tikzpicture}[scale=1.0]
\draw[->] (0,0) -- (7,0) node[right] {$x$};
\draw[->] (0,-1) -- (0,3) node[above] {$y$};

\foreach \x/\y in {1/1, 2/1.5, 3/2.3, 4/1.9, 5/1.5, 6/2}
  \filldraw[blue] (\x,\y) circle (2pt);

\draw[thick,smooth,tension=0.7] plot coordinates {(1,1) (2,1.5) (3,2.3) (4,1.9) (5,1.5) (6,2)};
\end{tikzpicture}
\end{center}

---

\subsection*{Аппроксимация: общая идея}

Аппроксимация применяется, когда функция неизвестна, но имеются измеренные значения с шумом. Здесь уже не требуется точное прохождение через точки.

\textbf{Идея:} найти «наилучшую» функцию $f(x)$, которая \textit{приблизительно} соответствует данным.

Часто ищут функцию в виде:
\[
f(x) = a_0 + a_1 x + a_2 x^2 + \dots + a_n x^n
\]

\subsection*{Аппроксимация методом наименьших квадратов (МНК)}

Пусть есть точки $(x_i, y_i)$, и нужно найти параметры $a_0, a_1, \dots, a_n$, минимизирующие отклонение:
\[
S(a_0, a_1, \dots, a_n) = \sum_{i=0}^n \left(f(x_i) - y_i\right)^2
\]

Минимум достигается при решении системы нормальных уравнений, которая получается из частных производных $S$ по параметрам $a_k$.

\textbf{Частный случай — линейная аппроксимация:}
\[
f(x) = a_0 + a_1 x
\]

Тогда минимизируется:
\[
S(a_0, a_1) = \sum_{i=1}^{n} \left(a_0 + a_1 x_i - y_i\right)^2
\]

Решение:
\[
\begin{cases}
n a_0 + a_1 \sum x_i = \sum y_i \\
a_0 \sum x_i + a_1 \sum x_i^2 = \sum x_i y_i
\end{cases}
\]

---

\subsection*{Сравнение: интерполяция vs аппроксимация}

\begin{tabular}{|c|c|c|}
\hline
Критерий & Интерполяция & Аппроксимация \\
\hline
Проходит через точки & Да & Не обязательно \\
\hline
Чувствительность к шуму & Высокая & Устойчивая \\
\hline
Сложность вычислений & Средняя–высокая & Низкая–средняя \\
\hline
Гладкость & Может не быть & Обычно есть \\
\hline
\end{tabular}

---

\subsection*{Практические применения}

\begin{itemize}
  \item \textbf{Интерполяция}:
    \begin{itemize}
      \item Таблицы и справочники
      \item Заполнение пропущенных значений
      \item Графическая визуализация
    \end{itemize}
  \item \textbf{Аппроксимация}:
    \begin{itemize}
      \item Обработка измерений с шумом
      \item Моделирование реальных процессов
      \item Предсказания, тренды
    \end{itemize}
\end{itemize}

---

\subsection*{Выводы по теме}

\begin{itemize}
  \item Интерполяция позволяет точно восстановить функцию внутри диапазона, но может колебаться.
  \item Аппроксимация — более устойчивая техника, особенно с шумными или неточными данными.
  \item Полиномы Лагранжа и кубические сплайны — мощные методы интерполяции.
  \item Метод наименьших квадратов — классический способ аппроксимации, широко используемый в статистике и машинном обучении.
\end{itemize}

\section{Производные. Необходимое и достаточное условия дифференцируемости функции. Частные и полные производные}

\subsection*{Понятие производной функции одной переменной}

Пусть $f(x)$ определена в окрестности точки $x_0$. Производная функции $f$ в точке $x_0$ — это предел отношения приращения функции к приращению аргумента:
\[
f'(x_0) = \lim_{\Delta x \to 0} \frac{f(x_0 + \Delta x) - f(x_0)}{\Delta x}
\]

Если предел существует, то говорят, что функция \textbf{дифференцируема} в точке $x_0$.

\textbf{Геометрический смысл:} производная — это угловой коэффициент касательной к графику функции в данной точке.

\subsection*{Дифференцируемость и непрерывность}

\begin{itemize}
  \item Если функция дифференцируема в точке $x_0$, то она непрерывна в этой точке.
  \item Обратное неверно: непрерывность не гарантирует существование производной.
\end{itemize}

\textbf{Пример (не дифференцируема):}
\[
f(x) = |x| \Rightarrow f'(0) \text{ не существует, хотя } f \text{ непрерывна в } 0
\]

---

\subsection*{Производная по направлению и частные производные}

Пусть $f(x, y)$ — функция двух переменных.

Производную по направлению можно определить через вектор направления $\vec{l} = (l_1, l_2)$:
\[
D_{\vec{l}} f(x_0, y_0) = \lim_{h \to 0} \frac{f(x_0 + h l_1, y_0 + h l_2) - f(x_0, y_0)}{h}
\]

Частные производные — это производные по отдельным переменным:
\[
\frac{\partial f}{\partial x}(x_0, y_0) = \lim_{h \to 0} \frac{f(x_0 + h, y_0) - f(x_0, y_0)}{h}
\]
\[
\frac{\partial f}{\partial y}(x_0, y_0) = \lim_{h \to 0} \frac{f(x_0, y_0 + h) - f(x_0, y_0)}{h}
\]

\textbf{Обозначения:}
\[
f_x(x, y), \quad f_y(x, y), \quad \text{или } \partial_x f, \ \partial_y f
\]

\textbf{Пример:} пусть $f(x, y) = x^2 y + \sin(xy)$

\[
\frac{\partial f}{\partial x} = 2x y + y \cos(xy), \quad
\frac{\partial f}{\partial y} = x^2 + x \cos(xy)
\]

---

\subsection*{Необходимое и достаточное условия дифференцируемости функции многих переменных}

Функция $f(x, y)$ называется \textbf{дифференцируемой в точке} $(x_0, y_0)$, если приращение можно представить в виде:
\[
\Delta f = f(x_0 + \Delta x, y_0 + \Delta y) - f(x_0, y_0) = A \Delta x + B \Delta y + o(\sqrt{\Delta x^2 + \Delta y^2})
\]

где $A$ и $B$ — постоянные (зависят от точки), а $o(\rho)$ — бесконечно малая по сравнению с $\rho$.

\textbf{Формально:} $f$ дифференцируема в $(x_0, y_0)$, если существует линейное отображение $L$, приближающее $\Delta f$.

\subsubsection*{Необходимое условие дифференцируемости}

Если $f$ дифференцируема в $(x_0, y_0)$, то:
\begin{itemize}
  \item Частные производные $\frac{\partial f}{\partial x}, \frac{\partial f}{\partial y}$ существуют
  \item $\Delta f = \frac{\partial f}{\partial x} \Delta x + \frac{\partial f}{\partial y} \Delta y + o(\sqrt{\Delta x^2 + \Delta y^2})$
\end{itemize}

\subsubsection*{Достаточное условие дифференцируемости}

Если:
\begin{itemize}
  \item Частные производные существуют в окрестности точки
  \item И они непрерывны в точке $(x_0, y_0)$
\end{itemize}
то $f$ дифференцируема в этой точке.

---

\subsection*{Градиент и направление наибольшего роста}

Градиент — это вектор, составленный из всех частных производных:
\[
\nabla f(x, y) = \left( \frac{\partial f}{\partial x}, \frac{\partial f}{\partial y} \right)
\]

Производная функции по направлению вектора $\vec{l}$ выражается как скалярное произведение:
\[
D_{\vec{l}} f = \nabla f \cdot \vec{l}
\]

\textbf{Свойства:}
\begin{itemize}
  \item Направление градиента — направление наибольшего возрастания функции.
  \item Если $\nabla f = \vec{0}$, то это критическая точка (возможно максимум, минимум или седло).
\end{itemize}

---

\subsection*{Полный дифференциал}

Если функция $f(x, y)$ дифференцируема, то её полное приращение можно выразить через полный дифференциал:
\[
df = \frac{\partial f}{\partial x} dx + \frac{\partial f}{\partial y} dy
\]

\textbf{Пример:} $f(x, y) = x^2 y + \sin(xy)$

\[
df = (2x y + y \cos(xy)) dx + (x^2 + x \cos(xy)) dy
\]

\textbf{Полный дифференциал} используется:
\begin{itemize}
  \item В оценке приращений функции
  \item В дифференциальной геометрии и анализе ошибок
  \item При переходе к новым координатам
\end{itemize}

---

\subsection*{Итоги}

\begin{itemize}
  \item Производная — это мера изменения функции.
  \item Частные производные — изменения по осям координат.
  \item Дифференцируемость функции двух переменных требует не только существования производных, но и их «согласованного» поведения.
  \item Градиент показывает направление роста функции.
  \item Полный дифференциал — обобщение производной на многомерные функции.
\end{itemize}

\section{Частные производные. Градиент функции. Производная по направлению}

\subsection*{Частные производные функции двух переменных}

Пусть $f(x, y)$ — функция двух переменных, определённая в окрестности точки $(x_0, y_0)$. Тогда:

\[
\frac{\partial f}{\partial x}(x_0, y_0) = \lim_{h \to 0} \frac{f(x_0 + h, y_0) - f(x_0, y_0)}{h}
\]
\[
\frac{\partial f}{\partial y}(x_0, y_0) = \lim_{h \to 0} \frac{f(x_0, y_0 + h) - f(x_0, y_0)}{h}
\]

Частные производные — это производные функции по одной переменной при фиксированной другой.

\textbf{Геометрический смысл:} производная по $x$ — это скорость изменения функции вдоль оси $x$, при фиксированном $y$.

\textbf{Обозначения:}
\[
f_x = \frac{\partial f}{\partial x}, \quad f_y = \frac{\partial f}{\partial y}
\]

\textbf{Пример:}
Пусть $f(x, y) = x^2 y + \sin(xy)$. Тогда:
\[
\frac{\partial f}{\partial x} = 2xy + y \cos(xy)
\]
\[
\frac{\partial f}{\partial y} = x^2 + x \cos(xy)
\]

---

\subsection*{Частные производные высших порядков}

Можно вычислять производные второго порядка и выше:
\[
\frac{\partial^2 f}{\partial x^2}, \quad
\frac{\partial^2 f}{\partial y^2}, \quad
\frac{\partial^2 f}{\partial x \partial y}, \quad
\frac{\partial^2 f}{\partial y \partial x}
\]

Если $f$ дважды непрерывно дифференцируема, то:
\[
\frac{\partial^2 f}{\partial x \partial y} = \frac{\partial^2 f}{\partial y \partial x}
\]
(теорема Шварца о равенстве смешанных производных)

---

\subsection*{Градиент функции}

Пусть $f(x, y)$ — дифференцируемая функция. Тогда \textbf{градиент} функции $f$ в точке $(x_0, y_0)$ — это вектор:
\[
\nabla f(x_0, y_0) = \left(
\frac{\partial f}{\partial x}(x_0, y_0),
\frac{\partial f}{\partial y}(x_0, y_0)
\right)
\]

Если $f$ зависит от $n$ переменных, то градиент — вектор из $n$ компонент:
\[
\nabla f = \left(
\frac{\partial f}{\partial x_1}, \dots, \frac{\partial f}{\partial x_n}
\right)
\]

\textbf{Геометрический смысл:}
\begin{itemize}
  \item Направление градиента — это направление \textit{наибольшего роста функции}.
  \item Его длина — скорость наибольшего изменения.
  \item Если $\nabla f = \vec{0}$, то точка является критической (возможный экстремум).
\end{itemize}

\textbf{Пример:}
Пусть $f(x, y) = x^2 + y^2 \Rightarrow \nabla f = (2x, 2y)$ — вектор, указывающий от начала координат.

---

\subsection*{Производная функции по направлению}

Пусть функция $f(x, y)$ определена в точке $(x_0, y_0)$, и задан единичный вектор направления:
\[
\vec{l} = (\cos \alpha, \cos \beta), \quad \|\vec{l}\| = 1
\]

\textbf{Производная функции $f$ по направлению} вектора $\vec{l}$ в точке $(x_0, y_0)$:
\[
D_{\vec{l}} f(x_0, y_0) = \lim_{h \to 0} \frac{f(x_0 + h \cos \alpha, y_0 + h \cos \beta) - f(x_0, y_0)}{h}
\]

\textbf{Свойство:}
Если функция $f$ дифференцируема, то производная по направлению вычисляется через градиент:
\[
D_{\vec{l}} f = \nabla f \cdot \vec{l} = \left(
\frac{\partial f}{\partial x}, \frac{\partial f}{\partial y}
\right) \cdot (\cos \alpha, \cos \beta)
\]

Это \textbf{скалярное произведение} векторов: градиента и направления.

\textbf{Следствия:}
\begin{itemize}
  \item Наибольшая производная по направлению достигается в направлении градиента.
  \item Если $\vec{l} \perp \nabla f$, то производная по направлению равна нулю (функция не меняется вдоль этого направления).
\end{itemize}

\subsubsection*{Пример:}

Пусть $f(x, y) = x^2 y + y$, точка $M = (1, 2)$, направление $\vec{l} = \frac{1}{\sqrt{2}}(1, 1)$.

\[
\nabla f = \left(2x y, x^2 + 1\right) \Rightarrow
\nabla f(1,2) = (4, 2)
\]

\[
D_{\vec{l}} f = \nabla f \cdot \vec{l} = \frac{1}{\sqrt{2}}(4 + 2) = \frac{6}{\sqrt{2}} = 3\sqrt{2}
\]

---

\subsection*{Итоги}

\begin{itemize}
  \item Частные производные показывают, как функция меняется по каждой координате.
  \item Градиент — главный вектор изменения, указывает направление наибольшего роста.
  \item Производная по направлению обобщает понятие производной на произвольное направление.
  \item Всё вместе используется в оптимизации, градиентном спуске, анализе поверхности.
\end{itemize}

\section{Численные методы решения задач оптимизации. Метод Ньютона и секущей. Методы покоординатного и градиентного спуска}

Оптимизация — это процесс нахождения минимума или максимума функции при заданных ограничениях (или без них). В задачах прикладной математики часто встречаются функции, которые слишком сложны для аналитического нахождения экстремума, поэтому используют численные методы.

В данной секции мы рассмотрим:
\begin{itemize}
    \item Метод Ньютона;
    \item Метод секущей;
    \item Метод покоординатного спуска;
    \item Метод градиентного спуска.
\end{itemize}

\subsection{Постановка задачи оптимизации}
Пусть дана функция:
\[
f: \mathbb{R}^n \to \mathbb{R}, \quad f(\mathbf{x}) = f(x_1, x_2, \dots, x_n)
\]
Необходимо найти точку $\mathbf{x}^* \in \mathbb{R}^n$, такую что:
\[
f(\mathbf{x}^*) \le f(\mathbf{x}), \quad \forall \mathbf{x} \in \mathbb{R}^n
\]
(для задачи минимизации).

Для поиска минимума часто используют производные:
\begin{itemize}
    \item \textbf{Необходимое условие экстремума:} $\nabla f(\mathbf{x}^*) = 0$
    \item \textbf{Достаточное условие минимума:} матрица Гессе $H_f(\mathbf{x}^*)$ положительно определена.
\end{itemize}

\subsection{Метод Ньютона}
Метод Ньютона — это численный метод, который использует разложение функции в ряд Тейлора второго порядка для приближения к экстремуму.

\subsubsection{Алгоритм в одномерном случае}
Для уравнения $f'(x) = 0$:
\[
x_{k+1} = x_k - \frac{f'(x_k)}{f''(x_k)}
\]
Здесь:
\begin{itemize}
    \item $f'(x_k)$ — первая производная функции в точке $x_k$;
    \item $f''(x_k)$ — вторая производная.
\end{itemize}

\subsubsection{Многомерный случай}
Для векторной функции:
\[
\mathbf{x}_{k+1} = \mathbf{x}_k - H_f^{-1}(\mathbf{x}_k) \nabla f(\mathbf{x}_k)
\]
где $H_f(\mathbf{x}_k)$ — матрица Гессе.

\subsubsection{Плюсы и минусы}
\begin{itemize}
    \item \textbf{Плюсы:} Быстрая сходимость (обычно квадратичная).
    \item \textbf{Минусы:} Нужно вычислять и инвертировать матрицу Гессе, что дорого для больших $n$.
\end{itemize}

\subsubsection{Графическая иллюстрация}
\begin{center}
\begin{tikzpicture}[scale=1]
\draw[->] (-1,0) -- (4,0) node[right] {$x$};
\draw[->] (0,-1) -- (0,4) node[above] {$f(x)$};
\draw[domain=-0.5:3.5,smooth,variable=\x,blue] plot ({\x},{0.5*(\x-1.5)^2+1});
\draw[dashed, red] (0.5,0) -- (0.5,2) node[above] {$x_0$};
\draw[dashed, red] (1.25,0) -- (1.25,1.05) node[above] {$x_1$};
\end{tikzpicture}
\end{center}

\subsection{Метод секущей}
Метод секущей — это упрощённая версия метода Ньютона, в которой вторую производную заменяют приближением по разностям.

\[
x_{k+1} = x_k - f'(x_k) \cdot \frac{x_k - x_{k-1}}{f'(x_k) - f'(x_{k-1})}
\]

\begin{itemize}
    \item \textbf{Плюс:} Не нужно вычислять вторую производную.
    \item \textbf{Минус:} Скорость сходимости ниже, чем у метода Ньютона.
\end{itemize}

\subsection{Метод покоординатного спуска}
Метод покоординатного спуска оптимизирует функцию, изменяя одну координату за раз, оставляя остальные фиксированными.

\subsubsection{Алгоритм}
\begin{enumerate}
    \item Выбираем начальную точку $\mathbf{x}_0$.
    \item Для каждой координаты $i$ ищем минимум функции по $x_i$ при фиксированных остальных координатах.
    \item Повторяем процесс до сходимости.
\end{enumerate}

\subsubsection{Особенности}
\begin{itemize}
    \item Подходит для задач, где минимизация по одной переменной проста.
    \item Может сходиться медленно, если переменные сильно связаны.
\end{itemize}

\subsection{Метод градиентного спуска}
Метод градиентного спуска — один из самых популярных методов оптимизации.

\subsubsection{Основная идея}
Движемся из текущей точки в направлении, противоположном градиенту функции (т.к. градиент указывает направление наибольшего роста).

\[
\mathbf{x}_{k+1} = \mathbf{x}_k - \alpha \nabla f(\mathbf{x}_k)
\]
где $\alpha > 0$ — шаг обучения.

\subsubsection{Выбор шага}
\begin{itemize}
    \item Слишком большой $\alpha$ — можем «перепрыгнуть» минимум.
    \item Слишком маленький $\alpha$ — сходимость медленная.
\end{itemize}

\subsubsection{Графическая иллюстрация}
\begin{center}
\begin{tikzpicture}[scale=1]
\draw[->] (-2,0) -- (3,0) node[right] {$x$};
\draw[->] (0,-0.5) -- (0,4) node[above] {$f(x)$};
\draw[domain=-1.5:2.5,smooth,variable=\x,blue] plot ({\x},{(\x-1)^2+0.5});
\foreach \x in {-1, -0.2, 0.5, 0.8, 1} {
    \fill[red] (\x,{(\x-1)^2+0.5}) circle (2pt);
}
\end{tikzpicture}
\end{center}

\subsubsection{Варианты}
\begin{itemize}
    \item \textbf{С постоянным шагом}
    \item \textbf{С адаптивным шагом} (например, Adam, RMSprop)
\end{itemize}

\subsection{Заключение}
Выбор метода оптимизации зависит от свойств задачи:
\begin{itemize}
    \item Метод Ньютона — быстрый, но требует вычислений второй производной.
    \item Метод секущей — компромисс, подходит для случаев, когда вторая производная недоступна.
    \item Покоординатный спуск — полезен при раздельной оптимизации переменных.
    \item Градиентный спуск — универсальный, особенно в задачах машинного обучения.
\end{itemize}

\section{Основные понятия теории вероятностей. Определение вероятности. Вероятность случайных событий. Формула полной вероятности}

\subsection*{Введение и мотивация}
Теория вероятностей изучает случайные явления и формализует интуицию о «шансах» наступления событий. Классические примеры: подбрасывание монеты, бросок игральной кости, выбор случайного человека в опросе. Цель — построить строгую математику для рассуждений о вероятности событий, их сочетаниях и последствиях.

\bigskip
В этой секции мы подробно разберём:
\begin{itemize}
  \item базовые понятия: пространство исходов, события;
  \item определения вероятности (классическое, частотное, аксиоматическое);
  \item свойства вероятности (аддитивность, монотонность и т.д.);
  \item условную вероятность и независимость;
  \item формулу полной вероятности и практические примеры;
  \item теорему Байеса и применение формулы полной вероятности при вычислении апостериорных вероятностей;
  \item включение иллюстраций и задач на проверку.
\end{itemize}

\subsection{Пространство элементарных исходов и события}
\paragraph{Определение.} \emph{Пространство элементарных исходов} (универсум, sample space) обозначается $\Omega$ и содержит все возможные исходы случайного эксперимента. Каждый элемент $\omega\in\Omega$ называется \emph{элементарным исходом}.

\paragraph{События.} Событие $A$ — любое подмножество $\Omega$ (в базовом подходе). Если при проведении эксперимента полученный исход $\omega$ лежит в $A$, то говорят, что событие $A$ произошло.

\medskip
\textbf{Примеры.}
\begin{itemize}
  \item Подбрасывание монеты: $\Omega=\{\text{Орел},\ \text{Решка}\}$.
  \item Бросок правильного шестигранного кубика: $\Omega=\{1,2,3,4,5,6\}$.
  \item Случайный выбор человека из группы: $\Omega$ — множество людей группы.
\end{itemize}

\subsection{Классическое определение вероятности}
Если $\Omega$ состоит из конечного числа равновозможных элементарных исходов (классическая ситуация), то вероятность события $A\subseteq\Omega$ определяется как
\[
P(A) = \frac{|\{\omega\in\Omega:\ \omega\in A\}|}{|\Omega|}.
\]
То есть — отношение числа благоприятных исходов к общему числу исходов.

\textbf{Пример.} При броске справедливой кости вероятность того, что выпадет четное число:
\[
P(\text{четное}) = \frac{3}{6} = \frac{1}{2}.
\]

\subsection{Частотная (эмпирическая) интерпретация}
В частотном подходе вероятность события $A$ понимается как предел относительной частоты при многократном повторении эксперимента:
\[
P(A) = \lim_{n\to\infty} \frac{N_n(A)}{n},
\]
где $N_n(A)$ — количество опытов из $n$, в которых событие $A$ произошло (при предположении существования предела).

\subsection{Аксиоматическое определение (Колмогоров)}
Для общей теории наиболее строгой и удобной является аксиоматическая постановка.

\paragraph{Определение.} Пусть $\Omega$ — множество исходов, $\mathcal{F}$ — множество событий (обычно $\sigma$-алгебра над $\Omega$). Функция $P:\mathcal{F}\to[0,1]$ называется вероятностной мерой, если выполняются аксиомы Колмогорова:
\begin{enumerate}
  \item (Неотрицательность) $\forall A\in\mathcal{F}$: $P(A)\ge 0$;
  \item (Нормировка) $P(\Omega)=1$;
  \item (Счётная аддитивность) Для любых попарно несовместных событий $A_1,A_2,\dots\in\mathcal{F}$ выполняется
  \[
  P\Bigl(\bigcup_{i=1}^{\infty} A_i\Bigr) = \sum_{i=1}^{\infty} P(A_i).
  \]
\end{enumerate}

\paragraph{Примечание.} Для дискретных задач достаточно конечной аддитивности; для непрерывных процессов нужна счётная аддитивность.

\subsection{Следствия из аксиом (свойства вероятности)}
Из аксиом Колмогорова легко выводятся важные свойства:
\begin{enumerate}
  \item $P(\varnothing)=0$.
  \item Для любого $A\in\mathcal{F}$: $0\le P(A)\le 1$.
  \item Моночленость: если $A\subseteq B$, то $P(A)\le P(B)$.
  \item Формула суммы для двух событий:
  \[
  P(A\cup B) = P(A) + P(B) - P(A\cap B).
  \]
  (Доказательство: разложить объединение на попарно несовместные части.)
  \item Формула включения—исключения для трёх событий:
  \[
  P(A\cup B\cup C) = P(A)+P(B)+P(C)-P(A\cap B)-P(A\cap C)-P(B\cap C)+P(A\cap B\cap C).
  \]
\end{enumerate}

\subsection{Условная вероятность}
\paragraph{Определение.} Пусть $P(B)>0$. \emph{Условная вероятность} события $A$ при условии $B$ определяется как
\[
P(A\mid B) = \frac{P(A\cap B)}{P(B)}.
\]
Интуиция: мы рассматриваем пространство исходов, ограниченное тем, что произошло событие $B$, и оцениваем долю тех исходов в $B$, при которых произошло также $A$.

\paragraph{Свойства.}
\begin{itemize}
  \item Для фиксированного $B$ с $P(B)>0$, $P(\cdot\mid B)$ является вероятностной мерой на $\mathcal{F}$.
  \item Из определения следует формула для совместной вероятности:
  \[
  P(A\cap B) = P(B) P(A\mid B) = P(A) P(B\mid A).
  \]
\end{itemize}

\subsubsection*{Простейший пример условной вероятности}
Бросаем две монеты. Событие $A$ — «вторая монета — орёл», событие $B$ — «хотя бы одна монета — орёл». Найдём $P(A\mid B)$.

Исходы: $\Omega=\{HH,HT,TH,TT\}$.  
$A=\{HH,TH\}$, $B=\{HH,HT,TH\}$. Тогда
\[
P(A\mid B) = \frac{P(A\cap B)}{P(B)} = \frac{P(\{HH,TH\})}{P(\{HH,HT,TH\})} = \frac{2/4}{3/4} = \frac{2}{3}.
\]

\subsection{Независимость событий}
\paragraph{Два события.} События $A$ и $B$ называются (статистически) \emph{независимыми}, если
\[
P(A\cap B) = P(A)P(B).
\]
Эквивалентно: $P(A\mid B) = P(A)$ (при $P(B)>0$).

\paragraph{Несколько событий.} Система событий $\{A_i\}_{i\in I}$ называется взаимно независимой (или попарно и всесторонне независимой) если для любой конечной подсемьи $A_{i_1},\dots,A_{i_k}$ выполняется:
\[
P(A_{i_1}\cap \dots \cap A_{i_k}) = P(A_{i_1})\cdot \ldots \cdot P(A_{i_k}).
\]

\medskip
\textbf{Важно} — попарная независимость не влечёт взаимной независимости для трёх и более событий (пример с братьями и сестрами и т.п.).

\subsection{Формула полной вероятности (закон полной вероятности)}
\paragraph{Формулировка.} Пусть $\{H_1, H_2, \dots, H_n\}$ — разбиение пространства $\Omega$ (то есть попарно несовместные события с $\bigcup_{i=1}^n H_i = \Omega$) и $P(H_i)>0$ для всех $i$. Тогда для любого события $A$ выполнена формула полной вероятности:
\[
P(A) = \sum_{i=1}^n P(H_i) \, P(A\mid H_i).
\]

\paragraph{Доказательство (простой):} Так как $\{H_i\}$ — разбиение, имеем представление множества $A$ как объединение попарно несовместных множеств:
\[
A = \bigcup_{i=1}^n (A\cap H_i),
\]
поэтому по аддитивности:
\[
P(A) = \sum_{i=1}^n P(A\cap H_i) = \sum_{i=1}^n P(H_i) P(A\mid H_i).
\quad\blacksquare
\]

\paragraph{Интуиция.} Формула полной вероятности разлагает вероятность события $A$ по сценариям $H_i$ — каждому сценарию придаётся вероятность $P(H_i)$ и свой вклад $P(A\mid H_i)$.

\subsubsection*{Пример: заводы и брак (развернуто)}
Пусть изделие может быть произведено на одном из трёх заводов $H_1,H_2,H_3$ с вероятностями выпуска $P(H_1)=0.6$, $P(H_2)=0.3$, $P(H_3)=0.1$. Пусть вероятность брака на каждом заводе: $P(A\mid H_1)=0.01$, $P(A\mid H_2)=0.02$, $P(A\mid H_3)=0.05$. Найдём общую вероятность брака:

По формуле полной вероятности:
\[
P(A)=0.6\cdot0.01 + 0.3\cdot0.02 + 0.1\cdot0.05 = 0.006 + 0.006 + 0.005 = 0.017.
\]

\subsection{Теорема Байеса (формула для апостериорных вероятностей)}
\paragraph{Формулировка.} При тех же условиях, что и для формулы полной вероятности, для любого $j$ имеем:
\[
P(H_j\mid A) = \frac{P(H_j)P(A\mid H_j)}{\sum_{i=1}^n P(H_i)P(A\mid H_i)}.
\]
Это следует непосредственно из определения условной вероятности и формулы полной вероятности:
\[
P(H_j\mid A) = \frac{P(H_j\cap A)}{P(A)} = \frac{P(H_j)P(A\mid H_j)}{P(A)}.
\]

\paragraph{Интуиция.} Теорема Байеса позволяет обновить априорные вероятности $P(H_j)$ на основе наблюдения события $A$ и получить апостериорную вероятность того, что гипотеза $H_j$ истинна.

\subsubsection*{Пример (обратный пример завода)}
Используем данные предыдущего примера и предположим, что обнаружен брак (событие $A$). Найдём вероятность того, что изделие произведено на заводе $H_3$:
\[
P(H_3\mid A) = \frac{0.1\cdot0.05}{0.017} = \frac{0.005}{0.017} \approx 0.294.
\]
То есть, при обнаруженном браке вероятность, что изделие с завода №3, существенно увеличилась с 0.1 до ≈0.294.

\subsection{Дерево вероятностей (иллюстрация)}
\begin{center}
\begin{tikzpicture}[>=stealth,level distance=18mm,
  level 1/.style={sibling distance=30mm},
  level 2/.style={sibling distance=20mm}]
\node {$\Omega$}
  child { node {$H_1$}
    child { node {$A$} edge from parent node[left] {$P(A\mid H_1)$} }
    child { node {$\overline{A}$} edge from parent node[right] {$1-P(A\mid H_1)$} }
    edge from parent node[left] {$P(H_1)$}
  }
  child { node {$H_2$}
    child { node {$A$} edge from parent node[left] {$P(A\mid H_2)$} }
    child { node {$\overline{A}$} edge from parent node[right] {$1-P(A\mid H_2)$} }
    edge from parent node[right] {$P(H_2)$}
  };
\end{tikzpicture}
\end{center}

Дерево помогает визуально аккумулировать произведения вероятностей вдоль ветвей (например, $P(H_1\cap A)=P(H_1)P(A\mid H_1)$).

\subsection{Формула полной вероятности для непрерывного случая}
Если пространство условно разбивается по непрерывному параметру (или $H$ — событие с непрерывным индексом), то формула принимает интегральную форму. Например, если случайная величина $\Theta$ имеет плотность $p_\Theta(\theta)$ и при фиксированном $\theta$ наблюдается событие $A$ с условной вероятностью $P(A\mid \Theta=\theta)$, то
\[
P(A) = \int_{-\infty}^{\infty} P(A\mid \Theta=\theta)\, p_\Theta(\theta)\, d\theta.
\]

\subsection{Некоторые важные следствия и полезные формулы}
\begin{itemize}
  \item \textbf{Комментирование условной вероятности:} $P(A\mid B) + P(\overline{A}\mid B) = 1$ при $P(B)>0$.
  \item \textbf{Закон умножения для нескольких событий:}
  \[
  P(A\cap B\cap C) = P(A)P(B\mid A)P(C\mid A\cap B).
  \]
  \item \textbf{Полная формула включения—исключения} (общая) позволяет вычислить вероятность объединения любого конечного числа событий.
\end{itemize}

\subsection{Типичные ошибки и ловушки}
\begin{itemize}
  \item Ошибочно приравнивать несвязанные понятия: равновероятность исходов — это сильное условие, не выполняющееся в реальных задачах без проверки.
  \item Путаница между условной вероятностью $P(A\mid B)$ и $P(B\mid A)$ — они, как правило, не равны.
  \item Пренебрежение условием $P(B)>0$ при использовании условных вероятностей.
  \item Не различать попарную независимость и взаимную независимость.
\end{itemize}

\subsection{Упражнения для самопроверки (с ответами)}
\begin{enumerate}
  \item (\textbf{Простой}) Подбрасывают две честные монеты. Какова вероятность того, что выпадет ровно один орёл? \\
  \textit{Решение:} исходы $\{HH,HT,TH,TT\}$, благоприятные: $\{HT,TH\}$, $P=2/4=1/2$.

  \item (\textbf{Формула полной вероятности}) Два источника генерируют сообщения: источник 1 — с вероятностью 0.7, источник 2 — с вероятностью 0.3. Вероятность ошибки в сообщении для 1-го — 0.01, для 2-го — 0.05. Какова общая вероятность ошибки? \\
  \textit{Решение:} $P(\text{ошиб.})=0.7\cdot0.01+0.3\cdot0.05=0.007+0.015=0.022$.

  \item (\textbf{Байес}) С учётом предыдущей задачи: найдите вероятность того, что сообщение пришло от источника 2, если оно оказалось ошибочным. \\
  \textit{Решение:} $P(H_2\mid \text{ошиб.}) = \dfrac{0.3\cdot0.05}{0.022} = \dfrac{0.015}{0.022}\approx0.6818$.

  \item (\textbf{Независимость}) Два броска правильной кости: событие $A$ — «в первом броске выпало 6», событие $B$ — «во втором выпало 6». Независимы ли $A$ и $B$? \\
  \textit{Решение:} Да, $P(A)=1/6$, $P(B)=1/6$, $P(A\cap B)=1/36 = P(A)P(B)$.

  \item (\textbf{Усложнённая}) В урне 10 шаров: 4 белых, 6 чёрных. Два шара извлекают без возвращения. Найдите вероятность того, что оба белые. \\
  \textit{Решение:} $P = \dfrac{4}{10}\cdot \dfrac{3}{9} = \dfrac{12}{90} = \dfrac{2}{15}$.
\end{enumerate}

\subsection{Короткие доказательства (наиболее нужные для экзамена)}
\paragraph{Доказательство: $P(A\cup B)=P(A)+P(B)-P(A\cap B)$.}  
Разложим объединение на непересекающиеся части:
\[
A\cup B = A \cup (B\setminus A),
\]
где $A$ и $B\setminus A$ попарно несовместны. Следовательно,
\[
P(A\cup B) = P(A) + P(B\setminus A) = P(A) + \bigl(P(B) - P(A\cap B)\bigr).
\]
Откуда требуется равенство. \qed

\paragraph{Доказательство формулы полной вероятности.}  
Дано разбиение $\{H_i\}$. Так как $A = \bigcup_i (A\cap H_i)$ — объединение попарно несовместных множеств, применяем аддитивность и получаем
\[
P(A)=\sum_i P(A\cap H_i) = \sum_i P(H_i)P(A\mid H_i).
\]
\qed

\subsection{Резюме — что запомнить точно}
\begin{itemize}
  \item Аксиомы Колмогорова — основа теории.
  \item Условная вероятность: $P(A\mid B)=P(A\cap B)/P(B)$.
  \item Формула полной вероятности для разбиения пространства.
  \item Теорема Байеса для обратных вероятностей.
  \item Различие между несовместностью и независимостью.
\end{itemize}

\subsection{Если хочешь — углубимся дальше}
Могу дополнить секцию подробными темами:
\begin{itemize}
  \item теория случайных величин (дискретные и непрерывные), плотности и функции распределения;
  \item математическое ожидание, дисперсия, ковариация;
  \item законы больших чисел и центральная предельная теорема;
  \item байесовский вывод и примеры с непрерывными априорными распределениями;
  \item более сложные практические задачи (моделирование, имитация Монте-Карло).
\end{itemize}

\bigskip
Если хочешь — сейчас разверну каждую подпункту ещё глубже (больше доказательств, задач, графиков и примеров) и подготовлю версию формата «тезисы + примеры + тесты» — говори, в каком виде удобнее: учебный конспект, задания с решениями или шпаргалка.

\section{Понятие случайной величины. Функция плотности распределения и её свойства. Дискретные и непрерывные законы распределения, их свойства}

\subsection*{1. Интуиция и формальное определение случайной величины}

Интуитивно \textbf{случайная величина} — это правило, которое каждому исходу случайного эксперимента сопоставляет некоторое числовое значение. Например, при броске монеты можно сопоставить ``орёл'' → 1, ``решка'' → 0; при броске кости — само выпавшее число.

Формально: пусть $(\Omega,\mathcal{F},P)$ — вероятностное пространство (пространство элементарных исходов $\Omega$, $\sigma$-алгебра событий $\mathcal{F}$ и вероятность $P$). Тогда случайная величина — это измеримая функция
\[
X:\Omega\to\mathbb{R},
\]
то есть для любого промежутка (или борелевского множества) $B\subseteq\mathbb{R}$ множество $\{\omega\in\Omega:\ X(\omega)\in B\}$ должно быть событием (лежать в $\mathcal{F}$).

\medskip
Классификация по типу значений:
\begin{itemize}
  \item \textbf{Дискретная} случайная величина принимает не более счётного множества значений.
  \item \textbf{Непрерывная} (абсолютно непрерывная) имеет плотность распределения относительно меры Лебега (нет атомов точной массы).
  \item \textbf{Смешанная} — содержит как дискретную, так и непрерывную составляющие.
\end{itemize}

\subsection*{2. Функция распределения (CDF) — основа описания закона}

Для произвольной случайной величины $X$ её \textbf{функция распределения} (Cumulative Distribution Function, CDF) определяется как
\[
F_X(x) := P(X \le x),\quad x\in\mathbb{R}.
\]
Функция распределения полностью задаёт закон случайной величины (включая дискретные и непрерывные части).

\medskip
\textbf{Ключевые свойства $F_X(x)$:}
\begin{enumerate}
  \item $F_X$ монотонно неубывает: если $x_1\le x_2$ то $F_X(x_1)\le F_X(x_2)$.
  \item Правосторонняя непрерывность: $\displaystyle \lim_{t\downarrow x} F_X(t)=F_X(x)$.
  \item Пределы на бесконечностях: $\displaystyle \lim_{x\to -\infty} F_X(x)=0,\quad \lim_{x\to +\infty} F_X(x)=1$.
  \item Для любых $a<b$ выполнено $P(a<X\le b)=F_X(b)-F_X(a)$.
\end{enumerate}

\medskip
\textbf{Атомы (точечные массы).} Если в точке $x_0$ сразу возникает положительный скачок $p_0=F_X(x_0)-\lim_{x\to x_0^-}F_X(x)$, то $P(X=x_0)=p_0>0$ — это дискретная (атомная) часть распределения.

\subsection*{3. Дискретные законы распределения}

\paragraph{Определение.} Случайная величина $X$ называется дискретной, если существует (счётное) набор значений $\{x_k\}$ такой, что $P(X\in\{x_k\})=1$. Тогда её закон задаётся \textbf{функцией вероятности} (PMF):
\[
p_X(x_k) := P(X=x_k),\qquad \sum_k p_X(x_k)=1,\quad p_X(x_k)\ge0.
\]

\paragraph{Ожидание и дисперсия.} Если суммы сходятся абсолютно, определяются математическое ожидание и дисперсия:
\[
\mathbb{E}[X] = \sum_k x_k p_X(x_k),\qquad
\operatorname{Var}(X)=\mathbb{E}[(X-\mathbb{E}X)^2]=\sum_k (x_k-\mathbb{E}X)^2 p_X(x_k).
\]

\paragraph{Типичные примеры (с формулами и краткой интерпретацией):}
\begin{itemize}
  \item \textbf{Бернуллиевская (Bernoulli)}: $X\in\{0,1\}$, $P(X=1)=p$, $P(X=0)=1-p$. \\
    $\mathbb{E}X=p$, $\operatorname{Var}(X)=p(1-p)$.
  \item \textbf{Биномиальная (Binomial)} $X\sim Bin(n,p)$: $P(X=k)=\binom{n}{k}p^k(1-p)^{n-k}$ для $k=0,\dots,n$. \\
    $\mathbb{E}X=np$, $\operatorname{Var}(X)=np(1-p)$. Модель: $n$ независимых испытаний Бернулли.
  \item \textbf{Геометрическая (Geometric)}: $P(X=k)=(1-p)^{k-1}p$ (номер первого успеха). $\mathbb{E}X=\frac{1}{p}$.
  \item \textbf{Пуассоновская (Poisson)} $X\sim Pois(\lambda)$: $P(X=k)=e^{-\lambda}\frac{\lambda^k}{k!}$, $k\ge0$. \\
    Это предел биномиального при $n\to\infty$, $p\to0$, $np\to\lambda$. $\mathbb{E}X=\lambda$, $\operatorname{Var}(X)=\lambda$.
\end{itemize}

\paragraph{Генерирующие функции (полезный инструмент).}
\begin{itemize}
  \item \textbf{Моментная функция (MGF):} $M_X(t)=\mathbb{E}[e^{tX}] = \sum_k e^{t x_k} p_X(x_k)$.
  \item \textbf{Функция порождающая (PGF) для неотрицательных целых:} $G_X(s)=\mathbb{E}[s^X] = \sum_{k\ge0} s^k p_X(k)$.
\end{itemize}
MGF и PGF удобны для вычисления моментов и сумм независимых случайных величин.

\subsection*{4. Непрерывные законы распределения и функция плотности (PDF)}

\paragraph{Определение.} В случае, когда $X$ непрерывна, её закон, как правило, задаётся \textbf{функцией плотности} $f_X(x)$ (PDF), такой что для любых множеств
\[
P(a<X\le b) = \int_a^b f_X(x)\,dx.
\]
При этом $f_X(x)\ge0$ почти везде и
\[
\int_{-\infty}^{\infty} f_X(x)\,dx = 1.
\]

\paragraph{Связь CDF и PDF.} Если $F_X$ дифференцируема в точке $x$, то $f_X(x)=F_X'(x)$. В общем случае $F_X$ может содержать дискретные скачки и непрерывную часть; тогда $F_X$ раскладывается в сумму атомов и интегральной части.

\paragraph{Ожидание и дисперсия (непрерывный случай):}
\[
\mathbb{E}[X] = \int_{-\infty}^{\infty} x f_X(x)\,dx,\qquad
\operatorname{Var}(X) = \int (x-\mathbb{E}X)^2 f_X(x)\,dx.
\]

\paragraph{Типичные непрерывные распределения:}
\begin{itemize}
  \item \textbf{Равномерное} $U(a,b)$: $f(x)=\dfrac{1}{b-a}$ при $a\le x\le b$, иначе $0$. $\mathbb{E}X=\dfrac{a+b}{2}$, $\operatorname{Var}(X)=\dfrac{(b-a)^2}{12}$.
  \item \textbf{Экспоненциальное} $Exp(\lambda)$: $f(x)=\lambda e^{-\lambda x}$ при $x\ge0$. Памятьlessness: $P(X>t+s\mid X>t)=P(X>s)$. $\mathbb{E}X=1/\lambda$, $\operatorname{Var}(X)=1/\lambda^2$.
  \item \textbf{Нормальное} $N(\mu,\sigma^2)$: $f(x)=\dfrac{1}{\sqrt{2\pi}\sigma}\exp\bigl(-\dfrac{(x-\mu)^2}{2\sigma^2}\bigr)$ для $x\in\mathbb{R}$. $\mathbb{E}X=\mu$, $\operatorname{Var}(X)=\sigma^2$. Центральная предельная теорема делает нормальное распределение фундаментальным.
  \item \textbf{Гамма, Бета, Коши} и др. — семейства с разными формами плотностей, полезные в разных задачах.
\end{itemize}

\paragraph{Пример вычислений (равномерное и экспоненциальное).}
\begin{itemize}
  \item $X\sim U(0,1)$: $\mathbb{E}[X]=\int_0^1 x\,dx=\tfrac12$, $\operatorname{Var}(X)=\int_0^1 (x-\tfrac12)^2 dx=\tfrac{1}{12}$.
  \item $X\sim Exp(\lambda)$: $\mathbb{E}[X]=\int_0^\infty x\lambda e^{-\lambda x}dx = \lambda\cdot \frac{1}{\lambda^2} = \frac{1}{\lambda}$ (полезно интегрировать по частям).
\end{itemize}

\subsection*{5. Смешанные распределения}

Реально встречаются законы, у которых есть и дискретная часть (атомы), и плотность. Тогда CDF распадается:
\[
F_X(x) = \sum_{x_k\le x} p_k + \int_{-\infty}^x f_{\text{cont}}(t)\,dt,
\]
где $p_k=P(X=x_k)$ — массы в точках, а $f_\text{cont}$ — плотность непрерывной части. Часто такие случаи возникают, например, при моделировании с выпадением особого события (атом) плюс «обычный» непрерывный шум.

\subsection*{6. Характеристики распределения: мода, медиана, квантили, моментные характеристики}

\begin{itemize}
  \item \textbf{Мода} — значение $x$ (необязательно единственное), в котором плотность (или PMF) достигает максимума.
  \item \textbf{Медиана} $m$ — решение $F_X(m)\ge 1/2$ и $F_X(m-)\le 1/2$; для непрерывных распределений часто единственна.
  \item \textbf{Квантили:} $q_\alpha = \inf\{x: F_X(x)\ge \alpha\}$.
  \item \textbf{Моменты:} $\mathbb{E}[X^k]$ при существовании; центральные моменты $\mathbb{E}[(X-\mathbb{E}X)^k]$; моментная функция $M_X(t)=\mathbb{E}[e^{tX}]$ (если существует в окрестности $0$).
\end{itemize}

Эти характеристики используются для описания асимметрии (скос), крутизны (эксцесс) и т.д.

\subsection*{7. Преобразования случайных величин}

\paragraph{Дискретный случай.} Если $X$ дискретна с $P(X=x_k)=p_k$, а $Y=g(X)$, то
\[
P(Y=y) = \sum_{k:\ g(x_k)=y} p_k.
\]

\paragraph{Непрерывный случай (монотонная функция).} Пусть $X$ имеет плотность $f_X$ и $Y=g(X)$, где $g$ монотонна и дифференцируема. Тогда плотность $f_Y$ для $y=g(x)$ даётся формулой
\[
f_Y(y) = f_X\bigl(g^{-1}(y)\bigr)\, \left| \frac{d}{dy} g^{-1}(y)\right|.
\]
Для многомерного случая используется якобиан преобразования.

\paragraph{Пример (известный):} Если $X\sim U(0,1)$ и $Y=-\ln X$, то $Y\sim Exp(1)$. Действительно, $P(Y\le y)=P(-\ln X\le y)=P(X\ge e^{-y})=1-e^{-y}$ для $y\ge0$, дифференцируя получаем плотность $\;f_Y(y)=e^{-y}\,$.

\subsection*{8. Совместные распределения, маргинальные и условные законы (кратко)}

Хотя основной вопрос — одномерные законы, важно упомянуть:
\begin{itemize}
  \item Для вектора случайных величин $(X,Y)$ задаётся \textbf{совместная} PMF или PDF $p_{X,Y}(x,y)$ или $f_{X,Y}(x,y)$.
  \item \textbf{Маргинальная} плотность: $f_X(x)=\int f_{X,Y}(x,y)\,dy$ (аналогично для дискретного: суммирование).
  \item \textbf{Условная} плотность: $f_{Y\mid X}(y\mid x)=\dfrac{f_{X,Y}(x,y)}{f_X(x)}$ при $f_X(x)>0$.
  \item Независимость: $X$ и $Y$ независимы ⇔ $f_{X,Y}(x,y)=f_X(x)f_Y(y)$ (или для дискретного — $p_{X,Y}(x,y)=p_X(x)p_Y(y)$).
\end{itemize}

\subsection*{9. Заключение — практические советы и «чек-лист» для экзамена}

\begin{itemize}
  \item \textbf{Запомнить определения:} CDF $F_X$, PMF $p_X$ для дискретных, PDF $f_X$ для непрерывных.
  \item \textbf{Свойства:} $F$ монотонен, правосторонне непрерывен, пределы 0 и 1; $p\ge0,\ \sum p=1$; $f\ge0,\ \int f=1$.
  \item \textbf{Переходы:} $F'(x)=f(x)$ (когда существует), $P(a<X\le b)=F(b)-F(a)$.
  \item \textbf{Частые формулы:} $\mathbb{E}[X]=\sum x_k p_k$ (дискретно) или $\int x f(x)\,dx$ (непрерывно).
  \item \textbf{Преобразования:} дискретный — суммирование по прообразам; непрерывный — замена переменной с модулем якобиана.
  \item \textbf{Типовые распределения:} знать формулы PMF/PDF, ожидание и дисперсию для Bernoulli, Binomial, Poisson, Geometric, Uniform, Exponential, Normal.
\end{itemize}

\subsection*{10. Задачи для закрепления (с краткими подсказками)}

\begin{enumerate}
  \item Докажите, что функция распределения $F_X$ права-непрерывна и монотонна. (\emph{Подсказка: используйте свойства вероятности и представление $F(x)=P(X\le x)$}.)
  \item Пусть $X\sim Bin(n,p)$. Найдите MGF $M_X(t)$ и используйте её, чтобы получить $\mathbb{E}X$ и $\operatorname{Var}(X)$. (\emph{Подсказка: $M_X(t)=(1-p+pe^t)^n$}.)
  \item Для $X\sim N(0,1)$ доказать, что плотность интегрируется в $1$ (можно сослаться на табличный интеграл или заменить в полярных координатах). (\emph{Подсказка: вычислите $I^2=\left(\int e^{-x^2/2}dx\right)^2$ через двойной интеграл}.)
  \item Пусть $X$ имеет CDF $F$. Покажите, что $P(X=a)=F(a)-\lim_{x\to a^-}F(x)$. (\emph{Подсказка: используйте определение вероятности точки как разности $P(X\le a)-P(X<a)$}.)
\end{enumerate}

\bigskip
Если хочешь, я могу:
\begin{itemize}
  \item развернуть этот материал ещё глубже (формулы для моментов более высоких порядков, характеристические функции, теорема Лебега о разложении мер в дискретную и абсолютно непрерывную части);
  \item подготовить набор типовых экзаменационных задач с полными решениями;
  \item сгенерировать иллюстрации в TikZ для PMF (столбиковые диаграммы) и PDF (кривые) для каждого приведённого распределения.
\end{itemize}
Скажи, что делаем дальше — рисуем графики или сразу задачи с подробными решениями?

\section{Наиболее употребимые теоретические законы распределения вероятностей. Примеры и свойства распределений для дискретных и непрерывных величин}

В теории вероятностей \textbf{закон распределения} случайной величины описывает, какие значения она может принимать и с какой вероятностью.  
Все законы делятся на два больших класса: \textbf{дискретные} и \textbf{непрерывные}.

\subsection{Дискретные законы распределения}

Дискретная случайная величина может принимать конечное или счётное число значений. Закон распределения задаётся таблицей или функцией вероятности:
\[
P(X = x_i) = p_i, \quad p_i \geq 0, \quad \sum_{i} p_i = 1.
\]

\subsubsection{Распределение Бернулли}
Описывает исход одного эксперимента с двумя результатами: ``успех'' (1) с вероятностью $p$ и ``неудача'' (0) с вероятностью $q = 1 - p$.
\[
P(X = 1) = p, \quad P(X = 0) = 1 - p.
\]
\textbf{Математическое ожидание:} $E[X] = p$.  
\textbf{Дисперсия:} $D[X] = p(1-p)$.

\begin{center}
\begin{tikzpicture}
\begin{axis}[
    ybar,
    symbolic x coords={0,1},
    xtick=data,
    ymin=0, ymax=1,
    xlabel={$X$}, ylabel={$P(X)$},
    width=7cm, height=5cm
]
\addplot coordinates {(0,0.3) (1,0.7)};
\end{axis}
\end{tikzpicture}
\end{center}

\subsubsection{Биномиальное распределение}
Описывает количество успехов в $n$ независимых испытаниях Бернулли с вероятностью успеха $p$.
\[
P(X = k) = \binom{n}{k} p^k (1-p)^{n-k}, \quad k = 0,1,\dots,n.
\]
\textbf{Математическое ожидание:} $E[X] = np$.  
\textbf{Дисперсия:} $D[X] = np(1-p)$.

\begin{center}
\begin{tikzpicture}
\begin{axis}[
    ybar,
    xtick=data,
    xlabel={$k$}, ylabel={$P(X=k)$},
    width=9cm, height=5cm
]
\addplot coordinates {
(0,0.002) (1,0.014) (2,0.059) (3,0.146) (4,0.242) (5,0.242) (6,0.146) (7,0.059) (8,0.014) (9,0.002)
};
\end{axis}
\end{tikzpicture}
\end{center}

\subsubsection{Геометрическое распределение}
Вероятность того, что первый успех произойдёт на $k$-м испытании:
\[
P(X = k) = (1-p)^{k-1}p, \quad k = 1,2,3,\dots
\]
\textbf{Математическое ожидание:} $E[X] = \frac{1}{p}$.  
\textbf{Дисперсия:} $D[X] = \frac{1-p}{p^2}$.

\subsection{Непрерывные законы распределения}

Непрерывная случайная величина может принимать любое значение на отрезке или на всей числовой прямой.  
Её распределение задаётся функцией плотности вероятности $f(x)$, удовлетворяющей условиям:
\[
f(x) \geq 0, \quad \int_{-\infty}^{\infty} f(x)\,dx = 1.
\]

\subsubsection{Равномерное распределение}
Если случайная величина $X$ равновероятно принимает значения на отрезке $[a,b]$, то
\[
f(x) = \frac{1}{b-a}, \quad a \leq x \leq b.
\]
\textbf{Математическое ожидание:} $E[X] = \frac{a+b}{2}$.  
\textbf{Дисперсия:} $D[X] = \frac{(b-a)^2}{12}$.

\begin{center}
\begin{tikzpicture}
\begin{axis}[
    xlabel={$x$}, ylabel={$f(x)$},
    width=8cm, height=5cm,
    ymin=0, ymax=1
]
\addplot+[ycomb, thick] coordinates {(1,0.5) (4,0.5)};
\addplot[domain=1:4, samples=2, thick] {0.5};
\end{axis}
\end{tikzpicture}
\end{center}

\subsubsection{Нормальное распределение}
Наиболее распространённое в природе и статистике.  
Функция плотности:
\[
f(x) = \frac{1}{\sigma \sqrt{2\pi}} e^{-\frac{(x-\mu)^2}{2\sigma^2}}, \quad -\infty < x < \infty.
\]
\textbf{Математическое ожидание:} $E[X] = \mu$.  
\textbf{Дисперсия:} $D[X] = \sigma^2$.

\begin{center}
\begin{tikzpicture}
\begin{axis}[
    xlabel={$x$}, ylabel={$f(x)$},
    width=8cm, height=5cm
]
\addplot[domain=-4:4, samples=100, thick] {1/sqrt(2*pi) * exp(-x^2/2)};
\end{axis}
\end{tikzpicture}
\end{center}

\subsubsection{Экспоненциальное распределение}
Часто описывает время ожидания между событиями.
\[
f(x) = \lambda e^{-\lambda x}, \quad x \geq 0.
\]
\textbf{Математическое ожидание:} $E[X] = \frac{1}{\lambda}$.  
\textbf{Дисперсия:} $D[X] = \frac{1}{\lambda^2}$.

\begin{center}
\begin{tikzpicture}
\begin{axis}[
    xlabel={$x$}, ylabel={$f(x)$},
    width=8cm, height=5cm
]
\addplot[domain=0:5, samples=100, thick] {exp(-x)};
\end{axis}
\end{tikzpicture}
\end{center}

\subsection{Выводы и сравнение}
Каждое распределение имеет свои особенности и применяется в определённых задачах:
\begin{itemize}
    \item Бернулли и биномиальное --- для дискретных экспериментов с успехами и неудачами.
    \item Геометрическое --- для моделирования числа попыток до первого успеха.
    \item Равномерное --- когда все значения равновероятны.
    \item Нормальное --- в большинстве природных и социальных явлений.
    \item Экспоненциальное --- для моделирования времени ожидания.
\end{itemize}

\section{Выборочные характеристики разброса и центральной тенденции дискретных и непрерывных случайных величин}

\subsection{Введение в выборочные характеристики}
В статистике важную роль играет описание данных с помощью характеристик, которые позволяют сделать выводы о распределении случайной величины на основе выборки.  
Под \textbf{выборочными характеристиками} понимают числовые показатели, вычисленные по данным выборки, которые используются для оценки свойств генеральной совокупности.  

Все характеристики можно условно разделить на две большие группы:
\begin{itemize}
    \item \textbf{Характеристики центральной тенденции} — показывают, вокруг каких значений сосредоточены наблюдения.
    \item \textbf{Характеристики разброса} — показывают, насколько сильно наблюдения отклоняются от центра.
\end{itemize}

\subsection{Центральная тенденция}
Центральная тенденция описывает "середину" данных, то есть значение, около которого сконцентрированы результаты.

\subsubsection{Выборочное среднее}
Выборочное среднее $\overline{x}$ — это сумма всех элементов выборки, делённая на их количество:
\[
\overline{x} = \frac{1}{n} \sum_{i=1}^n x_i
\]
Где:
\begin{itemize}
    \item $n$ — объём выборки
    \item $x_i$ — $i$-е наблюдение
\end{itemize}
Смысл: выборочное среднее — оценка математического ожидания генеральной совокупности.

\subsubsection{Медиана}
Медиана — значение, которое делит упорядоченные данные на две равные части:
\begin{itemize}
    \item Половина значений меньше или равна медиане
    \item Половина — больше или равна медиане
\end{itemize}
Для нечётного $n$: медиана — это значение с индексом $\frac{n+1}{2}$.  
Для чётного $n$: медиана — среднее арифметическое двух средних элементов.

\subsubsection{Мода}
Мода — значение, которое встречается чаще всего.  
Если все значения встречаются одинаково часто, то мода может отсутствовать или быть не единственной.

\subsection{Характеристики разброса}
Разброс характеризует, насколько сильно значения данных отклоняются от среднего.

\subsubsection{Размах}
Размах $R$ — разница между максимальным и минимальным значениями:
\[
R = x_{\max} - x_{\min}
\]
Показывает диапазон значений, но не учитывает их распределение.

\subsubsection{Выборочная дисперсия}
Выборочная дисперсия $S^2$ — среднее квадратов отклонений значений от выборочного среднего:
\[
S^2 = \frac{1}{n-1} \sum_{i=1}^n (x_i - \overline{x})^2
\]
Используем $n-1$ в знаменателе для получения несмещённой оценки дисперсии.

\subsubsection{Выборочное стандартное отклонение}
Стандартное отклонение $S$ — квадратный корень из дисперсии:
\[
S = \sqrt{S^2}
\]
Показывает среднее отклонение данных от среднего значения в тех же единицах, что и сами данные.

\subsubsection{Коэффициент вариации}
Коэффициент вариации $V$ — относительная мера разброса:
\[
V = \frac{S}{\overline{x}} \cdot 100\%
\]
Позволяет сравнивать вариацию в разных выборках, даже если их средние сильно различаются.

\subsection{Дискретные и непрерывные случайные величины}
\subsubsection{Дискретная случайная величина}
Если случайная величина $X$ принимает конечное или счётное множество значений $x_1, x_2, \dots, x_k$, то выборочные характеристики вычисляются по прямым формулам, приведённым выше.

\subsubsection{Непрерывная случайная величина}
Для непрерывных величин выборочные характеристики вычисляются так же, но на практике используют дискретное приближение (интервалы значений).  
При больших объёмах данных гистограммы и интегральные графики помогают визуализировать тенденции.

\subsection{Пример вычислений}
Пусть имеется выборка: $3, 5, 7, 5, 9$.
\begin{itemize}
    \item Выборочное среднее:
    \[
    \overline{x} = \frac{3+5+7+5+9}{5} = \frac{29}{5} = 5.8
    \]
    \item Медиана: $5$ (середина упорядоченного ряда $3, 5, 5, 7, 9$)
    \item Мода: $5$ (встречается дважды)
    \item Размах: $R = 9 - 3 = 6$
    \item Дисперсия:
    \[
    S^2 = \frac{(3-5.8)^2 + (5-5.8)^2 + (7-5.8)^2 + (5-5.8)^2 + (9-5.8)^2}{5-1} = \frac{7.84 + 0.64 + 1.44 + 0.64 + 10.24}{4} = \frac{20.8}{4} = 5.2
    \]
    \item Стандартное отклонение:
    \[
    S = \sqrt{5.2} \approx 2.28
    \]
    \item Коэффициент вариации:
    \[
    V \approx \frac{2.28}{5.8} \cdot 100\% \approx 39.3\%
    \]
\end{itemize}

\subsection{Заключение}
Выборочные характеристики центральной тенденции и разброса — это основа описательной статистики.  
Они позволяют понять структуру данных, выявить закономерности и сравнить разные выборки.  
В дискретных и непрерывных случаях подход к вычислению одинаков, но в непрерывных задачах часто используют аппроксимации и графические методы.

% End of sections

\end{document}

