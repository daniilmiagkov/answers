\section{Векторы. Линейная зависимость системы векторов. Базис линейного пространства. Скалярное произведение векторов}

\subsection{Векторы}

\textbf{Вектор} — это упорядоченный набор чисел (координат), который характеризуется направлением и величиной. Векторы принято обозначать буквами со стрелкой: $\vec{v}$, $\vec{a}$, $\vec{u}$.

Примеры векторов в $\mathbb{R}^2$ и $\mathbb{R}^3$:
\[
\vec{v} = \begin{pmatrix}1 \\ 2\end{pmatrix}, \quad
\vec{u} = \begin{pmatrix}3 \\ -1 \\ 4\end{pmatrix}.
\]

\textbf{Нуле́вой вектор} — это вектор, все координаты которого равны нулю:
\[
\vec{0} = \begin{pmatrix}0 \\ 0 \\ \dots \\ 0\end{pmatrix}.
\]

\subsubsection{Операции над векторами}

\begin{itemize}[leftmargin=*]
  \item \textbf{Сложение}: если $\vec{a}=(a_1,\dots,a_n)$ и $\vec{b}=(b_1,\dots,b_n)$, то
    \[
      \vec{a} + \vec{b} = (a_1+b_1,\; \dots,\; a_n+b_n).
    \]
  \item \textbf{Умножение на скаляр}: для $\lambda\in\mathbb{R}$ и $\vec{a}=(a_1,\dots,a_n)$
    \[
      \lambda \vec{a} = (\lambda a_1,\; \dots,\; \lambda a_n).
    \]
  \item \textbf{Противоположный вектор}: $-\vec{a} = (-1)\vec{a}$, при сложении даёт нулевой вектор.
\end{itemize}

\subsection{Линейная зависимость системы векторов}

Пусть заданы векторы $\vec{v}_1, \vec{v}_2, \dots, \vec{v}_k$ в векторном пространстве $V$ над полем $\mathbb{R}$. Рассмотрим их линейную комбинацию:
\[
\lambda_1 \vec{v}_1 + \lambda_2 \vec{v}_2 + \dots + \lambda_k \vec{v}_k = \vec{0}, 
\quad \lambda_i \in \mathbb{R}.
\]

\textbf{Система векторов} называется
\begin{itemize}[leftmargin=*]
  \item \textit{линейно зависимой}, если существуют коэффициенты $\lambda_i$, не все равные нулю, такие что комбинация равна нулевому вектору.
  \item \textit{линейно независимой}, если единственное решение $\lambda_1 \vec{v}_1 + \dots + \lambda_k \vec{v}_k = \vec{0}$ — это $\lambda_1 = \dots = \lambda_k = 0$.
\end{itemize}

\textbf{Пример.} Векторы в $\mathbb{R}^3$
\[
\vec{v}_1 = \begin{pmatrix}1\\0\\0\end{pmatrix},\quad
\vec{v}_2 = \begin{pmatrix}0\\1\\0\end{pmatrix},\quad
\vec{v}_3 = \begin{pmatrix}1\\1\\0\end{pmatrix}
\]
линейно зависимы, так как
\[
1\cdot \vec{v}_1 + 1\cdot \vec{v}_2 -1\cdot \vec{v}_3 = \vec{0},
\]
и при этом коэффициенты не все нули.

\subsection{Базис линейного пространства}

\textbf{Базис} линейного пространства $V$ — это упорядоченная система векторов $\{\vec{e}_1, \dots, \vec{e}_n\}$, обладающая двумя свойствами:
\begin{enumerate}[label=\arabic*)]
  \item \textbf{Линейная независимость}: ни один из базисных векторов не выражается через другие.
  \item \textbf{Порождающий (образующий) набор}: любой вектор $\vec{v}\in V$ можно единственным образом разложить в виде
    \[
      \vec{v} = x_1 \vec{e}_1 + \dots + x_n \vec{e}_n, 
      \quad x_i \in \mathbb{R}.
    \]
\end{enumerate}

Число $n$ называется \textbf{размерностью} пространства $V$ и совпадает с количеством векторов в любом базисе $V$.

\textbf{Пример.} В стандартном базисе $\mathbb{R}^3$:
\[
\vec{e}_1 = \begin{pmatrix}1\\0\\0\end{pmatrix},\quad
\vec{e}_2 = \begin{pmatrix}0\\1\\0\end{pmatrix},\quad
\vec{e}_3 = \begin{pmatrix}0\\0\\1\end{pmatrix}.
\]
Любой вектор $\vec{v}=(v_1,v_2,v_3)$ раскладывается как
\[
\vec{v} = v_1\vec{e}_1 + v_2\vec{e}_2 + v_3\vec{e}_3.
\]

\subsection{Скалярное произведение векторов}

Для векторов $\vec{a}=(a_1,\dots,a_n)$ и $\vec{b}=(b_1,\dots,b_n)$ в $\mathbb{R}^n$ \textbf{скалярное произведение} определяется как
\[
\langle \vec{a}, \vec{b} \rangle = a_1 b_1 + a_2 b_2 + \dots + a_n b_n.
\]

Основные свойства:
\begin{itemize}[leftmargin=*]
  \item \textbf{Коммутативность}: $\langle \vec{a}, \vec{b} \rangle = \langle \vec{b}, \vec{a} \rangle$.
  \item \textbf{Линейность по каждому аргументу}: 
    \[
      \langle \alpha \vec{a} + \beta \vec{c}, \vec{b} \rangle = \alpha \langle \vec{a}, \vec{b} \rangle + \beta \langle \vec{c}, \vec{b} \rangle.
    \]
  \item \textbf{Положительная определённость}: $\langle \vec{a}, \vec{a} \rangle \ge 0$ и равно нулю только для $\vec{a}=\vec{0}$.
\end{itemize}

\textbf{Связь с длиной и углом между векторами}. Длина (норма) вектора:
\[
\|\vec{a}\| = \sqrt{\langle \vec{a}, \vec{a} \rangle}.
\]
Косинус угла $\theta$ между $\vec{a}$ и $\vec{b}$:
\[
\cos\theta = \frac{\langle \vec{a}, \vec{b} \rangle}{\|\vec{a}\|\;\|\vec{b}\|}.
\]

\textbf{Пример.} Для $\vec{a}=(1,2,2)$ и $\vec{b}=(2,0,1)$:
\[
\langle \vec{a}, \vec{b} \rangle = 1\cdot2 + 2\cdot0 + 2\cdot1 = 4,
\quad
\|\vec{a}\| = 3,\;
\|\vec{b}\| = \sqrt{5},
\quad
\cos\theta = \frac{4}{3\sqrt{5}}.
\]

\subsubsection{Источники}

\begin{itemize}
  \item Г.\,С. Михалев, \textit{Дискретная математика. Базовый курс для вузов}.
  \item Ш.\,Л. Ланг, \textit{Линейная алгебра}.
  \item \href{https://ru.wikipedia.org/wiki/Вектор}{Википедия: Вектор}.
\end{itemize}
