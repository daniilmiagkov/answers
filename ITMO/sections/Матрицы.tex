\section{Матрицы. Их свойства. Транспонированная матрица. Ранг матрицы}

\subsection{Матрицы и их свойства}

\textbf{Матрица} — это прямоугольная таблица чисел, записанная в виде:
\[
A = \begin{pmatrix}
a_{11} & a_{12} & \dots & a_{1n} \\
a_{21} & a_{22} & \dots & a_{2n} \\
\vdots & \vdots & \ddots & \vdots \\
a_{m1} & a_{m2} & \dots & a_{mn}
\end{pmatrix},
\]
где $a_{ij}$ — элемент на $i$-й строке и $j$-м столбце.

Если $m = n$, то матрица называется \textbf{квадратной}.

\textbf{Основные типы матриц:}
\begin{itemize}[leftmargin=*]
  \item \textit{Нулевая матрица}: все элементы равны нулю.
  \item \textit{Диагональная матрица}: все элементы вне главной диагонали равны нулю.
  \item \textit{Единичная матрица} $I_n$: на главной диагонали — единицы, остальное — нули.
\end{itemize}

\textbf{Операции над матрицами:}
\begin{itemize}[leftmargin=*]
  \item \textit{Сложение и вычитание} — поэлементно, если размеры совпадают.
  \item \textit{Умножение на число}: каждый элемент умножается на скаляр.
  \item \textit{Умножение матриц}: определяется как
  \[
  (AB)_{ij} = \sum_k a_{ik} b_{kj},
  \]
  если количество столбцов $A$ равно количеству строк $B$.
\end{itemize}

\subsection{Транспонированная матрица}

Транспонирование — это операция, при которой строки матрицы становятся столбцами. Обозначается $A^T$.

\[
\text{Если } A = \begin{pmatrix}1 & 2\\3 & 4\end{pmatrix},\quad
A^T = \begin{pmatrix}1 & 3\\2 & 4\end{pmatrix}.
\]

Свойства:
\begin{itemize}[leftmargin=*]
  \item $(A^T)^T = A$,
  \item $(A + B)^T = A^T + B^T$,
  \item $(AB)^T = B^T A^T$.
\end{itemize}

\subsection{Ранг матрицы}

\textbf{Ранг матрицы} — это наибольшее число линейно независимых строк (или столбцов) матрицы.

Обозначается: $\text{rank}(A)$.

Методы вычисления:
\begin{itemize}[leftmargin=*]
  \item Приведение к ступенчатому виду методом Гаусса.
  \item Определение максимального размера ненулевого минорa.
\end{itemize}

\textbf{Пример.}
\[
A = \begin{pmatrix}
1 & 2 & 3 \\
2 & 4 & 6 \\
3 & 6 & 9
\end{pmatrix}
\Rightarrow \text{rank}(A) = 1,
\]
так как все строки пропорциональны первой.

\textbf{Свойства ранга:}
\begin{itemize}[leftmargin=*]
  \item Ранг не меняется при элементарных преобразованиях строк.
  \item $\text{rank}(A) \le \min(m, n)$ для матрицы $A$ размера $m \times n$.
\end{itemize}

\subsubsection{Источники}

\begin{itemize}
  \item Ш.\,Л. Ланг, \textit{Линейная алгебра}.
  \item Г.\,С. Михалев, \textit{Дискретная математика}.
  \item \href{https://ru.wikipedia.org/wiki/Матрица_(математика)}{Википедия: Матрица}.
\end{itemize}
