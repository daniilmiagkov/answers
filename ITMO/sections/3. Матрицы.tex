\section{Сложение, умножение матрицы на число, умножение матриц, транспонирование матриц. Обратная матрица}

\subsection*{Сложение матриц}

Две матрицы $A$ и $B$ одинакового размера ($m \times n$) можно сложить, если у них совпадают размеры. Сложение происходит поэлементно:
\[
(A + B)_{ij} = A_{ij} + B_{ij}
\]

\textbf{Пример:}
\[
A =
\begin{pmatrix}
1 & 2 \\
3 & 4
\end{pmatrix}, \quad
B =
\begin{pmatrix}
5 & 6 \\
7 & 8
\end{pmatrix}
\Rightarrow
A + B =
\begin{pmatrix}
6 & 8 \\
10 & 12
\end{pmatrix}
\]

\textbf{Свойства сложения:}
\begin{itemize}
  \item Коммутативность: $A + B = B + A$
  \item Ассоциативность: $(A + B) + C = A + (B + C)$
  \item Существование нулевой матрицы $O$ (нулевая поэлементно): $A + O = A$
\end{itemize}

---

\subsection*{Умножение матрицы на число}

Если $\lambda \in \mathbb{R}$ — число (скаляр), то умножение $\lambda \cdot A$ означает умножение каждого элемента матрицы на это число:
\[
(\lambda A)_{ij} = \lambda \cdot A_{ij}
\]

\textbf{Пример:}
\[
A =
\begin{pmatrix}
2 & -1 \\
0 & 3
\end{pmatrix}, \quad \lambda = 4
\Rightarrow
4A =
\begin{pmatrix}
8 & -4 \\
0 & 12
\end{pmatrix}
\]

\textbf{Свойства:}
\begin{itemize}
  \item $\lambda (\mu A) = (\lambda \mu) A$
  \item $(\lambda + \mu) A = \lambda A + \mu A$
  \item $\lambda (A + B) = \lambda A + \lambda B$
\end{itemize}

---

\subsection*{Умножение матриц}

Матрицы $A$ и $B$ можно перемножить, если \textbf{число столбцов в $A$} равно \textbf{числу строк в $B$}.

Если $A$ — размера $m \times n$, а $B$ — $n \times k$, то произведение $C = AB$ — это матрица $m \times k$, где:
\[
C_{ij} = \sum_{r=1}^{n} A_{ir} \cdot B_{rj}
\]

То есть: элемент $C_{ij}$ получается как скалярное произведение $i$-й строки $A$ и $j$-го столбца $B$.

\textbf{Пример:}
\[
A =
\begin{pmatrix}
1 & 2 \\
3 & 4
\end{pmatrix}, \quad
B =
\begin{pmatrix}
0 & 1 \\
2 & 3
\end{pmatrix}
\]

\[
AB =
\begin{pmatrix}
1\cdot0 + 2\cdot2 & 1\cdot1 + 2\cdot3 \\
3\cdot0 + 4\cdot2 & 3\cdot1 + 4\cdot3
\end{pmatrix}
=
\begin{pmatrix}
4 & 7 \\
8 & 15
\end{pmatrix}
\]

\textbf{Важно:} $AB \ne BA$ в общем случае! Умножение матриц \textbf{не коммутативно}.

\textbf{Свойства:}
\begin{itemize}
  \item Ассоциативность: $A(BC) = (AB)C$
  \item Дистрибутивность: $A(B + C) = AB + AC$
  \item $(AB)^T = B^T A^T$ — транспонирование произведения
\end{itemize}

---

\subsection*{Транспонирование матрицы}

Транспонирование — это операция, при которой строки становятся столбцами, а столбцы — строками.

Для любой матрицы $A$, её транспонированная матрица $A^T$ определяется как:
\[
(A^T)_{ij} = A_{ji}
\]

\textbf{Пример:}
\[
A =
\begin{pmatrix}
1 & 2 & 3 \\
4 & 5 & 6
\end{pmatrix}
\Rightarrow
A^T =
\begin{pmatrix}
1 & 4 \\
2 & 5 \\
3 & 6
\end{pmatrix}
\]

\textbf{Свойства транспонирования:}
\begin{itemize}
  \item $(A^T)^T = A$
  \item $(A + B)^T = A^T + B^T$
  \item $(\lambda A)^T = \lambda A^T$
  \item $(AB)^T = B^T A^T$
\end{itemize}

Эта операция часто используется при симметризации, а также в определениях симметрических и ортогональных матриц.

---

\subsection*{Обратная матрица}

\textbf{Обратная матрица} $A^{-1}$ к квадратной матрице $A$ определяется как:
\[
A \cdot A^{-1} = A^{-1} \cdot A = I
\]
где $I$ — единичная матрица той же размерности.

\textbf{Условия существования:}
\begin{itemize}
  \item Матрица должна быть \textbf{квадратной}.
  \item Её \textbf{определитель не должен равняться нулю} ($\det A \ne 0$).
  \item Ранг $A$ должен равняться её размерности: $\operatorname{rank}(A) = n$
\end{itemize}

\textbf{Пример:}
\[
A =
\begin{pmatrix}
1 & 2 \\
3 & 4
\end{pmatrix}
\Rightarrow
A^{-1} = \frac{1}{\det A}
\begin{pmatrix}
4 & -2 \\
-3 & 1
\end{pmatrix}
=
\frac{1}{(1\cdot4 - 2\cdot3)}
\begin{pmatrix}
4 & -2 \\
-3 & 1
\end{pmatrix}
=
\begin{pmatrix}
-2 & 1 \\
1.5 & -0.5
\end{pmatrix}
\]

\subsection*{Способы нахождения обратной матрицы}

\begin{enumerate}
  \item Для $2\times2$-матриц:
  \[
  A = \begin{pmatrix}a & b\\ c & d\end{pmatrix}
  \Rightarrow
  A^{-1} = \frac{1}{ad - bc} \begin{pmatrix}d & -b \\ -c & a\end{pmatrix}
  \]

  \item Для больших матриц:
  \begin{itemize}
    \item Через присоединённую матрицу (алгебраические дополнения + транспонирование + деление на определитель)
    \item Метод Гаусса: расширение $A$ до $[A | I]$ и приведение к $[I | A^{-1}]$
  \end{itemize}
\end{enumerate}

---

\subsection*{Свойства обратной матрицы}

\begin{itemize}
  \item $(A^{-1})^{-1} = A$
  \item $(AB)^{-1} = B^{-1} A^{-1}$
  \item $(A^T)^{-1} = (A^{-1})^T$
\end{itemize}

\textbf{Важно:} не все матрицы имеют обратную. Такие матрицы называются \textbf{вырожденными}.

---

\subsection*{Применения обратной матрицы}

\begin{itemize}
  \item Решение систем уравнений: $A\vec{x} = \vec{b} \Rightarrow \vec{x} = A^{-1}\vec{b}$
  \item Вывод формул в статистике и машинном обучении
  \item Нормализация линейных преобразований
  \item Преобразование координат
\end{itemize}

---

\subsection*{Выводы}

\begin{itemize}
  \item Операции над матрицами (сложение, умножение, транспонирование) формируют алгебраическую структуру.
  \item Умножение матриц — основа линейных отображений и систем уравнений.
  \item Транспонирование — полезная симметризующая операция.
  \item Обратная матрица существует только у невырожденных квадратных матриц и даёт способ обращения линейных операторов.
\end{itemize}
