\section{Векторы}

\textbf{Вектор} — это математический объект, описывающий как направление, так и величину. Векторы часто изображаются как направленные отрезки (стрелки), начинающиеся в начале координат.

В алгебраической записи вектор в $n$-мерном пространстве $\mathbb{R}^n$ — это упорядоченный набор $n$ чисел:
\[
\vec{v} = (v_1, v_2, \dots, v_n)
\]

\subsection*{Примеры векторов}
\begin{itemize}
  \item В $\mathbb{R}^2$: $\vec{a} = (3, -1)$
  \item В $\mathbb{R}^3$: $\vec{b} = (0, 2, 1)$
  \item В $\mathbb{R}^4$: $\vec{c} = (1, 0, 0, -1)$
\end{itemize}

\vspace{1em}
\textbf{Геометрическая интерпретация:} вектор — это перемещение из одной точки в другую. Например, вектор $(2, 1)$ соответствует сдвигу на 2 единицы вправо и 1 вверх.

\vspace{1em}
\begin{center}
\begin{tikzpicture}[scale=1.2,>=stealth]
\draw[->,gray!30] (-1,0) -- (4,0) node[right] {$x$};
\draw[->,gray!30] (0,-1) -- (0,3) node[above] {$y$};
\draw[->,thick,blue] (0,0) -- (2,1) node[above right] {$\vec{v}$};
\end{tikzpicture}
\end{center}

\subsection*{Операции с векторами}

\begin{enumerate}
  \item \textbf{Сложение:}
  \[
  (x_1, y_1) + (x_2, y_2) = (x_1 + x_2, y_1 + y_2)
  \]
  \item \textbf{Умножение на число (скаляр):}
  \[
  \lambda \cdot (x, y) = (\lambda x, \lambda y)
  \]
  \item \textbf{Нулевой вектор:}
  \[
  \vec{0} = (0, 0, \dots, 0)
  \]
\end{enumerate}

\textbf{Графически сложение векторов} выглядит как «перенос конца первого вектора к началу второго» — получаем диагональ параллелограмма:

\vspace{1em}
\begin{center}
\begin{tikzpicture}[scale=0.9,>=stealth]
\draw[->,gray!40] (-1,0) -- (5,0) node[right] {$x$};
\draw[->,gray!40] (0,-1) -- (0,4) node[above] {$y$};

\draw[->,blue,thick] (0,0) -- (2,1.5) node[above right] {$\vec{a}$};
\draw[->,red,thick] (0,0) -- (1,2) node[above left] {$\vec{b}$};
\draw[->,green!70!black,thick] (0,0) -- (3,3.5) node[right] {$\vec{a} + \vec{b}$};

\draw[dashed] (2,1.5) -- (3,3.5);
\draw[dashed] (1,2) -- (3,3.5);
\end{tikzpicture}
\end{center}

---

\section{Линейная зависимость системы векторов}

Пусть заданы векторы $\vec{v}_1, \vec{v}_2, \dots, \vec{v}_k$ в пространстве $V$. Мы говорим, что они \textbf{линейно зависимы}, если существует набор чисел $\lambda_1, \dots, \lambda_k$, не все равные нулю, такой что:
\[
\lambda_1 \vec{v}_1 + \lambda_2 \vec{v}_2 + \dots + \lambda_k \vec{v}_k = \vec{0}
\]

Если же такое равенство возможно только при $\lambda_1 = \dots = \lambda_k = 0$, то векторы \textbf{линейно независимы}.

\subsection*{Интуитивно:}
Если один вектор можно выразить через другие — система зависима.

\textbf{Пример 1.}
\[
\vec{v}_1 = (1, 2), \quad \vec{v}_2 = (2, 4)
\]
Очевидно, что $\vec{v}_2 = 2 \vec{v}_1$, значит, они линейно зависимы.

\vspace{1em}
\begin{center}
\begin{tikzpicture}[scale=1.2,>=stealth]
\draw[->,gray!40] (-0.5,0) -- (3,0);
\draw[->,gray!40] (0,-0.5) -- (0,3);

\draw[->,blue,thick] (0,0) -- (1,2) node[above left] {$\vec{v}_1$};
\draw[->,red,thick] (0,0) -- (2,4) node[above right] {$\vec{v}_2$};
\end{tikzpicture}
\end{center}

\textbf{Пример 2.}
Векторы $\vec{u}_1 = (1, 0)$ и $\vec{u}_2 = (0, 1)$ линейно независимы, так как невозможно выразить один через другой. Они формируют базис в $\mathbb{R}^2$.

---

\section{Базис линейного пространства}

Базис — это система векторов, которая:
\begin{enumerate}
  \item линейно независима;
  \item порождает всё пространство (любой вектор можно выразить через неё).
\end{enumerate}

Если базис состоит из $n$ векторов, говорят, что размерность пространства равна $n$.

\textbf{Пример.} В $\mathbb{R}^2$ стандартный базис:
\[
\vec{e}_1 = (1, 0), \quad \vec{e}_2 = (0, 1)
\]
Тогда любой вектор $\vec{v} = (x, y)$ записывается как:
\[
\vec{v} = x \vec{e}_1 + y \vec{e}_2
\]

\vspace{1em}
\begin{center}
\begin{tikzpicture}[scale=1.1,>=stealth]
\draw[->,gray!40] (-0.5,0) -- (3.5,0) node[right] {$x$};
\draw[->,gray!40] (0,-0.5) -- (0,3.5) node[above] {$y$};

\draw[->,thick] (0,0) -- (2,0) node[below] {$x\vec{e}_1$};
\draw[->,thick] (2,0) -- (2,1.5) node[right] {$y\vec{e}_2$};
\draw[->,blue,thick] (0,0) -- (2,1.5) node[above right] {$\vec{v}$};
\end{tikzpicture}
\end{center}

\subsection*{Важно:}
Базис может быть не единственным. Например, вектора $\vec{e}_1 = (1, 1)$ и $\vec{e}_2 = (1, -1)$ тоже образуют базис в $\mathbb{R}^2$.

---

\section{Скалярное произведение векторов}

Скалярное произведение двух векторов $\vec{a} = (a_1, \dots, a_n)$ и $\vec{b} = (b_1, \dots, b_n)$:
\[
\langle \vec{a}, \vec{b} \rangle = a_1b_1 + a_2b_2 + \dots + a_n b_n
\]

\textbf{Пример:}
\[
\vec{a} = (1, 2), \quad \vec{b} = (3, 4) \Rightarrow \langle \vec{a}, \vec{b} \rangle = 1\cdot3 + 2\cdot4 = 11
\]

\textbf{Свойства:}
\begin{itemize}
  \item Коммутативность: $\langle \vec{a}, \vec{b} \rangle = \langle \vec{b}, \vec{a} \rangle$
  \item Линейность по каждому аргументу
  \item $\langle \vec{a}, \vec{a} \rangle = \|\vec{a}\|^2$
\end{itemize}

\textbf{Геометрическая формула:}
\[
\langle \vec{a}, \vec{b} \rangle = \|\vec{a}\|\|\vec{b}\|\cos\theta
\]

\textbf{Если } $\langle \vec{a}, \vec{b} \rangle = 0$ $\Rightarrow$ векторы перпендикулярны (ортогональны).

\vspace{1em}
\begin{center}
\begin{tikzpicture}[scale=1.2,>=stealth]
\coordinate (O) at (0,0);
\coordinate (A) at (2,0);
\coordinate (B) at (1,2);
\draw[->,gray!40] (-0.5,0) -- (3,0);
\draw[->,gray!40] (0,-0.5) -- (0,3);
\draw[->,thick] (O) -- (A) node[below] {$\vec{a}$};
\draw[->,thick] (O) -- (B) node[above] {$\vec{b}$};
\draw (1,0) arc[start angle=0,end angle=63,radius=1cm];
\node at (1.1,0.2) {$\theta$};
\end{tikzpicture}
\end{center}

\subsection*{Приложение: длина и угол}
Длина вектора $\vec{a}$ (её называют \textbf{нормой}) выражается так:
\[
\|\vec{a}\| = \sqrt{\langle \vec{a}, \vec{a} \rangle}
\]

А угол между двумя векторами вычисляется по формуле:
\[
\cos\theta = \frac{\langle \vec{a}, \vec{b} \rangle}{\|\vec{a}\|\|\vec{b}\|}
\]

---

\subsection*{Выводы}
\begin{itemize}
  \item Векторы — базовые элементы линейной алгебры, описывающие направление и величину.
  \item Линейная зависимость позволяет понять, насколько векторы "разные" и важны.
  \item Базис даёт возможность представить любое состояние системы как комбинацию базовых движений.
  \item Скалярное произведение связывает векторы с геометрией: длиной и углом.
\end{itemize}
