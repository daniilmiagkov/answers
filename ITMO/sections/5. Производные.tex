\section{Производные. Необходимое и достаточное условия дифференцируемости функции. Частные и полные производные}

\subsection*{Понятие производной функции одной переменной}

Пусть $f(x)$ определена в окрестности точки $x_0$. Производная функции $f$ в точке $x_0$ — это предел отношения приращения функции к приращению аргумента:
\[
f'(x_0) = \lim_{\Delta x \to 0} \frac{f(x_0 + \Delta x) - f(x_0)}{\Delta x}
\]

Если предел существует, то говорят, что функция \textbf{дифференцируема} в точке $x_0$.

\textbf{Геометрический смысл:} производная — это угловой коэффициент касательной к графику функции в данной точке.

\subsection*{Дифференцируемость и непрерывность}

\begin{itemize}
  \item Если функция дифференцируема в точке $x_0$, то она непрерывна в этой точке.
  \item Обратное неверно: непрерывность не гарантирует существование производной.
\end{itemize}

\textbf{Пример (не дифференцируема):}
\[
f(x) = |x| \Rightarrow f'(0) \text{ не существует, хотя } f \text{ непрерывна в } 0
\]

---

\subsection*{Производная по направлению и частные производные}

Пусть $f(x, y)$ — функция двух переменных.

Производную по направлению можно определить через вектор направления $\vec{l} = (l_1, l_2)$:
\[
D_{\vec{l}} f(x_0, y_0) = \lim_{h \to 0} \frac{f(x_0 + h l_1, y_0 + h l_2) - f(x_0, y_0)}{h}
\]

Частные производные — это производные по отдельным переменным:
\[
\frac{\partial f}{\partial x}(x_0, y_0) = \lim_{h \to 0} \frac{f(x_0 + h, y_0) - f(x_0, y_0)}{h}
\]
\[
\frac{\partial f}{\partial y}(x_0, y_0) = \lim_{h \to 0} \frac{f(x_0, y_0 + h) - f(x_0, y_0)}{h}
\]

\textbf{Обозначения:}
\[
f_x(x, y), \quad f_y(x, y), \quad \text{или } \partial_x f, \ \partial_y f
\]

\textbf{Пример:} пусть $f(x, y) = x^2 y + \sin(xy)$

\[
\frac{\partial f}{\partial x} = 2x y + y \cos(xy), \quad
\frac{\partial f}{\partial y} = x^2 + x \cos(xy)
\]

---

\subsection*{Необходимое и достаточное условия дифференцируемости функции многих переменных}

Функция $f(x, y)$ называется \textbf{дифференцируемой в точке} $(x_0, y_0)$, если приращение можно представить в виде:
\[
\Delta f = f(x_0 + \Delta x, y_0 + \Delta y) - f(x_0, y_0) = A \Delta x + B \Delta y + o(\sqrt{\Delta x^2 + \Delta y^2})
\]

где $A$ и $B$ — постоянные (зависят от точки), а $o(\rho)$ — бесконечно малая по сравнению с $\rho$.

\textbf{Формально:} $f$ дифференцируема в $(x_0, y_0)$, если существует линейное отображение $L$, приближающее $\Delta f$.

\subsubsection*{Необходимое условие дифференцируемости}

Если $f$ дифференцируема в $(x_0, y_0)$, то:
\begin{itemize}
  \item Частные производные $\frac{\partial f}{\partial x}, \frac{\partial f}{\partial y}$ существуют
  \item $\Delta f = \frac{\partial f}{\partial x} \Delta x + \frac{\partial f}{\partial y} \Delta y + o(\sqrt{\Delta x^2 + \Delta y^2})$
\end{itemize}

\subsubsection*{Достаточное условие дифференцируемости}

Если:
\begin{itemize}
  \item Частные производные существуют в окрестности точки
  \item И они непрерывны в точке $(x_0, y_0)$
\end{itemize}
то $f$ дифференцируема в этой точке.

---

\subsection*{Градиент и направление наибольшего роста}

Градиент — это вектор, составленный из всех частных производных:
\[
\nabla f(x, y) = \left( \frac{\partial f}{\partial x}, \frac{\partial f}{\partial y} \right)
\]

Производная функции по направлению вектора $\vec{l}$ выражается как скалярное произведение:
\[
D_{\vec{l}} f = \nabla f \cdot \vec{l}
\]

\textbf{Свойства:}
\begin{itemize}
  \item Направление градиента — направление наибольшего возрастания функции.
  \item Если $\nabla f = \vec{0}$, то это критическая точка (возможно максимум, минимум или седло).
\end{itemize}

---

\subsection*{Полный дифференциал}

Если функция $f(x, y)$ дифференцируема, то её полное приращение можно выразить через полный дифференциал:
\[
df = \frac{\partial f}{\partial x} dx + \frac{\partial f}{\partial y} dy
\]

\textbf{Пример:} $f(x, y) = x^2 y + \sin(xy)$

\[
df = (2x y + y \cos(xy)) dx + (x^2 + x \cos(xy)) dy
\]

\textbf{Полный дифференциал} используется:
\begin{itemize}
  \item В оценке приращений функции
  \item В дифференциальной геометрии и анализе ошибок
  \item При переходе к новым координатам
\end{itemize}

---

\subsection*{Итоги}

\begin{itemize}
  \item Производная — это мера изменения функции.
  \item Частные производные — изменения по осям координат.
  \item Дифференцируемость функции двух переменных требует не только существования производных, но и их «согласованного» поведения.
  \item Градиент показывает направление роста функции.
  \item Полный дифференциал — обобщение производной на многомерные функции.
\end{itemize}
