\section{Понятие случайной величины. Функция плотности распределения и её свойства. Дискретные и непрерывные законы распределения, их свойства}

\subsection*{1. Интуиция и формальное определение случайной величины}

Интуитивно \textbf{случайная величина} — это правило, которое каждому исходу случайного эксперимента сопоставляет некоторое числовое значение. Например, при броске монеты можно сопоставить ``орёл'' → 1, ``решка'' → 0; при броске кости — само выпавшее число.

Формально: пусть $(\Omega,\mathcal{F},P)$ — вероятностное пространство (пространство элементарных исходов $\Omega$, $\sigma$-алгебра событий $\mathcal{F}$ и вероятность $P$). Тогда случайная величина — это измеримая функция
\[
X:\Omega\to\mathbb{R},
\]
то есть для любого промежутка (или борелевского множества) $B\subseteq\mathbb{R}$ множество $\{\omega\in\Omega:\ X(\omega)\in B\}$ должно быть событием (лежать в $\mathcal{F}$).

\medskip
Классификация по типу значений:
\begin{itemize}
  \item \textbf{Дискретная} случайная величина принимает не более счётного множества значений.
  \item \textbf{Непрерывная} (абсолютно непрерывная) имеет плотность распределения относительно меры Лебега (нет атомов точной массы).
  \item \textbf{Смешанная} — содержит как дискретную, так и непрерывную составляющие.
\end{itemize}

\subsection*{2. Функция распределения (CDF) — основа описания закона}

Для произвольной случайной величины $X$ её \textbf{функция распределения} (Cumulative Distribution Function, CDF) определяется как
\[
F_X(x) := P(X \le x),\quad x\in\mathbb{R}.
\]
Функция распределения полностью задаёт закон случайной величины (включая дискретные и непрерывные части).

\medskip
\textbf{Ключевые свойства $F_X(x)$:}
\begin{enumerate}
  \item $F_X$ монотонно неубывает: если $x_1\le x_2$ то $F_X(x_1)\le F_X(x_2)$.
  \item Правосторонняя непрерывность: $\displaystyle \lim_{t\downarrow x} F_X(t)=F_X(x)$.
  \item Пределы на бесконечностях: $\displaystyle \lim_{x\to -\infty} F_X(x)=0,\quad \lim_{x\to +\infty} F_X(x)=1$.
  \item Для любых $a<b$ выполнено $P(a<X\le b)=F_X(b)-F_X(a)$.
\end{enumerate}

\medskip
\textbf{Атомы (точечные массы).} Если в точке $x_0$ сразу возникает положительный скачок $p_0=F_X(x_0)-\lim_{x\to x_0^-}F_X(x)$, то $P(X=x_0)=p_0>0$ — это дискретная (атомная) часть распределения.

\subsection*{3. Дискретные законы распределения}

\paragraph{Определение.} Случайная величина $X$ называется дискретной, если существует (счётное) набор значений $\{x_k\}$ такой, что $P(X\in\{x_k\})=1$. Тогда её закон задаётся \textbf{функцией вероятности} (PMF):
\[
p_X(x_k) := P(X=x_k),\qquad \sum_k p_X(x_k)=1,\quad p_X(x_k)\ge0.
\]

\paragraph{Ожидание и дисперсия.} Если суммы сходятся абсолютно, определяются математическое ожидание и дисперсия:
\[
\mathbb{E}[X] = \sum_k x_k p_X(x_k),\qquad
\operatorname{Var}(X)=\mathbb{E}[(X-\mathbb{E}X)^2]=\sum_k (x_k-\mathbb{E}X)^2 p_X(x_k).
\]

\paragraph{Типичные примеры (с формулами и краткой интерпретацией):}
\begin{itemize}
  \item \textbf{Бернуллиевская (Bernoulli)}: $X\in\{0,1\}$, $P(X=1)=p$, $P(X=0)=1-p$. \\
    $\mathbb{E}X=p$, $\operatorname{Var}(X)=p(1-p)$.
  \item \textbf{Биномиальная (Binomial)} $X\sim Bin(n,p)$: $P(X=k)=\binom{n}{k}p^k(1-p)^{n-k}$ для $k=0,\dots,n$. \\
    $\mathbb{E}X=np$, $\operatorname{Var}(X)=np(1-p)$. Модель: $n$ независимых испытаний Бернулли.
  \item \textbf{Геометрическая (Geometric)}: $P(X=k)=(1-p)^{k-1}p$ (номер первого успеха). $\mathbb{E}X=\frac{1}{p}$.
  \item \textbf{Пуассоновская (Poisson)} $X\sim Pois(\lambda)$: $P(X=k)=e^{-\lambda}\frac{\lambda^k}{k!}$, $k\ge0$. \\
    Это предел биномиального при $n\to\infty$, $p\to0$, $np\to\lambda$. $\mathbb{E}X=\lambda$, $\operatorname{Var}(X)=\lambda$.
\end{itemize}

\paragraph{Генерирующие функции (полезный инструмент).}
\begin{itemize}
  \item \textbf{Моментная функция (MGF):} $M_X(t)=\mathbb{E}[e^{tX}] = \sum_k e^{t x_k} p_X(x_k)$.
  \item \textbf{Функция порождающая (PGF) для неотрицательных целых:} $G_X(s)=\mathbb{E}[s^X] = \sum_{k\ge0} s^k p_X(k)$.
\end{itemize}
MGF и PGF удобны для вычисления моментов и сумм независимых случайных величин.

\subsection*{4. Непрерывные законы распределения и функция плотности (PDF)}

\paragraph{Определение.} В случае, когда $X$ непрерывна, её закон, как правило, задаётся \textbf{функцией плотности} $f_X(x)$ (PDF), такой что для любых множеств
\[
P(a<X\le b) = \int_a^b f_X(x)\,dx.
\]
При этом $f_X(x)\ge0$ почти везде и
\[
\int_{-\infty}^{\infty} f_X(x)\,dx = 1.
\]

\paragraph{Связь CDF и PDF.} Если $F_X$ дифференцируема в точке $x$, то $f_X(x)=F_X'(x)$. В общем случае $F_X$ может содержать дискретные скачки и непрерывную часть; тогда $F_X$ раскладывается в сумму атомов и интегральной части.

\paragraph{Ожидание и дисперсия (непрерывный случай):}
\[
\mathbb{E}[X] = \int_{-\infty}^{\infty} x f_X(x)\,dx,\qquad
\operatorname{Var}(X) = \int (x-\mathbb{E}X)^2 f_X(x)\,dx.
\]

\paragraph{Типичные непрерывные распределения:}
\begin{itemize}
  \item \textbf{Равномерное} $U(a,b)$: $f(x)=\dfrac{1}{b-a}$ при $a\le x\le b$, иначе $0$. $\mathbb{E}X=\dfrac{a+b}{2}$, $\operatorname{Var}(X)=\dfrac{(b-a)^2}{12}$.
  \item \textbf{Экспоненциальное} $Exp(\lambda)$: $f(x)=\lambda e^{-\lambda x}$ при $x\ge0$. Памятьlessness: $P(X>t+s\mid X>t)=P(X>s)$. $\mathbb{E}X=1/\lambda$, $\operatorname{Var}(X)=1/\lambda^2$.
  \item \textbf{Нормальное} $N(\mu,\sigma^2)$: $f(x)=\dfrac{1}{\sqrt{2\pi}\sigma}\exp\bigl(-\dfrac{(x-\mu)^2}{2\sigma^2}\bigr)$ для $x\in\mathbb{R}$. $\mathbb{E}X=\mu$, $\operatorname{Var}(X)=\sigma^2$. Центральная предельная теорема делает нормальное распределение фундаментальным.
  \item \textbf{Гамма, Бета, Коши} и др. — семейства с разными формами плотностей, полезные в разных задачах.
\end{itemize}

\paragraph{Пример вычислений (равномерное и экспоненциальное).}
\begin{itemize}
  \item $X\sim U(0,1)$: $\mathbb{E}[X]=\int_0^1 x\,dx=\tfrac12$, $\operatorname{Var}(X)=\int_0^1 (x-\tfrac12)^2 dx=\tfrac{1}{12}$.
  \item $X\sim Exp(\lambda)$: $\mathbb{E}[X]=\int_0^\infty x\lambda e^{-\lambda x}dx = \lambda\cdot \frac{1}{\lambda^2} = \frac{1}{\lambda}$ (полезно интегрировать по частям).
\end{itemize}

\subsection*{5. Смешанные распределения}

Реально встречаются законы, у которых есть и дискретная часть (атомы), и плотность. Тогда CDF распадается:
\[
F_X(x) = \sum_{x_k\le x} p_k + \int_{-\infty}^x f_{\text{cont}}(t)\,dt,
\]
где $p_k=P(X=x_k)$ — массы в точках, а $f_\text{cont}$ — плотность непрерывной части. Часто такие случаи возникают, например, при моделировании с выпадением особого события (атом) плюс «обычный» непрерывный шум.

\subsection*{6. Характеристики распределения: мода, медиана, квантили, моментные характеристики}

\begin{itemize}
  \item \textbf{Мода} — значение $x$ (необязательно единственное), в котором плотность (или PMF) достигает максимума.
  \item \textbf{Медиана} $m$ — решение $F_X(m)\ge 1/2$ и $F_X(m-)\le 1/2$; для непрерывных распределений часто единственна.
  \item \textbf{Квантили:} $q_\alpha = \inf\{x: F_X(x)\ge \alpha\}$.
  \item \textbf{Моменты:} $\mathbb{E}[X^k]$ при существовании; центральные моменты $\mathbb{E}[(X-\mathbb{E}X)^k]$; моментная функция $M_X(t)=\mathbb{E}[e^{tX}]$ (если существует в окрестности $0$).
\end{itemize}

Эти характеристики используются для описания асимметрии (скос), крутизны (эксцесс) и т.д.

\subsection*{7. Преобразования случайных величин}

\paragraph{Дискретный случай.} Если $X$ дискретна с $P(X=x_k)=p_k$, а $Y=g(X)$, то
\[
P(Y=y) = \sum_{k:\ g(x_k)=y} p_k.
\]

\paragraph{Непрерывный случай (монотонная функция).} Пусть $X$ имеет плотность $f_X$ и $Y=g(X)$, где $g$ монотонна и дифференцируема. Тогда плотность $f_Y$ для $y=g(x)$ даётся формулой
\[
f_Y(y) = f_X\bigl(g^{-1}(y)\bigr)\, \left| \frac{d}{dy} g^{-1}(y)\right|.
\]
Для многомерного случая используется якобиан преобразования.

\paragraph{Пример (известный):} Если $X\sim U(0,1)$ и $Y=-\ln X$, то $Y\sim Exp(1)$. Действительно, $P(Y\le y)=P(-\ln X\le y)=P(X\ge e^{-y})=1-e^{-y}$ для $y\ge0$, дифференцируя получаем плотность $\;f_Y(y)=e^{-y}\,$.

\subsection*{8. Совместные распределения, маргинальные и условные законы (кратко)}

Хотя основной вопрос — одномерные законы, важно упомянуть:
\begin{itemize}
  \item Для вектора случайных величин $(X,Y)$ задаётся \textbf{совместная} PMF или PDF $p_{X,Y}(x,y)$ или $f_{X,Y}(x,y)$.
  \item \textbf{Маргинальная} плотность: $f_X(x)=\int f_{X,Y}(x,y)\,dy$ (аналогично для дискретного: суммирование).
  \item \textbf{Условная} плотность: $f_{Y\mid X}(y\mid x)=\dfrac{f_{X,Y}(x,y)}{f_X(x)}$ при $f_X(x)>0$.
  \item Независимость: $X$ и $Y$ независимы ⇔ $f_{X,Y}(x,y)=f_X(x)f_Y(y)$ (или для дискретного — $p_{X,Y}(x,y)=p_X(x)p_Y(y)$).
\end{itemize}

\subsection*{9. Заключение — практические советы и «чек-лист» для экзамена}

\begin{itemize}
  \item \textbf{Запомнить определения:} CDF $F_X$, PMF $p_X$ для дискретных, PDF $f_X$ для непрерывных.
  \item \textbf{Свойства:} $F$ монотонен, правосторонне непрерывен, пределы 0 и 1; $p\ge0,\ \sum p=1$; $f\ge0,\ \int f=1$.
  \item \textbf{Переходы:} $F'(x)=f(x)$ (когда существует), $P(a<X\le b)=F(b)-F(a)$.
  \item \textbf{Частые формулы:} $\mathbb{E}[X]=\sum x_k p_k$ (дискретно) или $\int x f(x)\,dx$ (непрерывно).
  \item \textbf{Преобразования:} дискретный — суммирование по прообразам; непрерывный — замена переменной с модулем якобиана.
  \item \textbf{Типовые распределения:} знать формулы PMF/PDF, ожидание и дисперсию для Bernoulli, Binomial, Poisson, Geometric, Uniform, Exponential, Normal.
\end{itemize}

\subsection*{10. Задачи для закрепления (с краткими подсказками)}

\begin{enumerate}
  \item Докажите, что функция распределения $F_X$ права-непрерывна и монотонна. (\emph{Подсказка: используйте свойства вероятности и представление $F(x)=P(X\le x)$}.)
  \item Пусть $X\sim Bin(n,p)$. Найдите MGF $M_X(t)$ и используйте её, чтобы получить $\mathbb{E}X$ и $\operatorname{Var}(X)$. (\emph{Подсказка: $M_X(t)=(1-p+pe^t)^n$}.)
  \item Для $X\sim N(0,1)$ доказать, что плотность интегрируется в $1$ (можно сослаться на табличный интеграл или заменить в полярных координатах). (\emph{Подсказка: вычислите $I^2=\left(\int e^{-x^2/2}dx\right)^2$ через двойной интеграл}.)
  \item Пусть $X$ имеет CDF $F$. Покажите, что $P(X=a)=F(a)-\lim_{x\to a^-}F(x)$. (\emph{Подсказка: используйте определение вероятности точки как разности $P(X\le a)-P(X<a)$}.)
\end{enumerate}

\bigskip
Если хочешь, я могу:
\begin{itemize}
  \item развернуть этот материал ещё глубже (формулы для моментов более высоких порядков, характеристические функции, теорема Лебега о разложении мер в дискретную и абсолютно непрерывную части);
  \item подготовить набор типовых экзаменационных задач с полными решениями;
  \item сгенерировать иллюстрации в TikZ для PMF (столбиковые диаграммы) и PDF (кривые) для каждого приведённого распределения.
\end{itemize}
Скажи, что делаем дальше — рисуем графики или сразу задачи с подробными решениями?
