\section{Частные производные. Градиент функции. Производная по направлению}

\subsection*{Частные производные функции двух переменных}

Пусть $f(x, y)$ — функция двух переменных, определённая в окрестности точки $(x_0, y_0)$. Тогда:

\[
\frac{\partial f}{\partial x}(x_0, y_0) = \lim_{h \to 0} \frac{f(x_0 + h, y_0) - f(x_0, y_0)}{h}
\]
\[
\frac{\partial f}{\partial y}(x_0, y_0) = \lim_{h \to 0} \frac{f(x_0, y_0 + h) - f(x_0, y_0)}{h}
\]

Частные производные — это производные функции по одной переменной при фиксированной другой.

\textbf{Геометрический смысл:} производная по $x$ — это скорость изменения функции вдоль оси $x$, при фиксированном $y$.

\textbf{Обозначения:}
\[
f_x = \frac{\partial f}{\partial x}, \quad f_y = \frac{\partial f}{\partial y}
\]

\textbf{Пример:}
Пусть $f(x, y) = x^2 y + \sin(xy)$. Тогда:
\[
\frac{\partial f}{\partial x} = 2xy + y \cos(xy)
\]
\[
\frac{\partial f}{\partial y} = x^2 + x \cos(xy)
\]

---

\subsection*{Частные производные высших порядков}

Можно вычислять производные второго порядка и выше:
\[
\frac{\partial^2 f}{\partial x^2}, \quad
\frac{\partial^2 f}{\partial y^2}, \quad
\frac{\partial^2 f}{\partial x \partial y}, \quad
\frac{\partial^2 f}{\partial y \partial x}
\]

Если $f$ дважды непрерывно дифференцируема, то:
\[
\frac{\partial^2 f}{\partial x \partial y} = \frac{\partial^2 f}{\partial y \partial x}
\]
(теорема Шварца о равенстве смешанных производных)

---

\subsection*{Градиент функции}

Пусть $f(x, y)$ — дифференцируемая функция. Тогда \textbf{градиент} функции $f$ в точке $(x_0, y_0)$ — это вектор:
\[
\nabla f(x_0, y_0) = \left(
\frac{\partial f}{\partial x}(x_0, y_0),
\frac{\partial f}{\partial y}(x_0, y_0)
\right)
\]

Если $f$ зависит от $n$ переменных, то градиент — вектор из $n$ компонент:
\[
\nabla f = \left(
\frac{\partial f}{\partial x_1}, \dots, \frac{\partial f}{\partial x_n}
\right)
\]

\textbf{Геометрический смысл:}
\begin{itemize}
  \item Направление градиента — это направление \textit{наибольшего роста функции}.
  \item Его длина — скорость наибольшего изменения.
  \item Если $\nabla f = \vec{0}$, то точка является критической (возможный экстремум).
\end{itemize}

\textbf{Пример:}
Пусть $f(x, y) = x^2 + y^2 \Rightarrow \nabla f = (2x, 2y)$ — вектор, указывающий от начала координат.

---

\subsection*{Производная функции по направлению}

Пусть функция $f(x, y)$ определена в точке $(x_0, y_0)$, и задан единичный вектор направления:
\[
\vec{l} = (\cos \alpha, \cos \beta), \quad \|\vec{l}\| = 1
\]

\textbf{Производная функции $f$ по направлению} вектора $\vec{l}$ в точке $(x_0, y_0)$:
\[
D_{\vec{l}} f(x_0, y_0) = \lim_{h \to 0} \frac{f(x_0 + h \cos \alpha, y_0 + h \cos \beta) - f(x_0, y_0)}{h}
\]

\textbf{Свойство:}
Если функция $f$ дифференцируема, то производная по направлению вычисляется через градиент:
\[
D_{\vec{l}} f = \nabla f \cdot \vec{l} = \left(
\frac{\partial f}{\partial x}, \frac{\partial f}{\partial y}
\right) \cdot (\cos \alpha, \cos \beta)
\]

Это \textbf{скалярное произведение} векторов: градиента и направления.

\textbf{Следствия:}
\begin{itemize}
  \item Наибольшая производная по направлению достигается в направлении градиента.
  \item Если $\vec{l} \perp \nabla f$, то производная по направлению равна нулю (функция не меняется вдоль этого направления).
\end{itemize}

\subsubsection*{Пример:}

Пусть $f(x, y) = x^2 y + y$, точка $M = (1, 2)$, направление $\vec{l} = \frac{1}{\sqrt{2}}(1, 1)$.

\[
\nabla f = \left(2x y, x^2 + 1\right) \Rightarrow
\nabla f(1,2) = (4, 2)
\]

\[
D_{\vec{l}} f = \nabla f \cdot \vec{l} = \frac{1}{\sqrt{2}}(4 + 2) = \frac{6}{\sqrt{2}} = 3\sqrt{2}
\]

---

\subsection*{Итоги}

\begin{itemize}
  \item Частные производные показывают, как функция меняется по каждой координате.
  \item Градиент — главный вектор изменения, указывает направление наибольшего роста.
  \item Производная по направлению обобщает понятие производной на произвольное направление.
  \item Всё вместе используется в оптимизации, градиентном спуске, анализе поверхности.
\end{itemize}
