\section{Матрицы. Их свойства. Транспонированная матрица. Ранг матрицы}

\textbf{Матрица} — это прямоугольная таблица чисел, организованная в строки и столбцы. Она записывается в виде:
\[
A =
\begin{pmatrix}
a_{11} & a_{12} & \dots & a_{1n} \\
a_{21} & a_{22} & \dots & a_{2n} \\
\vdots & \vdots & \ddots & \vdots \\
a_{m1} & a_{m2} & \dots & a_{mn}
\end{pmatrix}
\]
где $a_{ij}$ — элемент матрицы на $i$-й строке и $j$-м столбце.

\subsection*{Обозначения и размерность}

Матрицу обозначают заглавной латинской буквой ($A$, $B$, $C$ и т.д.). Размерность матрицы — это количество строк и столбцов. Если в матрице $m$ строк и $n$ столбцов, её размер обозначают как $m \times n$.

\textbf{Примеры:}
\[
A =
\begin{pmatrix}
1 & 2 \\
3 & 4
\end{pmatrix}, \quad
B =
\begin{pmatrix}
1 & 0 & -1 \\
2 & 3 & 5
\end{pmatrix}
\]
Здесь $A$ — квадратная матрица $2\times2$, $B$ — прямоугольная матрица $2\times3$.

---

\subsection*{Основные типы матриц}

\begin{itemize}
  \item \textbf{Нулевая матрица:} все элементы равны нулю.
  \item \textbf{Единичная матрица $I_n$:} квадратная матрица с единицами на главной диагонали и нулями вне её.
  \[
  I_3 =
  \begin{pmatrix}
  1 & 0 & 0 \\
  0 & 1 & 0 \\
  0 & 0 & 1
  \end{pmatrix}
  \]
  \item \textbf{Диагональная матрица:} все элементы вне главной диагонали равны нулю.
  \item \textbf{Квадратная матрица:} одинаковое число строк и столбцов.
  \item \textbf{Столбец (вектор-столбец):} матрица размером $m \times 1$.
  \item \textbf{Строка (вектор-строка):} матрица размером $1 \times n$.
\end{itemize}

---

\subsection*{Операции с матрицами}

\begin{enumerate}
  \item \textbf{Сложение:} складываются поэлементно. Возможно только для матриц одинакового размера.
  \[
  A + B = \left(a_{ij} + b_{ij}\right)
  \]

  \item \textbf{Умножение на скаляр:}
  \[
  \lambda A = \left(\lambda \cdot a_{ij}\right)
  \]

  \item \textbf{Умножение матриц:} если $A$ — матрица размера $m \times n$, а $B$ — $n \times k$, то их произведение $C = AB$ будет размером $m \times k$:
  \[
  c_{ij} = \sum_{r=1}^{n} a_{ir} \cdot b_{rj}
  \]

  \item \textbf{Транспонирование (см. ниже)} — замена строк и столбцов.
\end{enumerate}

---

\subsection*{Свойства операций}

\begin{itemize}
  \item Коммутативность сложения: $A + B = B + A$
  \item Ассоциативность: $(A + B) + C = A + (B + C)$
  \item Дистрибутивность: $\lambda(A + B) = \lambda A + \lambda B$
  \item $(AB)^T = B^T A^T$ — важное свойство транспонирования
\end{itemize}

---

\subsection*{Транспонированная матрица}

Матрица $A^T$ (читается: «A транспонированная») получается из $A$ заменой строк на столбцы. Формально:
\[
(A^T)_{ij} = A_{ji}
\]

\textbf{Пример:}
\[
A =
\begin{pmatrix}
1 & 2 \\
3 & 4 \\
5 & 6
\end{pmatrix} \quad \Rightarrow \quad
A^T =
\begin{pmatrix}
1 & 3 & 5 \\
2 & 4 & 6
\end{pmatrix}
\]

\textbf{Свойства транспонирования:}
\begin{itemize}
  \item $(A^T)^T = A$
  \item $(A + B)^T = A^T + B^T$
  \item $(\lambda A)^T = \lambda A^T$
  \item $(AB)^T = B^T A^T$
\end{itemize}

---

\subsection*{Ранг матрицы}

\textbf{Ранг матрицы} — это максимальное число линейно независимых строк (или столбцов) в матрице.

Обозначается: $\operatorname{rank}(A)$.

\textbf{Интуитивно:} ранг показывает, сколько "уникальной" информации содержится в строках или столбцах.

\textbf{Пример:}
\[
A =
\begin{pmatrix}
1 & 2 \\
2 & 4
\end{pmatrix}
\Rightarrow \text{строки линейно зависимы} \Rightarrow \operatorname{rank}(A) = 1
\]

\textbf{Другой пример:}
\[
B =
\begin{pmatrix}
1 & 2 & 3 \\
0 & 1 & 4 \\
0 & 0 & 1
\end{pmatrix}
\Rightarrow \operatorname{rank}(B) = 3
\]

\subsection*{Как находить ранг?}

Обычно с помощью преобразования матрицы к \textbf{ступенчатому виду} методом Гаусса. Количество ненулевых строк после преобразования и будет рангом.

\subsubsection*{Пример пошагово:}

Дана матрица:
\[
A =
\begin{pmatrix}
1 & 2 & 1 \\
2 & 4 & 2 \\
3 & 6 & 3
\end{pmatrix}
\]

Видим: вторая и третья строки — кратные первой. После приведения:
\[
\begin{pmatrix}
1 & 2 & 1 \\
0 & 0 & 0 \\
0 & 0 & 0
\end{pmatrix}
\Rightarrow \operatorname{rank}(A) = 1
\]

---

\subsection*{Геометрическая интерпретация ранга}

Векторы-строки (или столбцы) матрицы можно представить как векторы в пространстве. Ранг говорит о том, какое пространство они натягивают:
\begin{itemize}
  \item Ранг 1: все лежат на одной прямой
  \item Ранг 2: в одной плоскости
  \item Ранг 3: в трёхмерном пространстве и т.д.
\end{itemize}

---

\subsection*{Важность ранга}

Ранг используется в:
\begin{itemize}
  \item Исследовании решений линейных систем: число решений зависит от ранга матрицы коэффициентов.
  \item Анализе линейной зависимости строк/столбцов.
  \item Проверке обратимости матрицы: квадратная матрица обратима $\iff$ её ранг равен размерности.
\end{itemize}

---

\subsection*{Выводы по теме}

\begin{itemize}
  \item Матрицы — основа линейной алгебры. Они обобщают векторы, храня данные и операции.
  \item Транспонирование меняет строки и столбцы местами.
  \item Ранг показывает, сколько независимых строк/столбцов содержит матрица.
  \item Если ранг меньше полной размерности — значит, матрица "выражает" только подпространство.
\end{itemize}
