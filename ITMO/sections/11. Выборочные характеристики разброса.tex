\section{Выборочные характеристики разброса и центральной тенденции дискретных и непрерывных случайных величин}

\subsection{Введение в выборочные характеристики}
В статистике важную роль играет описание данных с помощью характеристик, которые позволяют сделать выводы о распределении случайной величины на основе выборки.  
Под \textbf{выборочными характеристиками} понимают числовые показатели, вычисленные по данным выборки, которые используются для оценки свойств генеральной совокупности.  

Все характеристики можно условно разделить на две большие группы:
\begin{itemize}
    \item \textbf{Характеристики центральной тенденции} — показывают, вокруг каких значений сосредоточены наблюдения.
    \item \textbf{Характеристики разброса} — показывают, насколько сильно наблюдения отклоняются от центра.
\end{itemize}

\subsection{Центральная тенденция}
Центральная тенденция описывает "середину" данных, то есть значение, около которого сконцентрированы результаты.

\subsubsection{Выборочное среднее}
Выборочное среднее $\overline{x}$ — это сумма всех элементов выборки, делённая на их количество:
\[
\overline{x} = \frac{1}{n} \sum_{i=1}^n x_i
\]
Где:
\begin{itemize}
    \item $n$ — объём выборки
    \item $x_i$ — $i$-е наблюдение
\end{itemize}
Смысл: выборочное среднее — оценка математического ожидания генеральной совокупности.

\subsubsection{Медиана}
Медиана — значение, которое делит упорядоченные данные на две равные части:
\begin{itemize}
    \item Половина значений меньше или равна медиане
    \item Половина — больше или равна медиане
\end{itemize}
Для нечётного $n$: медиана — это значение с индексом $\frac{n+1}{2}$.  
Для чётного $n$: медиана — среднее арифметическое двух средних элементов.

\subsubsection{Мода}
Мода — значение, которое встречается чаще всего.  
Если все значения встречаются одинаково часто, то мода может отсутствовать или быть не единственной.

\subsection{Характеристики разброса}
Разброс характеризует, насколько сильно значения данных отклоняются от среднего.

\subsubsection{Размах}
Размах $R$ — разница между максимальным и минимальным значениями:
\[
R = x_{\max} - x_{\min}
\]
Показывает диапазон значений, но не учитывает их распределение.

\subsubsection{Выборочная дисперсия}
Выборочная дисперсия $S^2$ — среднее квадратов отклонений значений от выборочного среднего:
\[
S^2 = \frac{1}{n-1} \sum_{i=1}^n (x_i - \overline{x})^2
\]
Используем $n-1$ в знаменателе для получения несмещённой оценки дисперсии.

\subsubsection{Выборочное стандартное отклонение}
Стандартное отклонение $S$ — квадратный корень из дисперсии:
\[
S = \sqrt{S^2}
\]
Показывает среднее отклонение данных от среднего значения в тех же единицах, что и сами данные.

\subsubsection{Коэффициент вариации}
Коэффициент вариации $V$ — относительная мера разброса:
\[
V = \frac{S}{\overline{x}} \cdot 100\%
\]
Позволяет сравнивать вариацию в разных выборках, даже если их средние сильно различаются.

\subsection{Дискретные и непрерывные случайные величины}
\subsubsection{Дискретная случайная величина}
Если случайная величина $X$ принимает конечное или счётное множество значений $x_1, x_2, \dots, x_k$, то выборочные характеристики вычисляются по прямым формулам, приведённым выше.

\subsubsection{Непрерывная случайная величина}
Для непрерывных величин выборочные характеристики вычисляются так же, но на практике используют дискретное приближение (интервалы значений).  
При больших объёмах данных гистограммы и интегральные графики помогают визуализировать тенденции.

\subsection{Пример вычислений}
Пусть имеется выборка: $3, 5, 7, 5, 9$.
\begin{itemize}
    \item Выборочное среднее:
    \[
    \overline{x} = \frac{3+5+7+5+9}{5} = \frac{29}{5} = 5.8
    \]
    \item Медиана: $5$ (середина упорядоченного ряда $3, 5, 5, 7, 9$)
    \item Мода: $5$ (встречается дважды)
    \item Размах: $R = 9 - 3 = 6$
    \item Дисперсия:
    \[
    S^2 = \frac{(3-5.8)^2 + (5-5.8)^2 + (7-5.8)^2 + (5-5.8)^2 + (9-5.8)^2}{5-1} = \frac{7.84 + 0.64 + 1.44 + 0.64 + 10.24}{4} = \frac{20.8}{4} = 5.2
    \]
    \item Стандартное отклонение:
    \[
    S = \sqrt{5.2} \approx 2.28
    \]
    \item Коэффициент вариации:
    \[
    V \approx \frac{2.28}{5.8} \cdot 100\% \approx 39.3\%
    \]
\end{itemize}

\subsection{Заключение}
Выборочные характеристики центральной тенденции и разброса — это основа описательной статистики.  
Они позволяют понять структуру данных, выявить закономерности и сравнить разные выборки.  
В дискретных и непрерывных случаях подход к вычислению одинаков, но в непрерывных задачах часто используют аппроксимации и графические методы.
