\section{Аппроксимация и интерполяция функций}

\textbf{Аппроксимация} и \textbf{интерполяция} — это два метода приближённого описания функций, основанные на наборе дискретных точек.

\textbf{Интерполяция} — это построение функции, которая точно проходит через заданные точки.  
\textbf{Аппроксимация} — это построение функции, которая приближённо описывает данные, но может не проходить через все точки.

---

\subsection*{Постановка задачи}

Пусть дана таблица значений:
\[
(x_0, y_0), \ (x_1, y_1), \ \dots, \ (x_n, y_n)
\]
Наша цель — построить функцию $f(x)$ такую, что:
\begin{itemize}
  \item \textbf{Для интерполяции:} $f(x_i) = y_i$ для всех $i$
  \item \textbf{Для аппроксимации:} $f(x_i) \approx y_i$
\end{itemize}

---

\subsection*{Интерполяция: идея и цель}

Интерполяция позволяет восстанавливать значение функции в промежуточных точках, не выходя за пределы интервала $[x_0, x_n]$.

\textbf{Пример:} если известно, что
\[
f(1) = 2, \quad f(2) = 4, \quad f(3) = 6
\]
можем интерполировать $f(x)$, скажем, через многочлен второй степени и вычислить $f(1.5)$.

---

\subsection*{Линейная интерполяция}

Между двумя точками $(x_0, y_0)$ и $(x_1, y_1)$ интерполяционная функция задаётся по формуле:
\[
f(x) = y_0 + \frac{y_1 - y_0}{x_1 - x_0}(x - x_0)
\]

Это уравнение прямой, проходящей через две точки. Очень просто, но недостаточно точно для сложных функций.

---

\subsection*{Полиномиальная интерполяция}

Если заданы $n+1$ точек, можно построить единственный многочлен степени не выше $n$, который проходит через все точки.

\textbf{Формула Лагранжа:}
\[
P_n(x) = \sum_{i=0}^{n} y_i \cdot L_i(x), \quad \text{где } L_i(x) = \prod_{\substack{j=0 \\ j \ne i}}^{n} \frac{x - x_j}{x_i - x_j}
\]

Каждое $L_i(x)$ — базисный многочлен Лагранжа, равный 1 в точке $x_i$ и 0 в остальных $x_j$.

\textbf{Проблема:} при увеличении числа узлов интерполяция может сильно колебаться (эффект Рунге), особенно на концах интервала.

---

\subsection*{Сплайны (кубическая интерполяция)}

\textbf{Сплайн-интерполяция} делит интервал на участки и на каждом строит многочлен степени 3 (кубический сплайн), с условием сглаженности в стыках.

\begin{itemize}
  \item Гладкость первого и второго порядка: $C^2$-непрерывность
  \item Сплайны хорошо подходят для графиков, траекторий и данных с шумом
\end{itemize}

\textbf{Визуально:}
\begin{center}
\begin{tikzpicture}[scale=1.0]
\draw[->] (0,0) -- (7,0) node[right] {$x$};
\draw[->] (0,-1) -- (0,3) node[above] {$y$};

\foreach \x/\y in {1/1, 2/1.5, 3/2.3, 4/1.9, 5/1.5, 6/2}
  \filldraw[blue] (\x,\y) circle (2pt);

\draw[thick,smooth,tension=0.7] plot coordinates {(1,1) (2,1.5) (3,2.3) (4,1.9) (5,1.5) (6,2)};
\end{tikzpicture}
\end{center}

---

\subsection*{Аппроксимация: общая идея}

Аппроксимация применяется, когда функция неизвестна, но имеются измеренные значения с шумом. Здесь уже не требуется точное прохождение через точки.

\textbf{Идея:} найти «наилучшую» функцию $f(x)$, которая \textit{приблизительно} соответствует данным.

Часто ищут функцию в виде:
\[
f(x) = a_0 + a_1 x + a_2 x^2 + \dots + a_n x^n
\]

\subsection*{Аппроксимация методом наименьших квадратов (МНК)}

Пусть есть точки $(x_i, y_i)$, и нужно найти параметры $a_0, a_1, \dots, a_n$, минимизирующие отклонение:
\[
S(a_0, a_1, \dots, a_n) = \sum_{i=0}^n \left(f(x_i) - y_i\right)^2
\]

Минимум достигается при решении системы нормальных уравнений, которая получается из частных производных $S$ по параметрам $a_k$.

\textbf{Частный случай — линейная аппроксимация:}
\[
f(x) = a_0 + a_1 x
\]

Тогда минимизируется:
\[
S(a_0, a_1) = \sum_{i=1}^{n} \left(a_0 + a_1 x_i - y_i\right)^2
\]

Решение:
\[
\begin{cases}
n a_0 + a_1 \sum x_i = \sum y_i \\
a_0 \sum x_i + a_1 \sum x_i^2 = \sum x_i y_i
\end{cases}
\]

---

\subsection*{Сравнение: интерполяция vs аппроксимация}

\begin{tabular}{|c|c|c|}
\hline
Критерий & Интерполяция & Аппроксимация \\
\hline
Проходит через точки & Да & Не обязательно \\
\hline
Чувствительность к шуму & Высокая & Устойчивая \\
\hline
Сложность вычислений & Средняя–высокая & Низкая–средняя \\
\hline
Гладкость & Может не быть & Обычно есть \\
\hline
\end{tabular}

---

\subsection*{Практические применения}

\begin{itemize}
  \item \textbf{Интерполяция}:
    \begin{itemize}
      \item Таблицы и справочники
      \item Заполнение пропущенных значений
      \item Графическая визуализация
    \end{itemize}
  \item \textbf{Аппроксимация}:
    \begin{itemize}
      \item Обработка измерений с шумом
      \item Моделирование реальных процессов
      \item Предсказания, тренды
    \end{itemize}
\end{itemize}

---

\subsection*{Выводы по теме}

\begin{itemize}
  \item Интерполяция позволяет точно восстановить функцию внутри диапазона, но может колебаться.
  \item Аппроксимация — более устойчивая техника, особенно с шумными или неточными данными.
  \item Полиномы Лагранжа и кубические сплайны — мощные методы интерполяции.
  \item Метод наименьших квадратов — классический способ аппроксимации, широко используемый в статистике и машинном обучении.
\end{itemize}
