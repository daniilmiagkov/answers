\subsection{3. Отношения и их свойства}

\subsubsection{3.1. Что такое отношение}

\textbf{Бинарное отношение} $R$ между двумя множествами $A$ и $B$ — это множество упорядоченных пар:
\[
R \subseteq A \times B,
\]
где $A \times B$ — декартово произведение:
\[
A \times B = \{ (a, b) \mid a \in A,\, b \in B \}.
\]

Если $(a, b) \in R$, то говорят, что \textit{$a$ связано с $b$} отношением $R$, и пишут $a\,R\,b$.

\subsubsection{3.2. Примеры}

\begin{itemize}[leftmargin=*]
  \item Отношение \textbf{«меньше»} на $\mathbb{N}$: $R = \{(a, b) \mid a < b\}$.
  \item Отношение \textbf{«быть делителем»} на $\mathbb{N}$: $R = \{(a, b) \mid a \mid b\}$.
  \item Отношение \textbf{«равенство по модулю»} на $\mathbb{Z}$: $a \equiv b \pmod{n}$.
\end{itemize}

\subsubsection{3.3. Область и область значений}

\begin{itemize}[leftmargin=*]
  \item \textbf{Область определения (domain)}:
  \[
  \operatorname{dom}(R) = \{ a \in A \mid \exists b \in B\colon (a,b) \in R \}.
  \]
  \item \textbf{Область значений (range)}:
  \[
  \operatorname{ran}(R) = \{ b \in B \mid \exists a \in A\colon (a,b) \in R \}.
  \]
\end{itemize}

\subsubsection{3.4. Свойства бинарных отношений (на $A \times A$)}

Пусть $R \subseteq A \times A$. Тогда отношение может обладать следующими свойствами:

\begin{itemize}[leftmargin=*]
  \item \textbf{Рефлексивность:}
  \[
  \forall a \in A\colon (a,a) \in R.
  \]
  Пример: $=$, $\le$.

  \item \textbf{Антирефлексивность (иррефлексивность):}
  \[
  \forall a \in A\colon (a,a) \notin R.
  \]
  Пример: $<$.

  \item \textbf{Симметричность:}
  \[
  \forall a,b \in A\colon (a,b) \in R \Rightarrow (b,a) \in R.
  \]
  Пример: «$a$ и $b$ живут в одном доме».

  \item \textbf{Антисимметричность:}
  \[
  \forall a,b \in A\colon (a,b)\in R \wedge (b,a)\in R \Rightarrow a = b.
  \]
  Пример: $\le$.

  \item \textbf{Транзитивность:}
  \[
  \forall a,b,c \in A\colon (a,b)\in R \wedge (b,c)\in R \Rightarrow (a,c)\in R.
  \]
  Пример: $\le$, $<$.
\end{itemize}

\subsubsection{3.5. Особые классы отношений}

\begin{itemize}[leftmargin=*]
  \item \textbf{Отношение эквивалентности} — рефлексивное, симметричное и транзитивное.  
  Пример: $a \equiv b \pmod{n}$.

  Такое отношение разбивает множество $A$ на \textit{классы эквивалентности}.

  \item \textbf{Отношение частичного порядка} — рефлексивное, антисимметричное и транзитивное.  
  Пример: $\le$ на $\mathbb{N}$.

  Если дополнительно выполняется, что любые два элемента сравнимы, то это \textbf{полный порядок}.
\end{itemize}

\subsubsection{3.6. Графическое представление}

Бинарное отношение на множестве $A$ можно представить в виде \textbf{ориентированного графа}:

\begin{itemize}[leftmargin=*]
  \item Вершины соответствуют элементам $A$.
  \item Направленное ребро $a \to b$ рисуется, если $(a,b) \in R$.
\end{itemize}

Пример: на множестве $A = \{1, 2, 3\}$ отношение $R = \{(1,2), (2,3), (1,3)\}$ — транзитивное.

\subsubsection{3.7. Табличное представление}

Отношение $R$ на множестве $A = \{a_1, a_2, \dots, a_n\}$ можно представить в виде \textbf{таблицы}, где в ячейке на пересечении строки $i$ и столбца $j$ стоит $1$, если $(a_i, a_j) \in R$, и $0$ — иначе. Это называется \textbf{матрицей смежности}.

\subsubsection{Источники}

\begin{itemize}
  \item Г.С. Михалев, \textit{Дискретная математика}.
  \item Р. Джонсонбауг, \textit{Дискретная математика}, Pearson Education.
  \item \href{https://ru.wikipedia.org/wiki/Бинарное_отношение}{Википедия: Бинарное отношение}
\end{itemize}