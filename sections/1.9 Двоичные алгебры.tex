\subsection{9. Двоичные алгебры}

\subsubsection{9.1. Понятие двоичной (бинарной) операции}

\begin{definition}
Пусть $A$ — непустое множество. \emph{Двоичной операцией} на $A$ называется отображение
\[
  * : A \times A \;\longrightarrow\; A,
  \quad (x,y)\mapsto x*y.
\]
\end{definition}

\emph{Интуиция:} берём два элемента из $A$, «складываем» их по правилу $*$ и получаем снова элемент из $A$.

\subsubsection{9.2. Свойства двоичной операции}

Пусть $*$ — двоичная операция на $A$. Говорят, что $*$ обладает свойствами:

\begin{itemize}[leftmargin=*]
  \item \textbf{Замкнутость}: по определению $x*y\in A$ для любых $x,y\in A$.
  \item \textbf{Ассоциативность}:
    \[
      (x*y)*z = x*(y*z),\quad \forall x,y,z\in A.
    \]
    Позволяет не ставить скобок при многократном применении.
  \item \textbf{Коммутативность}:
    \[
      x*y = y*x,\quad \forall x,y\in A.
    \]
  \item \textbf{Нейтральный (единичный) элемент}: существует $e\in A$ такое, что
    \[
      e*x = x*e = x,\quad \forall x\in A.
    \]
    Его часто обозначают $0$ или $1$ в зависимости от контекста.
  \item \textbf{Обратимые элементы}: элемент $x\in A$ называется обратимым, если существует $y\in A$ такой, что
    \[
      x*y = y*x = e.
    \]
    Тогда $y$ называют \emph{обратным} к $x$ и обозначают $x^{-1}$.
\end{itemize}

\subsubsection{9.3. Классификация двоичных алгебр}

\begin{enumerate}[label=\arabic*)]
  \item \textbf{Магма}: $(A,*)$ — любое множество с двоичной операцией (требуется лишь замкнутость).
  \item \textbf{Полугруппа}: магма с ассоциативной операцией.
  \item \textbf{Моноид}: полугруппа, в которой есть единица $e$.
  \item \textbf{Группа}: моноид, в котором каждый элемент обратим.
  \item \textbf{Абелева (коммутативная) группа}: группа с коммутативным $*$.
\end{enumerate}

\subsubsection{9.4. Примеры}

\begin{enumerate}[label=\arabic*)]
  \item $(\mathbb{Z}, +)$ — абелева группа, где единица $0$, обратный к $x$ есть $-x$.
  \item $(\mathbb{N}, +)$ — моноид (нет обратных элементов, кроме $0$).
  \item $(\{0,1\}, \wedge)$ — коммутативная монода, где $0\wedge1=0$, единица $1$.
  \item $(\{0,1\}, \oplus)$ (сумма по модулю 2) — абелева группа:  
    \[
      0\oplus0=0,\quad 0\oplus1=1,\quad1\oplus1=0;
      \quad e=0,\;x^{-1}=x.
    \]
  \item $(M_n(\mathbb{R}), \cdot)$ — полугруппа матриц; моноид при наличии единичной матрицы.
\end{enumerate}

\subsubsection{9.5. Таблица Кэли}

Для конечных алгебр удобно задавать операцию таблицей.  
\emph{Пример:} группа $(\{0,1\},\oplus)$:

\[
\begin{array}{c|cc}
\oplus & 0 & 1 \\ \hline
0 & 0 & 1 \\
1 & 1 & 0
\end{array}
\]

\subsubsection{9.6. Связь с булевыми алгебрами}

Булева алгебра — это \emph{расширенная} коммутативная группа с дополнительными операциями «и», «или» и «не» на множестве $\{0,1\}$.  
В частности, структура $(\{0,1\},\wedge,\vee,\neg)$ удовлетворяет ряду аксиом идемпотентности и дистрибутивности.

\subsubsection{9.7. Зачем нужны двоичные алгебры?}

\begin{itemize}[leftmargin=*]
  \item Моделирование и анализ абстрактных операций (сложение, умножение, логические связки).
  \item Основа теории групп и её приложений: симметрии, криптография, теории кодирования.
  \item В информатике: операции над битами, булевы функции, конечные автоматы.
\end{itemize}

\subsubsection{Источники}

\begin{itemize}
  \item С.\,Ланг, \emph{Алгебра}.
  \item Д.\,С. Джонсонбауг, \emph{Дискретная математика}, Pearson.
  \item \href{https://ru.wikipedia.org/wiki/Бинарная_операция}{Википедия: Бинарная операция}.
\end{itemize}